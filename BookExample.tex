\documentclass[a5paper,openany]{book}

\usepackage{cmap}  
\usepackage[utf8]{inputenc}
\usepackage[T2A]{fontenc} 
\usepackage{index} 
\usepackage[russian]{babel} 
\usepackage{amsmath,amssymb} 
\usepackage{euscript,upref}  
\usepackage{array,longtable}
\usepackage{indentfirst} 
\usepackage{graphicx} 
%\usepackage{caption} 
\usepackage[justification=centering]{caption}
%\usepackage{calrsfs} 
%\usepackage{url}
\usepackage{multirow,makecell,array}
%\usepackage{setspace} 
%\usepackage{todonotes}
%\usepackage{calligra}
%\usepackage{makeidx}
\usepackage{pgf,tikz}
\usepackage{pgfplots}
\usepackage{pgfplotstable}
\usepackage{etoolbox}

%%%%%%%%%%%%%%%%%%%%%%%%%%%%%%%%%%%%%%%%%%%%%%%%%%%%%%%%%%%%%%%%%%%%%%%%%%%%%%%%%%%%%%%%
%
%           Определения новых команд и макросов
%    
%\DeclareMathAlphabet{\mathcalligra}{T1}{calligra}{m}{n}
%\DeclareFontShape{T1}{calligra}{m}{n}{<->s*[1.8]callig15}{}
\newcommand{\mbf}[1]{\protect\text{\boldmath$#1$}}
\newcommand{\mbb}{\mathbb}
\newcommand{\mrm}{\mathrm}
\newcommand{\mcl}{\mathcal}
\newcommand{\msf}{\mathsf}
\newcommand{\eus}{\EuScript}
\newcommand{\ov}{\overline}
\newcommand{\un}{\underline}
\newcommand{\m}{\mathrm{mid}\;}
\newcommand{\w}{\mathrm{wid}\;}
\newcommand{\Uni}{\mathrm{Uni}\,}
\newcommand{\Tol}{\mathrm{Tol}\,} 
\newcommand{\Uss}{\mathrm{Uss}\,} 
\newcommand{\Ab}{(\mbf{A}, \mbf{b})}
\newcommand{\Arg}{\mathrm{Arg}\;} 
\newcommand{\sgn}{\mathrm{sgn}\;} 
\newcommand{\ran}{\mathrm{ran}\,} 
\newcommand{\pro}{\mathrm{pro}\,} 
\newcommand{\dom}{\mathrm{dom}\,} 
\newcommand{\IVE}{\mathrm{IVE}\,} 
\newcommand{\IED}{\mathrm{IED}\,} 
\newcommand{\calX}{\mathrsfs{X}} 
\newcommand{\cond}{\mathrm{cond}} 
\newcommand{\mode}{\mathrm{mode}\,} 
\newcommand{\dual}{\mathrm{dual}\,} 
\newcommand{\dist}{\mathrm{dist}\,} 
\newcommand{\Dist}{\mathrm{Dist}\,} 
\newcommand{\const}{\mathrm{const}} 
\newcommand{\USS}{\Xi_{\hspace{-0.5pt}uni}} 
\newcommand{\TSS}{\Xi_{\hspace{-0.5pt}tol}} 
\newcommand{\NExt}{_{\scalebox{0.57}{$\natural$}}}
\newcommand{\ih}{\scalebox{0.67}[0.87]{$\Box$\hspace*{1pt}}}

\renewcommand{\r}{\mathrm{rad}\;} 
\renewcommand{\vert}{\mathrm{vert}\,} 
\newcommand{\md}{\operatorname{med}}

%%%%%%%%%%%%%%%%%%%%%%%%%%%%%%%%%%%%%%%%%%%%%%%%%%%%%%%%%%%%%%%%%%%%%%%%%%%%%%%%%%%%%%%%%%%%%%%%%%%%%

\newenvironment{example}[1]
{%	\par %\vspace{\baselineskip}
	\refstepcounter{Examp}	
	%	{\noindent\textbf{Пример~\theExamp~(#1)} 
	{\textbf{Пример~\theExamp.~#1} 					
		\protect\addcontentsline{exp}{example}{\protect\numberline{\theExamp}\hspace{10pt}~#1}}
	%{\protect\numberline{\listOfExamples}}	
	%	{\noindent\ignorespaces}
}
%{\hfill$\blacksquare$\par\addvspace{3ex}}
%{\hfill\par\addvspace{3ex}}
%


%%% Создание списка собственной переменной окружения
\newcommand{\listOfExamples}{Список примеров}
%\newlistof{example}{exp}{\listOfExamples} 
\newcounter{Examp}
\setcounter{Examp}{0}
\renewcommand\theExamp{\thesection.\arabic{Examp}} %Эта команда - макрос для счетчика Examp, чтобы каждый раз не писать длинное выражение 
%Если нужна сквозная нумерация примеров:
%\renewcommand\theExamp{\arabic{Examp}}

%\captionsetup[table]{labelformat=empty}
%\newcommand{\TABLENAME}{\raggedleft\small\textit{\tablename}\hspace{1mm}}

%%%%%%%%%%%%%%%%%%%%%%%%%%%%%%%%%%%%%%%%%%%%%%%%%%%%%%%%%%%%%%%%%%%%%%%%%%%%%%%%%%%%%%%%%%%%%%%%%%%%%


%%%%%%%%%%%%%%%%%%%%% GREEK TEXT %%%%%%%%%%%%%%%%%%%%%%%%%%%%%%%%%%%%%%%%
\usepackage{textgreek}
\renewcommand{\mu}{\text{\textmu}} 
\renewcommand{\delta}{\text{\textdelta}} 
\renewcommand{\Delta}{\text{\textDelta}} 
\renewcommand{\alpha}{\text{\textalpha}} 
\renewcommand{\beta}{\text{\textbeta}} 
\renewcommand{\epsilon}{\text{\textepsilon}}
\renewcommand{\rho}{\text{\textrho}}
\renewcommand{\gamma}{\text{\textgamma}}
%%%%%%%%%%%%%%%%%%%%% GREEK TEXT %%%%%%%%%%%%%%%%%%%%%%%%%%%%%%%%%%%%%%%%



\usepackage{imakeidx}
\makeindex [title=ПРЕДМЕТНЫЙ УКАЗАТЕЛЬ]

%%%%%%%%%%%%%%%%%%%%%%%%%%%%% CHAPTER %%%%%%%%%%%%%%%%%%%%%%%%%%%%%
% https://latex.org/forum/viewtopic.php?f=5&t=2521#p9922
\usepackage[compact]{titlesec} 
\titleformat{\chapter}{%
	%	\centering\normalfont\Large\bfseries}{\thechapter}{1em}{}
\centering\normalfont\Large\bfseries}{ {\large  {\tt Г~л~а~в~а~ \thechapter}} }{1em}{}
\titlespacing*{\chapter} {0pt}{3.5ex plus 1ex minus .2ex}{2.3ex plus .2ex}
%  \titleformat{\section}{%
%	\centering\normalfont\Large\bfseries}{\thesection}{1em}{}
%%%%%%%%%%%%%%%%%%%%%%%%%%%%% CHAPTER %%%%%%%%%%%%%%%%%%%%%%%%%%%%%

%%%%%%%%%%%%%%%%%%%%%%%%%%%%% SECTION %%%%%%%%%%%%%%%%%%%%%%%%%%%%%
\titleformat{\section}
{\centering\Large\bfseries}{\thesection.}{1em}{}
\titleformat{\subsection}
{\centering\large\bfseries\itshape}{\thesubsection.}{1em}{}
%%%%%%%%%%%%%%%%%%%%%%%%%%%%% SECTION %%%%%%%%%%%%%%%%%%%%%%%%%%%%%	

%%%%%%%%%%%%%%%%%%%%%%%%%%%%% SECTION  Adds . in TOC %%%%%%%%%%%%%%
\usepackage{secdot}% Adds . after sectional unit numbers
\usepackage{etoolbox}
% \patchcmd{<cmd>}{<search>}{<replace>}{<success>}{<failure>}
\patchcmd{\numberline}{\hfil}{.\hfil}{}{}
%%%%%%%%%%%%%%%%%%%%%%%%%%%%% SECTION  Adds . in TOC %%%%%%%%%%%%%%	

%%%%%%%%%%%%%%%%%%%%%%%%%%%%% PAGE SIZE %%%%%%%%%%%%%%	
\textwidth=114truemm
\textheight=165truemm
\oddsidemargin=-1cm
\evensidemargin=\oddsidemargin
%\topmargin=-1cm
\topmargin=-2cm
\sloppy

\pagestyle{plain}
%\mathsurround=1pt
%\tolerance=400
%\hfuzz=2pt
%%%%%%%%%%%%%%%%%%%%%%%%%%%%% PAGE SIZE %%%%%%%%%%%%%%	

%%%%%%%%%%%%%%%%%%%%%%%%%%%%% PAGE NUMBERS %%%%%%%%%%%%%%%%%%%%%%%%%%%%%%
\usepackage{fancyhdr}
\fancyhf{} % clear all header and footers
\renewcommand{\headrulewidth}{0pt} % remove the header rule
\fancyfoot[LE,RO]{\thepage} % Left side on Even pages; Right side on Odd pages
\pagestyle{fancy}
\fancypagestyle{plain}{%
\fancyhf{}%
\renewcommand{\headrulewidth}{0pt}%
\fancyhf[lef,rof]{\thepage}%
}
%%%%%%%%%%%%%%%%%%%%%%%%%%%%% PAGE NUMBERS %%%%%%%%%%%%%%%%%%%%%%%%%%%%%%


%%%%%%%%%%%%%%%%%%%%%%%%%%%%%%%%%%%%%%%%%%%%%%%%%%%%%%%%%%%%%%%%%%%%%%%%%%%%%%%%%%%%%%%%%%%%%%%%%%%%%%%%%%%%
\usepackage{enumerate} % Тонкая настройка списков
\usepackage{indentfirst} % Красная строка после заголовка
\usepackage{float} % Расширенное управление плавающими объектами


%%%%%%%%%%%%%%%%%%%%%%% ОГЛАВЛЕНИЕ %%%%%%%%%%%%%%%%%%%%%%%
%\renewcommand{\contentsname}{ОГЛАВЛЕНИЕ}

\usepackage{tocloft} % Alter the style of the Table of Contents
% 2022-12-08
\renewcommand{\cftsecfont}{\rmfamily\mdseries\upshape}
%\renewcommand{\cftsecpagefont}{\rmfamily\mdseries\upshape} % No bold!

% Содержание
% ОГЛАВЛЕНИЕ
%\renewcommand{\cfttoctitlefont}{\normalfont\MakeUppercase}
%\renewcommand{\cfttoctitlefont}{\hspace*{\fill}\normalfont\MakeUppercase}
\renewcommand{\cfttoctitlefont}{\hspace{4cm}\normalfont\bfseries\MakeUppercase}
%\renewcommand{\cfttoctitlefont}{\centering\normalfont\bfseries\MakeUppercase}
%\renewcommand{\cftaftertoctitle}{\hfill}
% СОДЕРЖАНИЕ
%\renewcommand{\cftsecfont}{\hspace{0pt}}            % Имена секций в содержании не жирным шрифтом
\renewcommand\cftsecleader{\cftdotfill{\cftdotsep}} % Точки для секций в содержании
\renewcommand\cftsecpagefont{\mdseries}             % Номера страниц не жирные
\setcounter{tocdepth}{3}                            % Глубина оглавления, до subsubsection
\renewcommand{\cftsecaftersnum}{.}					% Точки после номера секций в содержании
\renewcommand{\cftsubsecaftersnum}{.}				% Точки после номера подсекций в содержании
%

%\addto\captionsenglish{\renewcommand{\chaptername}{Lecture}}
\addto\captionsrussian{\renewcommand{\chaptername}{Г~л~а~в~а}}
%\renewcommand{\chaptername}{Раздел}
\renewcommand{\cftchappagefont}{\normalfont}
%\renewcommand{\cftchapfont}{\normalfont\large\itshape}    % \chapter font in ToC
%\renewcommand{\cftchapfont}{\bfseries\large}    % \chapter font in ToC
\renewcommand{\cftchapfont}{\bfseries}    % \chapter font in ToC

\setlength{\cftchapnumwidth}{0pt}
%\renewcommand{\cftchappresnum}{\MakeUppercase\chaptername\ }
\renewcommand{\cftchappresnum}{\normalfont\chaptername\ }
\renewcommand{\cftchapaftersnum}{.}						% Точка после номера главы в содержании
%\renewcommand{\cftchapaftersnumb}{\newline}
\renewcommand{\cftchapaftersnumb}{~~~~~~~~~~~~~}
\renewcommand{\cftchapdotsep}{\cftdotsep}
\renewcommand\cftchapleader{\cftdotfill{\cftdotsep}}	% Точки после имени главы в содержании
%%%%%%%%%%%%%%%%%%%%%%%  PART %%%%%%%%%%%%%%%%%%%%%%%%%%%%%%%%%%%%%%%%%%%%%%%%%%%%%%%%%%
% для нестандартного оформления оглавления
\renewcommand{\cftpartpagefont}{\normalfont}   		 % номер страницы
\renewcommand{\cftpartfont}{\normalfont} 			 % шрифт раздела
\renewcommand\cftpartleader{\cftdotfill{\cftdotsep}} % Точки 
%%%%%%%%%%%%%%%%%%%%%%%  PART %%%%%%%%%%%%%%%%%%%%%%%%%%%%%%%%%%%%%%%%%%%%%%%%%%%%%%%%%%

%%%%%%%%%%%%%%%%%%%%%%% ОГЛАВЛЕНИЕ %%%%%%%%%%%%%%%%%%%%%%%

%%%%%%%%%%%%%%%%%%%%%%%%%%%%%%%%  Списки %%%%%%%%%%%%%%	
\usepackage{enumitem}
\setlist[enumerate,itemize]{leftmargin=0pt,itemindent=2.5em} % Отступы в списках
\makeatletter
\AddEnumerateCounter{\asbuk}{\@asbuk}{м)}
\makeatother
\setlist{nolistsep}
\renewcommand{\labelitemi}{--}
\renewcommand{\labelenumi}{\asbuk{enumi})}
\renewcommand{\labelenumii}{\arabic{enumii})}
%%%%%%%%%%%%%%%%%%%%%%%%%%%%%%%%  Списки %%%%%%%%%%%%%%	


%%%%%%%%%%%%%%%%%%%%%%%%%%%%%%%%%%%%%%%%%%%%%%%%%%%%%%%%%%%%%%%%%%%%%%%%%%%%%%%%%%%%%%%%%%%%%%%%%%%%%%%%%%%%


%%%%%%%%%%%%%%%%%%%%%%%%%%%%% CAPTIONS TABLES FIGURES %%%%%%%%%%%%%%%%%%%%%%%%%%%%%%
\captionsetup{font=small,labelsep=period,margin=7mm} 
%%%%%%%%%%%%%%%%%%%%%%%%%%%%% CAPTIONS %%%%%%%%%%%%%%%%%%%%%%%%%%%%%
\captionsetup[figure]{
	font = small,
	labelfont = it
}
\captionsetup[table]{
	font = small,
	labelfont = it
}	
%\usepackage{lipsum, mwe}
\setlength\abovecaptionskip{0pt plus 0pt minus 4pt}
\setlength\belowcaptionskip{0pt plus 0pt minus 4pt}	

\captionsetup[table]{labelformat=empty}
\newcommand{\TABLENAME}{\raggedleft\small\textit{\tablename}\hspace{1mm}}
%%%%%%%%%%%%%%%%%%%%%%%%%%%%% CAPTIONS TABLES FIGURES %%%%%%%%%%%%%%%%%%%%%%%%%%%%%%

%%%%%%%%%%%%%%%%%%%%%%%%%%%%%%% thebibliography %%%%%%%%%%%%%%%%%%%%%%%%%%%%%%% 
\makeatletter
\renewcommand\@biblabel[1]{#1.}
\makeatother

%%%%%%%%%%%%%%%% Библиография: отступы и межстрочный интервал %%%%%%%%%%%%%%%% 
\makeatletter
\renewenvironment{thebibliography}[1]
{\section*{\refname}
\list{\@biblabel{\@arabic\c@enumiv}}
{\settowidth\labelwidth{\@biblabel{#1}}
	\leftmargin\labelsep
	\itemindent 10mm %16.7mm
	\@openbib@code
	\usecounter{enumiv}
	\let\p@enumiv\@empty
	\renewcommand\theenumiv{\@arabic\c@enumiv}
}
\setlength{\itemsep}{0pt}
}
\makeatother
%%%%%%%%%%%%%%%% Библиография: отступы и межстрочный интервал %%%%%%%%%%%%%%%% 

%%%%%%%%%%%%%%%%%%%%%%%%%%%%%% СКОБКИ СВЕТЛЫЕ %%%%%%%%%%%%%%%%%%%%%%%%%%%%%%
%\usepackage{graphicx}
\usepackage{environ}
\NewEnviron{thincases}{\scalebox{0.5}[1]{$\displaystyle
	\left\{\scalebox{2}[1]{\setlength{\arraycolsep}{3pt}% <- I did not look up the "correct" value О	
		$\displaystyle\begin{array}{ll}
			\BODY
		\end{array}$}\right.$}}%}
%%%%%%%%%%%%%%%%%%%%%%%%%%%%%% СКОБКИ  СВЕТЛЫЕ %%%%%%%%%%%%%%%%%%%%%%%%%%%%%%



	\begin{document}
		%\maketitle
		
%%%%
				\newpage
				\tableofcontents
				
				\newpage
				
				%\listofexample 
				
				%\cleardoublepage
				%\newpage
				
				%\listoffigures
				
				%\newpage
				
%	\section*{ВВЕДЕНИЕ}	
	\begin{center}
		ВВЕДЕНИЕ
	\end{center}
%		\part*{ВВЕДЕНИЕ}				
%	\addcontentsline{toc}{chapter}{Введение}   
	\addcontentsline{toc}{part}{Введение}  
					
				Учебное пособие является дополнением к книге коллектива авторов
				<<Обработка и анализ данных с интервальной неопределенностью>>.
				
				
				Задача пособия --- дать обучащимся краткие сведения о теории и рассмотреть ряд примеров, иллюстрирующих интервальный подход. 


\chapter[Обзор методов статистики и обработки данных]% 
{\\Обзор методов \\ статистики и обработки данных} 

	\section{Данные, погрешности и их обработка} 


На практике данные не бывают точными. В действительности нам известно приближенное
значение измеряемой величины, а также некоторая информация (качественная 
и количественная) о погрешности этого значения.  \index{неопределенность} 
На результаты измерений могут оказывать влияние изменчивость измеряемых величин, 
их непостоянство во времени или пространстве. На измерения могут влиять внешние неконтролируемые факторы, так 
называемые <<шумы>>. 

Согласно <<Малой Советской Энциклопедии>> \cite{MalSovEncycl},  \emph{статистика} 
--- это <<отрасль знаний, в которой излагаются общие вопросы сбора, измерения и 
анализа массовых статистических (количественных или качественных) данных \ldots>>. 

\subsection{Статистика нечетких данных} 
\label{FuzzyStatSect} 


Элементами арифметики $\mbb{KR}$ являются пары чисел, взятые в квадратные скобки,  \text{\textalpha}

вида $[\text{\textalpha}, \beta]$, которые будем называть \textit{интервалами}. 
При этом возможны ситуации, когда $\text{\textalpha}\leq\beta$ или $\text{\textalpha} > \beta$. Если 
$\text{\textalpha}\leq \beta$, то $[\text{\textalpha}, \beta]$ обозначает обычный интервал вещественной оси, 
его называют \textit{правильным}. Если же $\text{\textalpha} >  \beta$, то $[\text{\textalpha}, \beta]$ 
--- \textit{неправильный интервал}. 


\emph{Нечетким множеством} %(см. \cite{DuboisPrade,Zadeh}) 
называется множество $X$, 
образованное элементами произвольной природы, которое дополнено так называемой 
\emph{функцией принадлежности} $\mu: X\to[0, 1]$, значение которой \textmu(x) на элементе 
$x\in X$ показывает степень принадлежности $x$ множеству $X$. 
У стандартной функции принадлежности множества (называемой также \emph{индикаторной функцией} 
множества) значения могут быть равны только 0 или 1, поэтому допущение для функции $\text{\textmu}$ 
непрерывного ряда значений из интервала $[0, 1]$ позволяет характеризовать ситуации, когда 
нет уверенности в принадлежности элемента множеству, невозможно оперировать количественной мерой \index{нечеткое множество} \index{функция принадлежности} 
уверенности и строить на этой основе  выводы и заключения.

\begin{equation}
	\label{MeasurDiffsAbs} 
	\Delta_k = | \mathring{x}_k  - x_c |,  \quad  k = 1, \ \ldots, \ n.  
\end{equation}  


\chapter[Базовые понятия и математический аппарат]% 
{\\Базовые понятия и \\* математический аппарат} 
\label{PrimaryConceptChap} 


Предметом этой главы книги является введение базовых понятий и математического 
аппарата, относящихся к обработке интервальных данных.   

	Графическое представление твина $\mbf{T}$  дано на рис. \ref{TwinsTemp}. 
\begin{figure}[ht]
	\centering\small 
	\setlength{\unitlength}{1mm}
	\begin{picture}(70,17)
		\put(0,0){\includegraphics[width=70mm]{TwinTemp}}
		\put(-5,6.6){\vector(1,0){80}} \put(71.5,7.6){$\mbb{R}$} 
		\put(18,10){{\footnotesize 19,15}} 
		\put(28,10){{\footnotesize 19,85}} 
		\put(37,12){{\footnotesize 20,15}} 
		\put(47,12){{\footnotesize 20,85}} 	
		\put(35,1){$\mbf{T}$}  
	\end{picture}
	\caption{Температура как твин.} 
	\label{TwinsTemp} 
\end{figure}
Форма записи температуры в виде твина $\mbf{T}$ четко
и полно представляет информацию об измеряемых данных.

%%%%%%%%%%%%%%%%%%%%%%%%%%%%%%%%%%%%%%%%%%%%%%%%%%%%%%%%%%%%%%%%%%%%%%%%%%%%%%%%%%%%%%%% 

\section{Интервалы} 
\label{IntervalSect} 


\paragraph{Вещественные интервалы.} 
Первичное понятие интервального анализа --- \emph{интервал}. Это просто описываемое 
множество, задающее целый диапазон значений интересующей нас величины, с помощью 
которого можно рассматривать неопределённости и неоднозначности. 


\begin{equation*} 
	\langle\mbf{a}\rangle\  = \;\min\,\{  |a| \mid a \in \mbf{a} \} \ 
	= 		\left\{ 	
	\begin{array}{cl}
		\min\,\{  |\un{\mbf{a}}|, |\ov{\mbf{a}}|  \}, & \text{ если } 0\not\in\mbf{a},\\[2mm]  
		0, & \text{ если } 0\in\mbf{a}. 
	\end{array} 
\right. 
\end{equation*}

\begin{equation*} 
	\langle\mbf{a}\rangle\  = \;\min\,\{  |a| \mid a \in \mbf{a} \} \ = 	
\begin{thincases}
	\begin{array}{cl}
	\min\,\{  |\un{\mbf{a}}|, |\ov{\mbf{a}}|  \}, & \text{ если } 0\not\in\mbf{a},\\[2mm]  
	0, & \text{ если } 0\in\mbf{a}. 
	\end{array}
\end{thincases}
\end{equation*}

\begin{equation*} 
	\pro\mbf{a} = 
	\left\{ 
	\begin{array}{cl}
		\!	\mbf{a}, & \text{если $\mbf{a}$ --- правильный,} \\[1mm] 
		\!	\dual\mbf{a}, & \text{если $\mbf{a}$ --- неправильный.} 
	\end{array} 
	\right. 
\end{equation*} 

\begin{equation*} 
\pro\mbf{a} = 	
	\begin{thincases}
	\begin{array}{cl}
		\!	\mbf{a}, & \text{если $\mbf{a}$ --- правильный,} \\[1mm] 
		\!	\dual\mbf{a}, & \text{если $\mbf{a}$ --- неправильный.} 
	\end{array} 
	\end{thincases}
\end{equation*}

\chapter[Измерение постоянной величины]%
{\\Измерение постоянной величины} 
\label{MeasrConstChap}

%%%%%%%%%%%%%%%%%%%%%%%%%%%%%%%%%%%%%%%%%%%%%%%%%%%%%%%%%%%%%%%%%%%%%%%%%%%%%%%%%%%%%%%%  

\section[Выборка измерений и интервалы их неопределённости]% 
{Выборка измерений \\*  и интервалы их неопределённости} 
\label{MeasSetInte}   

Напомним, что \emph{постоянная величина} --- это величина, которая в рассматриваемом 
процессе сохраняет свое значение неизменным. К примеру, рост человека не меняется 
заметно в процессе его измерения, и потому может считаться постоянной величиной, 
хотя на протяжении жизни человека рост, конечно же, непостоянен. 
\index{постоянная величина} 

Пусть имеется выборка измерений некоторой величины, 
\begin{equation}
	\label{ISample} 
	\mbf{x}_{1}, \mbf{x}_{2}, \ldots, \mbf{x}_{n}, 
\end{equation}                                 
или, кратко, $\{\,\mbf{x}_{k}\}_{k=1}^n$, где $k$ --- номер измерения, $\mbf{x}_k$ 
--- интервальный результат измерения, полученный, к примеру, какой-либо из процедур, 
описанных в предыдущих параграфах. Таким образом, согласно терминологии интервального 
анализа, рассматриваемая выборка --- это вектор интервалов, или интервальный вектор 
$\mbf{x} = (\mbf{x}_{1}, \mbf{x}_{2}, \ldots, \mbf{x}_{n})$. Число $n$ --- размерность 
вектора данных --- будем, как обычно, называть \emph{длиной выборки} (или объёмом 
выборки).\index{длина выборки} По интервальным результатам измерений или наблюдений 
требуется построить оценку для интересующей нас величины. 
\index{задача измерения постоянной величины} 

%%%%%%%%%%%%%%%%%%%%%%%%%%%%%%%%%%%%%%%%%%%%%%%%%%%%%%%%%%%%%%%%%%%%%%%%%%%%%%%%%%%%%%%% 

\chapter[Задача восстановления зависимостей]%
{\\Задача восстановления зависимостей} 
\label{FuncFitChap}


В этой главе даются определения новых понятий и терминов, которые возникают в связи 
с восстановлением функциональных зависимостей по данным их измерений и наблюдений, 
имеющих интервальную неопределённость. Мы рассмотрим также основные идеи и типичные 
приёмы восстановления зависимостей по интервальным данным, а также возникающие 
при этом проблемы. 


%\tablename~\thetable
%\TABLENAME \ref{TableWidbeta}
%\floatbox[\capbeside]{table}{\caption{\\Размеры множества параметров линейной модели данных}
\begin{table}[h!]
	\TABLENAME \ref{TableWidbeta}
	\caption{\\ {\bfseries\small Размеры множества параметров линейной модели данных}}
	\label{TableWidbeta}
	\begin{center}
		\begin{tabular}{ | c | c | c | }
			\hline
			{\small	$\epsilon$} & {\small$\w \mbf{\beta}_1$} & {\small $\w \mbf{\beta}_2$} \\
			\hline
			~~~~~~~~~~~100~~~~~~~~~~~~	 & ~~~~~~~~~~~~$\simeq$ 29~~~~~~~~~~~~ & ~~~~~~~~~~~~$ \simeq$ 4~~~~~~~~~~~~ \\
			\hline				
			150	 & $\simeq$ 250 & $\simeq$ 38\\
			\hline
		\end{tabular}
	\end{center}
	\vspace{-4mm}
\end{table}


Более подробно исследуется случай простейшей линейной зависимости, 
но большинство построений и рассуждений легко переносятся на общий нелинейный случай. 
%%%%%%%%%%%%%%%%%%%%%%%%%%%%%%%%%%%%%%%%%%%%%%%%%%%%%%%%%%%%%%%%%%%%%%%%%%%%%%%%%%%%%%%% 

\section{Постановка задачи} 

Предположим, что величина $y$ является функцией некоторого заданного вида от 
независимых переменных $x_1$, $x_2$, \ldots, $x_m$, т.\,е. 
\begin{equation}
	\label{ParamFunc} 
	y = f(x, \beta), 
\end{equation}
где $x = (x_{1}, \ldots, x_{m})$ --- вектор независимых переменных, $\beta = (\beta_1, 
\ldots, \beta_l)$ --- вектор параметров функции. Имея набор значений переменных $x$ 
и $y$, нам нужно найти $\beta_1, \ldots, \beta_l$, которые соответствуют конкретной 
функции $f$ из параметрического семейства \eqref{ParamFunc}. Эту задачу мы будем 
называть \emph{задачей восстановления зависимости}, и она будет основным предметом 
рассмотрения в этой главе.                \index{задача восстановления зависимости} 


посредством решения задачи условной оптимизации. Найти
\begin{equation} 
	\label{MinSumW_Subj} 
	%\text{Найти} \quad 
	\min_{w,\, \beta}\;\Sigma_{i \, = \, 1}^n w_{i} 
\end{equation}
при ограничениях
\begin{equation} 
	\label{MinSumW_Constr} 
	\left\{  
	\begin{gathered}
		\mathring{y}_i - w_i \, \epsilon_i \leq f(x_i,\beta) 
		\leq \mathring{y}_i + w_i \, \epsilon_i,    \\[2pt]   
		w_i \geq 1, 
	\end{gathered}
	\qquad 
	i = 1, \ \ \ldots, \ n. 
	\right. 
\end{equation}

\begin{equation*} 
	\begin{thincases}
	\begin{gathered}
	\mathring{y}_i - w_i \, \epsilon_i \leq f(x_i,\beta) 
	\leq \mathring{y}_i + w_i \, \epsilon_i,    \\[2pt]   
	w_i \geq 1, 
\end{gathered}
\qquad 
i = 1, \ \ \ldots, \ n. 
	\end{thincases}
\end{equation*}



\begin{equation} 
	\label{MinSumW_Subj_Nonsym} 
	%\text{Найти} \quad 
	\min_{w^{-},\,w^{+},\,\beta}\;\Sigma_{i=1}^n (w_i^{-} + w_i^{+}) 
\end{equation}
при ограничениях
\begin{equation} 
%	\label{MinSumW_Constr_Nonsym} 
	\left\{ \ 
	\begin{gathered}
		\mathring{y}_i - w_i^{-} \, \epsilon^{-}_i \leq f(x_i,\beta) \leq 
		\mathring{y}_i + w_i^{+} \, \epsilon^{+}_i, \\[2pt] 
		w_i^{-} \geq 1, \\[2pt]  
		w_i^{+} \geq 1, 
	\end{gathered}
	\qquad i = 1,\dots,n. 
	\right. 
\end{equation}

\begin{equation} 
\label{MinSumW_Constr_Nonsym} 
\begin{thincases}
	\begin{gathered}
		\mathring{y}_i - w_i^{-} \, \epsilon^{-}_i \leq f(x_i,\beta) \leq 
		\mathring{y}_i + w_i^{+} \, \epsilon^{+}_i, \\[2pt] 
		w_i^{-} \geq 1, \\[2pt]  
		w_i^{+} \geq 1, 
	\end{gathered}
\qquad i = 1,\dots,n. 
\end{thincases}
\end{equation}


В интервальной системе линейных алгебраических уравнений 
\begin{equation*} 
	\arraycolsep=2pt 
\begin{thincases}
	\begin{array}{ccccccccccc}
		\beta_0 &+& x_{11}\beta_1 &+& 
		x_{12} \beta_2 &+& \ldots &+& x_{1m}\beta_m &=& \mbf{y}_{1}, \\[3pt] 
		\beta_0 &+& x_{21}\beta_1 &+& 
		x_{22} \beta_2 &+& \ldots &+& x_{2m}\beta_m &=& \mbf{y}_{2}, \\[3pt] 
		\vdots &&  \vdots && \vdots && \ddots && \vdots && \vdots                  \\[3pt]  
		\beta_0 &+& x_{n1}\beta_1 &+& 
		x_{n2} \beta_2 &+& \ldots &+& x_{nm}\beta_m &=& \mbf{y}_{n}, 
	\end{array} 
\end{thincases}
\end{equation*} 
интервальность присутствует только в правой части. С другой стороны, 
система \eqref{LinIneqs} равносильна системе \eqref{LinIneqSys} 
\begin{equation} 
	\label{LinIneqSys} 
	\arraycolsep=2pt 
\begin{thincases}
	\begin{array}{ccccc}
		\un{\mbf{y}}_{1} & \leq & \beta_0 + \beta_1 x_{11} + 
		\beta_2 x_{12} + \ldots + \beta_m x_{1m} & \leq & \ov{\mbf{y}}_{1}, \\[3pt] 
		\un{\mbf{y}}_{2} & \leq & \beta_0 + \beta_1 x_{21} + 
		\beta_2 x_{22} + \ldots + \beta_m x_{2m} & \leq & \ov{\mbf{y}}_{2}, \\[3pt] 
		\vdots & \vdots &   \ddots & \vdots & \vdots \\[3pt] 
		\un{\mbf{y}}_{n} & \leq & \beta_0 + \beta_1 x_{n1} + 
		\beta_2 x_{n2} + \ldots + \beta_m x_{nm} & \leq & \ov{\mbf{y}}_{n}. 
	\end{array} 
\end{thincases}
\end{equation} 
Это система двусторонних линейных неравенств 


\newpage
%\renewcommand{\bibname}{\centering \normalsize БИБЛИОГРАФИЧЕСКИЙ СПИСОК}  
%\renewcommand{\bibname}{\centering \small БИБЛИОГРАФИЧЕСКИЙ СПИСОК} 

% for the report or book class 
%\addcontentsline{toc}{chapter}{Библиографический список}   
\addcontentsline{toc}{part}{Библиографический список}   
\makeatletter
\renewcommand\@biblabel[1]{#1.}
\makeatother
\patchcmd{\thebibliography}{\section*{\refname}}{\centering  БИБЛИОГРАФИЧЕСКИЙ СПИСОК}{}{}
	


%\addto\captionsrussian{\def\refname{\centering \small БИБЛИОГРАФИЧЕСКИЙ СПИСОК}}
\input{BiblioBookExample.txt}
%\addcontentsline{toc}{part}{Предметный указатель}
\raggedright\small\printindex   

%%%%%%%%%%%%%%%%%%%%%%%%%%%%%%%%%%%%%%%%%%%%%%%%%%%%%%%%%%%%%%%%%%%%%%%%%%%%%%%%%%%%%%%%%%%%%%%%%%%%%%%
% 1 стр пустая- № 1
%\blankpage

\newpage
\begin{center}
	\hfill \break
	
	\Large{\it Баженов Александр Николаевич\\
		\hfill \break		\hfill \break		}
	{\Large	\bf{ВВЕДЕНИЕ В АНАЛИЗ ДАННЫХ\\
			С ИНТЕРВАЛЬНОЙ НЕОПРЕДЕЛЕННОСТЬЮ}}\\
	\hfill \break 	\hfill \break	
	\Large{	Учебное пособие	
	}\\
\end{center}

\hfill \break	

\begin{center}
	Редактор \emph{Л.\,В.\,Ларионова} \\
	%	Корректор \emph{Н.\,Б.\,Цветкова} \\
	Оригинал-макет подготовлен автором\\
	Дизайн обложки \emph{Е.\,В.\,Гладышевой}\\
	\hfill \break	
	Санитарно-эпидемиологическое заключение\\
	№ 78.01.07.953.П.001342.01.07 от 24.01.2007 г. \\
	\hfill \break	
	Налоговая льгота --- Общероссийский классификатор продукции \\
	ОК 005-93, т. 2; 95 3005 --- учебная литература\\
%	\hfill \break	
	$\ov{~~~~~~~~~~~~~~~~~~~~~~~~~~~~~~~~~~~~~~~~~~~~~~~~~~~~~~~~~~~~~~~~~~~~~~~~~~~~~~~~~~~~~~~~~~~~~~~~~~~~~~~~~~~~~~~}$\\
%	\hfill \break	
	Подписано в печать 08.12.2022. Формат 60$\times$84/16. Печать цифровая. \\
	Усл. печ. л. 6,0. Тираж 50 экз. Заказ ~~~~ \\
%	\hfill \break
	$\ov{~~~~~~~~~~~~~~~~~~~~~~~~~~~~~~~~~~~~~~~~~~~~~~~~~~~~~~~~~~~~~~~~~~~~~~~~~~~~~~~~~~~~~~~~~~~~~~~~~~~~~~~~~~~~~~~}$\\ 
%	\hfill \break
	Отпечатано в Издательско-полиграфическом центре \\
	Политехнического университета. \\
	195251, Санкт-Петербург, Политехническая ул., 29. \\
	Тел.: (812) 552-77-17; 550-4014.
	
\end{center}


\thispagestyle{empty} % выключаем отображение номера для этой страницы

\end{document}

