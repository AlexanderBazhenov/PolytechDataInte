\documentclass[a5paper,openany]{book}

\usepackage{cmap}  
\usepackage[utf8]{inputenc}
\usepackage[T2A]{fontenc} 
\usepackage{index} 
\usepackage[russian]{babel} 
\usepackage{amsmath,amssymb} 
\usepackage{euscript,upref}  
\usepackage{array,longtable}
\usepackage{indentfirst} 
\usepackage{graphicx} 
%\usepackage{caption} 
\usepackage[justification=centering]{caption}
\usepackage{calrsfs} 
\usepackage{url}
\usepackage{multirow,makecell,array}
%\usepackage{setspace} 
\usepackage{todonotes}
%\usepackage{calligra}
%\usepackage{makeidx}
%%%%%%%%%%%%%%%%%%%%%%%%%%%%%%%%%%%%%%%%%%%%%%%%%%%%%%%%%%%%%%%%%%%%%%%%%%%%%%%%%%%%%%%%
%
%           Определения новых команд и макросов
%    
%\DeclareMathAlphabet{\mathcalligra}{T1}{calligra}{m}{n}
%\DeclareFontShape{T1}{calligra}{m}{n}{<->s*[1.8]callig15}{}
\newcommand{\mbf}[1]{\protect\text{\boldmath$#1$}}
\newcommand{\mbb}{\mathbb}
\newcommand{\mrm}{\mathrm}
\newcommand{\mcl}{\mathcal}
\newcommand{\msf}{\mathsf}
\newcommand{\eus}{\EuScript}
\newcommand{\ov}{\overline}
\newcommand{\un}{\underline}
\newcommand{\m}{\mathrm{mid}\;}
\newcommand{\w}{\mathrm{wid}\;}
\newcommand{\Uni}{\mathrm{Uni}\,}
\newcommand{\Tol}{\mathrm{Tol}\,} 
\newcommand{\Uss}{\mathrm{Uss}\,} 
\newcommand{\Ab}{(\mbf{A}, \mbf{b})}
\newcommand{\Arg}{\mathrm{Arg}\;} 
\newcommand{\sgn}{\mathrm{sgn}\;} 
\newcommand{\ran}{\mathrm{ran}\,} 
\newcommand{\pro}{\mathrm{pro}\,} 
\newcommand{\dom}{\mathrm{dom}\,} 
\newcommand{\IVE}{\mathrm{IVE}\,} 
\newcommand{\IED}{\mathrm{IED}\,} 
\newcommand{\calX}{\mathrsfs{X}} 
\newcommand{\cond}{\mathrm{cond}} 
\newcommand{\mode}{\mathrm{mode}\,} 
\newcommand{\dual}{\mathrm{dual}\,} 
\newcommand{\dist}{\mathrm{dist}\,} 
\newcommand{\Dist}{\mathrm{Dist}\,} 
\newcommand{\const}{\mathrm{const}} 
\newcommand{\USS}{\varXi_{\hspace{-0.5pt}uni}} 
\newcommand{\TSS}{\varXi_{\hspace{-0.5pt}tol}} 
\newcommand{\NExt}{_{\scalebox{0.57}{$\natural$}}}
\newcommand{\ih}{\scalebox{0.67}[0.87]{$\Box$\hspace*{1pt}}}

\renewcommand{\r}{\mathrm{rad}\;} 
\renewcommand{\vert}{\mathrm{vert}\,} 
\newcommand{\md}{\operatorname{med}}

%%%%%%%%%%%%%%%%%%%%%%%%%%%%%%%%%%%%%%%%%%%%%%%%%%%%%%%%%%%%%%%%%%%%%%%%%%%%%%%%%%%%%%%

\textwidth=114truemm
\textheight=165truemm
\oddsidemargin=-1cm
\evensidemargin=\oddsidemargin
\topmargin=-1cm
\sloppy

\pagestyle{plain}
%\mathsurround=1pt
%\tolerance=400
%\hfuzz=2pt
\makeindex

\captionsetup{font=small,labelsep=period,margin=7mm} 
%%% ToC (table of contents) APPEARANCE
\usepackage[nottoc,notlof,notlot]{tocbibind} % Put the bibliography in the ToC
\usepackage{tocloft} % Alter the style of the Table of Contents
\renewcommand{\cftsecfont}{\rmfamily\mdseries\upshape}
\renewcommand{\cftsecpagefont}{\rmfamily\mdseries\upshape} % No bold!

\newtheorem{definition}{Определение}[section] 


%\renewcommand{\bibname}{References} 
%\addto{\captionsenglish}{\renewcommand{\bibname}{References}}
%\renewcommand\bibname{Библиографический список}

% переименовываем  список литературы в "список используемой литературы"
%\addto\captionsrussian{\def\refname{Список используемой литературы}}

%  в Литературе ссылки БЕЗ квадратных скобок
\makeatletter
\renewcommand\@biblabel[1]{#1.}
\makeatother
%%% Создание списка собственной переменной окружения

\newcommand{\listOfExamples}{Список примеров}
\newlistof{example}{exp}{\listOfExamples} 
\newcounter{Examp}
\setcounter{Examp}{0}
\renewcommand\theExamp{\thesection.\arabic{Examp}} %Эта команда - макрос для счетчика Examp, чтобы каждый раз не писать длинное выражение 
%Если нужна сквозная нумерация примеров:
%\renewcommand\theExamp{\arabic{Examp}}

\newenvironment{example}[1]
{	\par\vspace{\baselineskip}
	\refstepcounter{Examp}	
	{\noindent\textbf{Пример~\theExamp~(#1)} 			
		\protect\addcontentsline{exp}{example}{\protect\numberline{\theExamp}\hspace{10pt}~#1}}
	%{\protect\numberline{\listOfExamples}}	
	{\noindent\ignorespaces}
}
%{\hfill$\blacksquare$\par\addvspace{3ex}}
{\hfill\par\addvspace{3ex}}
%

\newcounter{DefNum}
\setcounter{DefNum}{1}






\begin{document}
%\maketitle

\begin{center}
	\hfill \break
Министерство науки и высшего образования  Российской Федерации\\
%	\hfill \break
$\ov{~~~~~~~~~~~~~}$\\
	\normalsize{	САНКТ-ПЕТЕРБУРГСКИЙ \\
		ПОЛИТЕХНИЧЕСКИЙ УНИВЕРСИТЕТ ПЕТРА ВЕЛИКОГО}\\ 
$\ov{~~~~~~~~~~~~~~~~~~~~~~~~~~~~~~~~~~~~~~~~~~~~~~~~~~~~~~~~~~~~~~~~~~~~~~~~~~~~~~~~~~~~~~~~~~~~~~}$\\	
	{\small Физико-механический институт\\
	Высшая школа прикладной математики и вычислительной физики}\\
	\hfill \break
	\Large{\it А.\,Н.\,Баженов\\
		\hfill \break		\hfill \break		}
	{\Large	ВВЕДЕНИЕ В АНАЛИЗ ДАННЫХ\\
		С ИНТЕРВАЛЬНОЙ НЕОПРЕДЕЛЕННОСТЬЮ}\\
	\hfill \break 	\hfill \break	
	\Large{	Учебное пособие	
	}\\
\end{center}

		\hfill \break		\hfill \break	
\begin{figure}[h]
	\centering
	\includegraphics[width=60mm]{PolytechPressRu.png}
	%	\label{f:cover}	
\end{figure}
%\hfill \break
%\hfill \break
\begin{center}\Large{Санкт-Петербург \\
%		\hfill \break
		2022} \end{center}
\thispagestyle{empty} % выключаем отображение номера для этой страницы


\newpage
\begin{tabular}{rl}
	УДК & 519.9 \\	
	ББК  & ~~~\\
	~~~ & Б
\end{tabular}


\begin{center}
Р е ц е н з е н т ы:\\

Кандидат физико-математических наук, научный сотрудник Физико-технического института им. А.\,Ф.\,Иоффе
{\it А.\,А.\,Красилин}\\
Кандидат физико-математических наук, доцент Санкт-Петербургского политехнического  университета \\ Петра Великого {\it Л.\,В.\,Павлова}
 \end{center}

%Доктор физико-математических наук, профессор НГУ С.П.Шарый
{\it Баженов\,А.\,Н.}
{\bf Введение в анализ данных с интервальной неопределенностью} : учеб. пособие /  А.\,Н.\,Баженов.
--- СПб. : ПОЛИТЕХ-ПРЕСС, 2022. --- XXX с.
\hfill \break

{\small 
	Учебное пособие соответствует образовательному стандарту высшего
образования Санкт-Петербургского политехнического университета Петра Великого по направлению подготовки бакалавров 01.03.02 <<Прикладная математика и информатика>>, по дисциплине «Интервальный анализ».


Пособие посвящено введению в анализ данных с интервальной неопределенностью
и демонстрации его применения в различных задачах.  
Важной частью пособия являются примеры, от самых простых, иллюстрирующих базовые конструкции и операции, до более сложных.

Пособие предназначено для студентов, аспирантов, научных сотрудников и инженеров, 
занимающихся анализом данных.


\hfill \break
Табл.~6. Ил.~34. Библиогр.: 40 назв.
\hfill \break
\hfill \break

 \begin{center}
{\small  	
Печатается по решению\\
Совета по издательской деятельности Ученого совета\\
Санкт-Петербургского политехнического  университета Петра Великого. }
 \end{center}

\hfill \break
\begin{tabular}{ll}
	~ & \copyright  \ Баженов\,А.\,Н., 2022 \\
{\bf ISBN 978-5-7422-\ldots} & \copyright \
Санкт-Петербургский политехнический \\
doi:10.18720/SPBPU/2/id22-\ldots & ~~~~~университет Петра Великого, 2022
\end{tabular}


\thispagestyle{empty}

\newpage
% 2 стр пустые
\begin{center}
	\hfill \break
	Министерство науки и высшего образования  Российской Федерации\\
	%	\hfill \break
$\ov{~~~~~~~~~~~~~}$\\
	\normalsize{	САНКТ-ПЕТЕРБУРГСКИЙ \\
		ПОЛИТЕХНИЧЕСКИЙ УНИВЕРСИТЕТ ПЕТРА ВЕЛИКОГО}\\ 
	$\ov{~~~~~~~~~~~~~~~~~~~~~~~~~~~~~~~~~~~~~~~~~~~~~~~~~~~~~~~~~~~~~~~~~~~~~~~~~~~~~~~~~~~~~~~~~~~~~~}$\\	
	{\small Физико-механический институт\\
	Высшая школа прикладной математики и вычислительной физики}\\
	\hfill \break
	\Large{\it А.\,Н.\,Баженов\\
		\hfill \break		\hfill \break		}
	{\Large	ВВЕДЕНИЕ В АНАЛИЗ ДАННЫХ\\
		С ИНТЕРВАЛЬНОЙ НЕОПРЕДЕЛЕННОСТЬЮ}\\
	\hfill \break 	\hfill \break	
	\Large{	Учебное пособие	
	}\\
\end{center}

\hfill \break		\hfill \break	
\begin{figure}[h]
	\centering
	\includegraphics[width=60mm]{PolytechPressRu.png}
	%	\label{f:cover}	
\end{figure}
%\hfill \break
%\hfill \break
\begin{center}\Large{Санкт-Петербург \\
		2022} \end{center}
\thispagestyle{empty} % выключаем отображение номера для этой страницы


\newpage
\begin{tabular}{rl}
УДК & 519.9 \\	
ББК  & ~~~\\
~~~ & Б
\end{tabular}






\begin{center}
	Р е ц е н з е н т ы:\\
	
	Кандидат физико-математических наук, научный сотрудник Физико-технического института им. А.\,Ф.\,Иоффе
	{\it А.\,А.\,Красилин}\\
	Кандидат физико-математических наук, доцент Санкт-Петербургского политехнического  университета \\ Петра Великого {\it Л.\,В.\,Павлова}
\end{center}

%Доктор физико-математических наук, профессор НГУ С.П.Шарый
{\it Баженов\,А.\,Н.}
{\bf Введение в анализ данных с интервальной неопределенностью} : учеб. пособие /  А.\,Н.\,Баженов.
--- СПб. : ПОЛИТЕХ-ПРЕСС, 2022. --- XXX с.
\hfill \break

{\small 
	Учебное пособие соответствует образовательному стандарту высшего
	образования Санкт-Петербургского политехнического университета Петра Великого по направлению подготовки бакалавров 01.03.02 <<Прикладная математика и информатика>>, по дисциплине «Интервальный анализ».
	
	
	Пособие посвящено введению в анализ данных с интервальной неопределенностью
	и демонстрации его применения в различных задачах.  
	Важной частью пособия являются примеры, от самых простых, иллюстрирующих базовые конструкции и операции, до более сложных.

	Пособие предназначено для студентов, аспирантов, научных сотрудников и инженеров, 
	занимающихся анализом данных.
	
	 
	
	
	\hfill \break
Табл.~6. Ил.~34. Библиогр.: 40 назв.
	\hfill \break
	\hfill \break
	
	\begin{center}
		{\small  	
			Печатается по решению\\
			Совета по издательской деятельности Ученого совета\\
			Санкт-Петербургского политехнического  университета Петра Великого. }
	\end{center}
	
	\hfill \break
	\begin{tabular}{ll}
		~ & \copyright  \ Баженов\,А.\,Н., 2022 \\
		{\bf ISBN 978-5-7422-\ldots} & \copyright \
		Санкт-Петербургский политехнический \\
		doi:10.18720/SPBPU/2/id22-\ldots &  ~~~~~университет Петра Великого, 2022
	\end{tabular}
	
	
	\thispagestyle{empty}

\newpage
\tableofcontents

\newpage

\listofexample 

%\cleardoublepage
\newpage

%\listoffigures

\newpage

	\chapter*{Введение}
\addcontentsline{toc}{chapter}{Введение}   



Учебное пособие является дополнением к книге коллектива авторов
<<Обработка и анализ данных с интервальной неопределенностью>> \cite{MetodikaBook}, изложение материала в которой носит методически основательный характер.
В пособии представлена согласованная система понятий и терминов, относящихся к обработке данных, имеющих интервальную  неопределенность, а также
дан краткий обзор основных и наиболее значимых результатов научного направления, которое можно назвать <<статистикой интервальных данных>>, или <<анализом интервальных 
данных>>. В пособии много ссылок на  \cite{MetodikaBook}, в которой теоретические аспекты изложены обстоятельно и подробно.
Задача пособия --- дать обучащимся краткие сведения о теории и рассмотреть ряд примеров, иллюстрирующих интервальный подход. 


Фундаментом статистики данных с интервальной неопределенностью является интервальный анализ. 
Основы интервального анализа представлены в  \cite{SPbSTU2020}. В  \cite{SPbSTU2021} дана картина применения интервальных данных и интервального анализа в более широком контексте. 
Наиболее полное изложение идей и методов интервального анализа дано в  \cite{SSharyBook}. Изучение материала  книги  \cite{SSharyBook} требует более основательной математической подготовки и рекомендуется для углубленного изучения данного вопроса.

В общем виде задачи статистики данных с интервальной неопределенностью состоят в решении практических проблем в тех областях обработки данных, в которых недостаточны ранее развитые методы.
В практике обработки экспериментальных данных в настоящее время широко используются 
статистические методы, основанные на идеях и результатах теории вероятностей. Эти методы 
опираются на использование ряда допущений о вероятностных свойствах погрешностей 
измерений, а также на наличие выборок представительной длины (как минимум в несколько 
десятков измерений). 

Однако практики часто сталкиваются с ситуациями, когда выборки 
измерений коротки, а погрешности 
измерений не могут быть адекватно описаны с помощью инструментов теории вероятностей 
или же информация о вероятностных характеристиках погрешностей отсутствует. 

В этих ситуациях можно применить методы интервальной статистики, основанные на идеях и результатах интервального анализа, использующих его подходы, алгоритмы и соответствующее программное обеспечение. 
Интервальные методы широко представлены практически для всех популярных платформ программирования. В некоторых интегрированных средах, как, например, {\tt Mathematica}, {\tt Octave}, поддержка базовых интервальных конструкций встроенная. \index{Mathematica} \index{Octave} 
Для использования наиболее популярного в настоящее время языка программирования {\tt Python} также есть реализации основных конструкций и методов интервального анализа. \index{Python}

Терминология интервальной 
статистики наследует 
многое из традиционной статистики,  развитый понятийный аппарат которой уже сложился. 
Различным аспектам анализа интервальных данных посвящены, в частности,   \cite{SSharyJCT2017}-\cite{Kumkov2013}, \cite{NguyenKreinWuXiang}.

Следует отметить, что в XX в. в статистике различные математические методы  продолжали развиваться и использоваться, но как будто не входя в математическую статистику. Дж.\,Тьюки в конце 50-х годов прошлого века предложил оформить 
новую научную дисциплину <<анализ данных>>,\index{анализ данных} в которой 
охватывались те математические методы обработки данных, которые не подпадали 
под математическую статистику в узком смысле этого слова \cite{Tukey1962}.


Материал учебного пособия апробирован в учебных курсах для студентов  
Высшей школы прикладной математики и вычислительной физики
 Физико-механического института Санкт-Петербургского политехнического университета Петра Великого
 и аспирантов Физико-технического института им.~А.\,Ф.\,Иоффе Российской академии наук.
 

%{\color{red} заимствовано}	
	\chapter[Краткие сведения о методах статистики и обработки данных]% 
{Краткие сведения о методах статистики\\* и обработки данных} 

	\section{Данные, погрешности и их обработка} 


На практике данные не бывают точными. В действительности нам известно приближенное
значение измеряемой величины, а также некоторая информация (качественная 
и количественная) о погрешности этого значения.  \index{неопределенность} 
На результаты измерений могут оказывать влияние изменчивость измеряемых величин, 
их непостоянство во времени или пространстве. На измерения могут влиять внешние неконтролируемые факторы, так 
называемые <<шумы>>. 
У применяемой аппаратуры имеются собственные погрешности. 
В процессе математической обработки данных на результат влияют
неизбежные неточности расчетов (ошибки представления,  округления и т.\,п.). 

В \cite{Malikov} погрешности измерений и наблюдений разделяются на три класса: 
\begin{list}{}{\leftmargin=14mm\itemsep=5pt\topsep=3pt\parsep=0pt} 
	\item [1.] 
	Систематические погрешности. \index{систематическая погрешность}
	\item[2.] 
	Случайные погрешности. \index{случайная погрешность}
	\item[3.] 
	Промахи (или выбросы). \index{промах}\index{выброс} 
\end{list} 

\emph{Систематической погрешностью} измерения называется составляющая погрешности 
измерения, которая остается постоянной или изменяется по какому-то определенному 
закону при повторных измерениях одной и той же величины. \emph{Случайными погрешностями} 
называются неопределенные по своей величине и природе погрешности, в появлении каждой 
из которых не наблюдается какой-либо явной закономерности. 
\emph{Промахами} (\emph{выбросами}) называются погрешности, приводящие к явному 
искажению результата измерений. 
Для выявления выбросов и промахов организуют специальный этап общей технологии 
обработки данных --- \emph{предобработку}, который предшествует применению формальных 
математических методов. На этом этапе промахи (выбросы)   должны быть определены и удалены 
из обрабатываемых данных.               \index{предобработка} 
Что касается случайных погрешностей, то в \cite{Malikov} указывается: <<Мы считаем случайными те явления, которые определяются сложной совокупностью переменных 
причин, трудно поддающихся анализу; к этим явлениям индивидуальный подход невозможен, 
и лишь для их совокупности могут быть установлены определенные закономерности>>. 
Таким образом, термин <<случайный>> в этом понимании  фактически означает
<<непредсказуемый>> или же такой, в чем  отсутствует закономерность. 

%\paragraph
{\bf Как учитывать случайные погрешности в данных?}
Прежде всего, сам факт присутствия таких погрешностей в данных можно учесть подходящей 
математической постановкой задачи обработки этих данных. Например, при восстановлении 
функциональных зависимостей (см. гл.~4) вместо задачи интерполяции данных нужно 
рассматривать задачу их аппроксимации (приближения), так как не имеет смысла требовать 
точных равенств значений функции измеренным значениям. Вообще говоря, получение результата 
измерения или наблюдения как решения задачи некоторого математического приближения 
к данным, учитывающей модель исследуемого объекта или явления, является основой 
\emph{аппроксимационных методов} \index{аппроксимационные методы} 
обработки данных.   

Если о природе случайных погрешностей ничего более не известно, то на этом можно 
и нужно остановиться и применять далее аппроксимационные методы. Если о природе 
случайных погрешностей известно что-то определенное, то можно применить для обработки 
данных более точные методы, учитывающие дополнительную информацию. 

В настоящее время существует несколько различных подходов к описанию 
случайности, и некоторые их них чрезвычайно развиты и популярны. Прежде всего, 
это теоретико-вероятностная модель погрешностей, основанная на аппарате математической 
теории вероятности и приводящая к \emph{теоретико-вероятностным методам} обработки 
данных. 
\index{теоретико-вероятностные методы} 
Теоретико-вероятностная модель погрешностей за прошедшие два 
века получила большое развитие и распространение, став одним из основных 
инструментов обработки данных. Также следует отметить методы нечеткой статистики 
(см.~п.~\ref{FuzzyStatSect}) 
и \emph{эвристические методы} обработки данных,\index{эвристические методы} которые 
применяются при анализе малоизученных явлений, когда отсутствует четкая модель и нет 
представления об искомых характеристиках явления или объекта.  
  \index{вероятностная статистика} 

%%%%%%%%%%%%%%%%%%%%%%%%%%%%%%%%%%%%%%%%%%%%%%%%%%%%%%%%%%%%%%%%%%%%%%%%%%%%%%%%%%%%%%%% 
	\section{Критика вероятностной статистики \\  и альтернативные подходы} 

Развернутая критика вероятностной статистики  содержится в  \cite{MetodikaBook}. Кратко перечислим основные пункты, по которым в  в  \cite{MetodikaBook} проведено обсуждение.

{\bf Статистическая устойчивость.} 
Главной интерпретацией  \index{вероятность} \index{частотная интерпретация}
понятия вероятности является так называемая \emph{частотная интерпретация}, при 
которой вероятность понимается как предел относительной частоты рассматриваемого 
события в серии однородных независимых испытаний (экспериментов и т.\,п.). 

Многие явления окружающего нас мира, в отношении которых применимо слово <<случайный>>, не обладают свойством существования устойчивой относительной 
частоты, так как при росте числа наблюдений она для них не устанавливается. 
Для описания и анализа подобных явлений традиционная теория вероятностей непригодна. 
\index{статистическая устойчивость} 

{\bf Проблема малых выборок.} 
Вероятностные закономерности проявляются как тенденции, которые наиболее заметны 
в массовых явлениях.  Фактически
при обработке экспериментальных данных почти всегда стоит вопрос о том, достаточен ли 
объем выборки (количество измерений и т.\,п.) для того, чтобы выводы, получаемые 
на основе теоретико-вероятностной модели погрешностей, имели приемлемую практическую 
достоверность. 

Существующие промышленные стандарты и методики обработки экспериментальных данных (например, \cite{GUM} -\cite{GOSTDirect}) регламентируют способы работы с выборками размера лишь 
более $15$. При этом результаты обработки выборок размера от $16$ до $50$ рекомендуется 
рассматривать как не очень надежные и сопровождать оговорками, а обработка выборок 
размером не более чем из $15$ измерений стандартами вообще не рассматривается. 

{\bf Неизвестные вероятностные характеристики распределения.} 
Если законы теории вероятностей применимы к анализу погрешностей, то каков конкретный 
вид вероятностных распределений погрешностей? Каковы его числовые характеристики? 
Это непростые вопросы, на которые  не всегда есть ответ.

Например, считается, что типичным законом вероятностного распределения погрешностей 
является нормальное гауссово распределение. Но насколько оно соответствует действительности?  \index{нормальное распределение} 
Известно высказывание А.\,Пуанкаре  \cite{Poincare}: <<$\ldots$все верят в этот закон  ~\ldots\,  потому что экспериментаторы думают, 
что это математическое утверждение, а математики --- что это результат экспериментов>>.  
Реальные 
распределения погрешностей измерений в различных ситуациях могут сильно отличаться 
от нормального гауссового.
Для того чтобы выяснить, какое вероятностное распределение имеют 
анализируемые данные, часто требуется большая дополнительная работа, требующая выборок более 1000 измерений \cite{Orlov2016}. 

Конкретный вид функций распределения случайных величин, которые фигурируют в задачах 
обработки данных, может оказывать существенное влияние на способ их решения. Методология 
максимума правдоподобия для случая нормально распределенных погрешностей данных указывает на 
метод наименьших квадратов, метод наименьших модулей для распределения Лапласа или метод чебышевского сглаживания (минимаксное приближение данных) для равномерно распределенных погрешностей. 


{\bf Независимость данных.}  \index{корреляция}\index{независимость} 
Еще одна группа вопросов касается часто используемых в теории 
вероятностей понятий \emph{независимости} и \emph{корреляции} случайных величин. 
Имеют ли данные корреляцию  между собой? Или же они независимы? Многие классические 
результаты вероятностной статистики требуют, как известно,  независимости 
рассматриваемых случайных величин, представляющих результаты измерений, либо 
заданного уровня их корреляции. Проверка этих условий на практике почти невозможна. 

{\bf Наличие погрешностей различных типов.} 
В ходе измерений, помимо статистических погрешностей, всегда присутствуют и систематические. Последние могут имет разые источники, их весьма сложно  оценить. Часто эта оценка намного сложнее получения собственно результатов. Но даже если эти оценки получены, появляется вопрос: <<Каким образом можно получить совокупную ошибку?>>

\subsection{Статистика нечетких данных} 
\label{FuzzyStatSect} 

При нечетком описании результатов измерений и наблюдений мы полагаем, что вместо их 
точных значений нам известны так называемые \emph{функции принадлежности} нечетких чисел, 
получающихся в результате измерений \cite{NguyenKreinWuXiang}. 
\index{нечеткие методы} 
Возникновение нечетких чисел в природных явлениях на примере спектров возбуждения и эмиссии \cite{Javoruk2021} представлено на рис.~\ref{FuzzyNumbers}.
%%%%%%%%%%%%%%%%%%%%%%%%%%%%%%%%%%%%%%%%%%%%%%%%%%%%%%%%%%%%%%%%%%%%%%%%%%%%%%%%%%%%%%%%  

\begin{figure}[ht]
	\centering\small  
	\setlength{\unitlength}{1mm} 
	\begin{picture}(110,45) 
%	\put(0,0){\includegraphics[width=50mm]{FuzzyNumber-11.eps}} 
%	\put(47,4){$x$}
%	\put(34,20){$\mu_{1}(x)$}
%	\put(58,0){\includegraphics[width=50mm]{FuzzyNumber-22.eps}} 
%	\put(87,20){$\mu_{2}(x)$} 
%	\put(105,4){$x$}
	\put(25,0){\includegraphics[width=60mm]{ExcitationEmission.png}}
	\end{picture} 
	\caption{Спектры возбуждения-эмиссии как нечеткие числа \cite{Javoruk2021}}
	\label{FuzzyNumbers}  
\end{figure} 

%%%%%%%%%%%%%%%%%%%%%%%%%%%%%%%%%%%%%%%%%%%%%%%%%%%%%%%%%%%%%%%%%%%%%%%%%%%%%%%%%%%%%%%%

\emph{Нечетким множеством} %(см. \cite{DuboisPrade,Zadeh}) 
называется множество $X$, 
образованное элементами произвольной природы, которое дополнено так называемой 
\emph{функцией принадлежности} $\mu: X\to[0, 1]$, значение которой $\mu(x)$ на элементе 
$x\in X$ показывает степень принадлежности $x$ множеству $X$ (рис.~\ref{FuzzyNumbers}). 
У стандартной функции принадлежности множества (называемой также \emph{индикаторной функцией} 
множества) значения могут быть равны только 0 или 1, поэтому допущение для функции $\mu$ 
непрерывного ряда значений из интервала $[0, 1]$ позволяет характеризовать ситуации, когда 
нет уверенности в принадлежности элемента множеству, невозможно оперировать количественной мерой \index{нечеткое множество} \index{функция принадлежности} 
уверенности и строить на этой основе наши выводы и заключения.

Для построения содержательной теории нечеткого вывода и нечетких неопределенностей 
обычно ограничивают общность функции принадлежности $\mu$, требуя, чтобы она была 
\emph{квазивогнутой}. 
В одномерном случае они являются интервалами. 
Нечеткие множества с квазивогнутыми 
функциями принадлежности называются \emph{нечеткими числами}, и они могут быть 
эквивалентным образом заданы как семейства вложенных друг в друга интервалов, 
которые соответствуют различным уровням принадлежности. \index{нечеткое число}  

Для обработки данных, имеющих нечеткую неопределенность, предложены разные подходы 
(см., например, \cite{NguyenKreinWuXiang}), в частности, 
большое развитие получили методы восстановления зависимостей по нечетким данным \cite{Boukezzoula2021}. 
%Обзор применения различных вариантов методов нечетких множеств и интервальных подходов дан в публикации \cite{Boukezzoula2021}.
%%%%%%%%%%%%%%%%%%%%%%%%%%%%%%%%%%%%%%%%%%%%%%%%%%%%%%%%%%%%%%%%%%%%%%%%%%%%%%%%%%%%%%%%   



	\section[Место и особенности интервального подхода]%  
{Место и особенности \\* интервального подхода} 


\subsection{Почему интервалы?} \label{NatureIntervals} 

{\bf Устройство природы.}
Фундаментальная причина использования интервалов для описания данных состоит в том, что некоторые физические (химические, биологические и 
т.\,п.) величины принципиально не могут быть выражены точечными значениями, а лишь 
интервалами. Поэтому интервалы представляют собой новый удобный тип данных, которым 
уместно дополнить элементарные типы данных, использующиеся в метрологии. 
Большое количество примеров из разных областей науки и техники приведено в \cite{SPbSTU2021}. 

%%%%%%%%%%%%%%%%%%%%%%%%%%%%%%%%%%%%%%%%%%%%%%%%%%%%%%%%%%%%%%%%%%%%%%%%%%%%%%%%%%%%%%%%   

\begin{example}{Интервальные веса химических элементов} 
С 2009 г. атомные веса некоторых элементов в Периодической системе  
элементов Д.\,И.\,Менделеева, поддерживаемой Международным союзом теоретической 
	и прикладной химии (ИЮПАК, IUPAC), стали выражаться интервалами \cite{IUPAC}. 
Почти каждый химический элемент представлен в природе смесью своих 
изотопов, т.\,е. разновидностями атомов, сходных по своим химическим свойствам 
(структуре электронных оболочек), но отличающихся массой ядер. Относительная 
доля различных изотопов существенно меняется в зависимости от места и характера 
взятия пробы. Например, в тканях живых организмов преобладают более легкие изотопы 
химических элементов, нежели в неживой природе. Отличаются друг от друга 
относительные доли изотопов элементов на суше и в морях и т.\,п. 
	
Известны изотопы ртути с массовыми числами от 170 до 216 (количество протонов --- 80, нейтронов --- от 90 до 136). Природная ртуть состоит из смеси семи стабильных изотопов, гистограмма частот изотопов  показана на рис.~\ref{f:HistHg}.
	\begin{figure}[ht] 
		\centering\small
%			\setlength{\unitlength}{0.5mm}
		\begin{tikzpicture}[scale = 0.9]
		%\draw[help lines] (0,0) grid (11,5);
		\draw[->] (0,0) -- (0,4);
		\draw[->] (0,0) -- (10,0);
		\draw (-0.5, 1) node {10 \%};
		\draw (-0.5, 2) node {20 \%};
		\draw (-0.5, 3) node {30 \%};
		\draw[red] (0,0.0155) -- (1,0.0155);
		\draw[red] (1,0.0155) -- (1,0);
		\draw[red] (1,0) -- (2,0);
		\draw[red] (2,0) -- (2,1.004);
		\draw[red] (2,1.004) -- (3,1.004);
		\draw[red] (3,1.004) -- (3,1.694);
		\draw[red] (3,1.694) -- (4,1.694);
		\draw[red] (4,1.694) -- (4,2.314);
		\draw[red] (4,2.314) -- (5,2.314);
		\draw[red] (5,2.314) -- (5,2.314);
		\draw[red] (5,2.314) -- (5,1.317);
		\draw[red] (5,1.317) -- (6,1.317);
		\draw[red] (6,1.317) -- (6,2.974);
		\draw[red] (6,2.974) -- (7,2.974);
		\draw[red] (7,2.974) -- (7,0);
		\draw[red] (7,0) -- (8,0);
		\draw[red] (8,0) -- (8,0.682);
		\draw[red] (8,0.682) -- (9,0.682);
		\draw[red] (9,0.682) -- (9,0);
		\draw (0.6,-0.5) node {196};
		\draw (2.6,-0.5) node {198};
		\draw (4.6,-0.5) node {200};
		\draw (6.6,-0.5) node {202};
		\draw (8.6,-0.5) node {204};
		\draw (10.5,-0.5) node {Масса};
		\draw (10.5,-1) node {изотопа};
		\draw (2,4) node {Распространенность};
		\end{tikzpicture}
		\caption{Распространенность изотопов ртути на Земле}
		\label{f:HistHg}
	\end{figure}
\end{example} 

{\bf Математические причины.}
В чем преимущества и недостатки интервалов в сравнении с другими способами описания 
неопределенности? Имеется ряд причин, по которым интервалы нужны и важны при обработке данных.  

Интервалы являются средством описания и представления типа неопределенностей, часто встречающихся в реальной жизни, ограниченных 
по величине неопределенностей. Интервалы проще, чем вероятностные распределения или 
нечеткие множества. 
Интервал --- это <<бесструктурный объект>>, который более сжато описывает неопределенность.
Следствием этой простоты является лучшая развитость теории 
интервального анализа и интервальных вычислительных методов. 

Дале, интервалы 
и интервальные арифметики оказываются  уникальными во многих 
отношениях, в частности, по своим алгебраическим свойствам, простоте и богатству опеределения отношений между объектами и результатами операций.

Наконец, интервалы являются предельным случаем сумм независимых ограниченных величин. 
В большинстве практических ситуаций погрешность измерения возникает в результате 
накопления и наложения большого количества независимых факторов. 
Если некоторая величина есть сумма большого количества малых независимых слагаемых, 
то множество всевозможных значений этой величины близко к интервалу.
Этот результат составляет содержание <<предельной 
теоремы Крейновича>> и ее обобщений \cite{SSharyBook}.
      \index{теорема Крейновича предельная}  

Кроме того, на основе интервалов можно строить составные математические объекты, описывающие аспекты данных и вычислений, которые недоступны в рамках вещественной арифметики, твины и мультиинтервалы. Кратко рассмотрим их в п. \ref{CompositeIntervalTypes}.

\subsection{Статистика интервальных данных} 
\label{InteStatistiSect}    


\textit{Интервальной неопределенностью} называется состояние частичного знания 
о величине, которая не известна точно, но известны нижняя и верхняя границы ее возможных 
значений, или,\index{интервальная неопределенность} иными словами, известен интервал 
возможных значений этой величины. 

В одномерном случае интервалы являются практически наиболее важными ограниченными 
множествами, поэтому другие множества неопределенности используются нечасто. Но 
в многомерном случае множествами возможных значений величины, имеющей ограниченную 
неопределенность, могут быть брусы, многогранники, параллелотопы (зонотопы), 
эллипсоиды и прочие объекты. Мы их  относим к объектам интервальной 
статистики и интервального анализа данных. 

Отличительной чертой представляемого подхода является его применимость 
к выборкам любого объема, начиная с нескольких измерений (в предельном случае --- 
одного). Как следствие, проблемы <<малых выборок>>, характерной для вероятностной 
статистики, в интервальном подходе не существует. Это свойство особенно 
ценно, когда технические или экономические причины не позволяют проводить 
много экспериментов. В частности, такова ситуация с алгоритмами обработки результатов 
разрушающих измерений или измерений быстропротекающих процессов в реальном масштабе 
времени. 
Интервальные методы имеют \index{аппроксимационные методы}
\emph{аппроксимационный} характер, т.\,е. осуществляют приближение (аппроксимацию) 
данных в нужном смысле. Следовательно, для их применения  массовость не требуется.   


Развиваемые идеи впервые были оформлены в пионерской работе на данную тему
Л.\,В.\,Канторовича \cite{Kantorovich} и далее неоднократно использовались (или 
даже переоткрывались) разными авторами. 
В то время для обозначения \index{минимаксный подход} 
аналогичных  подходов в литературе по математике использовались разные термины --- <<минимаксный подход>> и др. 

Интервальный подход позволяет построить достаточно простую и 
элегантную методику определения выбросов в данных. 
При анализе постоянных величин  имеется ряд обобщений традиционных методов, дающих более обширную информацию. 

В задаче восстановления зависимостей 
неопределенности входных и выходных переменных учитываются естественным образом. 
Оценка погрешности результатов получается автоматически в процессе вычислений, не требует дополнительного анализа и напрямую зависит от неопределенности 
данных задачи. 

{\bf Принцип соответствия.} 
В методологии науки \textit{принцип соответствия} называют утверждение, что любая 
новая научная теория должна включать старую теорию и ее результаты как частный 
предельный случай. 	Далее будем использовать принцип соответствия как инструмент проверки адекватности используемых конструкций, понятий и методов обработки данных с интервальными 
неопределенностями, который позволяет отсекать заведомо <<неразумные>>.
\index{принцип соответствия}


	\chapter[Базовые понятия и математический аппарат]% 
{Базовые понятия\\ и  математический аппарат} 
\label{PrimaryConceptChap} 



	\section{Интервалы} 
\label{IntervalSect} 


{\bf Вещественные интервалы.} 
Первичное понятие интервального анализа --- \emph{интервал}. Это множество, задающее целый диапазон значений интересующей нас величины, с помощью 
которого можно рассматривать неопределенности и неоднозначности. 

Интервалы могут определяться на вещественной оси, на комплексной плоскости, а также 
в многомерных пространствах \cite{SSharyBook}. Далее будут рассматриваться вещественные 
интервалы, интервальные векторы и матрицы, так как именно они играют главную роль 
в измерениях и их обработке. 
\begin{definition}   
	\textit{Интервалом} $[a,b]$ вещественной оси $\mbb{R}$ называется  
	множество всех чисел, расположенных между заданными числами $a$ и $b$, 
	включая их самих, т.\,е.                           \index{интервал} 
	\begin{equation*} 
		[a, b] := \{\, x\in\mbb{R} \mid a\leq x\leq b\,\}. 
	\end{equation*} 
	При этом $a$ и $b$ называются \textit{концами} интервала $[a,b]$, \textit{левым} 
	(или нижним) и \textit{правым} (или верхним) соответственно. 
\end{definition}
Аналогичные термины, которые часто используются в математических текстах, --- 
это \emph{числовой промежуток} (замкнутый), \emph{отрезок}, \emph{сегмент} 
вещественной оси.
\begin{figure}[hbt]
	\centering\small 
	\setlength{\unitlength}{1mm}
	\begin{picture}(70,10)
		\put(0,0){\includegraphics[width=70mm]{IntervalR1.eps}}
		\put(10,6.6){\vector(1,0){45}} \put(55,7.6){$\mbb{R}$} 
		\put(20,9.5){$a$} \put(40,9.5){$b$} 
		\put(30,2.3){$\mbf{x}$} 
	\end{picture}
	\caption{Интервал на вещественной оси} 
	\label{IntervalsPic} 
\end{figure}

%%%%%%%%%%%%%%%%%%%%%%%%%%%%%%%%%%%%%%%%%%%%%%%%%%%%%%%%%%%%%%%%%%%%%%%%%%%%%%%%

Множество всех интервалов из $\mbb{R}$ обозначается символом $\mbb{IR}$. 
В противоположность интервалам и интервальным величинам будем называть 
\emph{точечными} те величины, значениями которых являются отдельные точки 
вещественной оси или пространства более высокой размерности. 
Множество вещественных чисел $\mbb{R}$ можно рассматривать как подмножество 
множества интервалов, т.\,е. как интервалы с совпадающими концами. 
%Их обычно называют \textit{вырожденными интервалами}. 
Итак, 
$$\mbb{R}\subseteq\mbb{IR}.$$ 

Используемая система обозначений следует неформальному международному стандарту обозначений в интервальном анализе \cite{InteNotation}. 
В частности, интервалы и другие интервальные величины (векторы, матрицы и др.) всюду 
в тексте обозначаются полужирным шрифтом, например, $\mbf{A}$, $\mbf{B}$, 
~ \ldots, ~ $\mbf{y}$, $\mbf{z}$, тогда как неинтервальные 
(точечные) величины никак специально не выделяются. Для интервала $\mbf{a}$ 
посредством $\un{\mbf{a}}$ или $\,\inf\mbf{a}$ обозначается его левый конец, тогда 
как $\ov{\mbf{a}}$ или $\,\sup\mbf{a}$ --- это его правый конец. 

В целом $\mbf{a} = [\un{\mbf{a}}, \ov{\mbf{a}}]$, поэтому
\begin{equation}
	\label{LowUpRepres}
	\mbf{a} = \{\,x\in\mbb{R} \mid \,\un{\mbf{a}}\leq x\leq \ov{\mbf{a}}\,\}.
\end{equation} 
{\bf Характеристики интервала.} \label{InrevalProp}
Любой интервал полностью задается двумя числами --- своими концами, но на практике 
широко используются также другие характеристики интервалов и представления интервалов 
на их основе. 

Важнейшими характеристиками интервала являются его \emph{середина} (центр) 
\begin{equation}\label{midint}
	\textstyle\index{середина интервала} 
	\m\mbf{a} = \frac{1}{2}(\ov{\mbf{a}} + \un{\mbf{a}}),
\end{equation}
и его \emph{радиус} 
\begin{equation}\label{radint} 	\textstyle\index{радиус интервала}
	\r\mbf{a} = \frac{1}{2}(\ov{\mbf{a}} - \un{\mbf{a}}).
\end{equation} 
В ряде случаев (например, п.~\ref{MeasuresSampleSect}) вместо радиуса рассматривается эквивалентное понятие \emph{ширины} 
интервала  \index{ширина интервала} 
\begin{equation}\label{widint}
	\w\mbf{a} = \ov{\mbf{a}} - \un{\mbf{a}}. 
\end{equation}
В целом $\mbf{a} = \m\mbf{a} + [-1, 1]\cdot\r\mbf{a}$, что равносильно представлению 
\begin{equation}
	\label{MidRadRepres}
	\mbf{a} = \{\,x\in\mbb{R} \mid \,|x-\m\mbf{a}|\leq \r\mbf{a}\,\}.
\end{equation} 
Таким образом, задание середины и радиуса интервала также однозначно определяет его.

Середина интервала --- это точка, которая представляет его наилучшим образом, 
так как она наименее удалена от остальных точек этого интервала. 

Радиус и ширина характеризуют разброс (рассеяние) точек интервала, т.\,е. абсолютную 
меру неопределенности или неоднозначности, выражаемой этим интервалом. 
Интервалы нулевой ширины (нулевого радиуса) обычно называют  
\textit{вырожденными}. Они отождествляются с вещественными числами, поэтому, 
к примеру, $[1, 1]$ --- это то же самое, что и $1$. %, а $[0, 0] = 0$. 
\index{интервал вырожденный} 

С одной стороны, важной характеристикой интервала является его \textit{модуль} (\emph{магнитуда}, \textit{абсолютное 	значение}), \index{магнитуда}\index{абсолютное значение} определяемое как максимум модулей точек из интервала 
\begin{equation*} 
	\index{магнитуда}\index{модуль интервала}  
	|\mbf{a}|\  = \;\max\,\{ \ |a| \mid a\in\mbf{a} \ \} \ 
	= \;\max\,\{ \ |\un{\mbf{a}}|, |\ov{\mbf{a}}| \ \}.  
\end{equation*} 
Модуль интервала --- это наибольшее отклонение его точек от нуля. 

С другой стороны, величина, показывающая, насколько 
далеко отделен от нуля тот или иной интервал вне зависимости от его знака
называется \emph{мигнитудой}, которая определяется как  \index{мигнитуда} 
\begin{equation*} 
	\langle\mbf{a}\rangle\  = \;\min\,\{ \ |a| \mid a \in \mbf{a}\,\} \ 
	= \left\{ 
	\begin{array}{cl}
		\min\,\{ \ |\un{\mbf{a}}|, |\ov{\mbf{a}}| \ \}, & \text{ если } 0\not\in\mbf{a},\\[2mm]  
		0, & \text{ если } 0\in\mbf{a}. 
	\end{array} 
	\right. 
\end{equation*} 

Среди интервалов особую роль играют интервалы вида $[-a, a]$, имеющие своей серединой 
нуль. Их называют \emph{уравновешенными}. Среди всех интервалов 
с данным абсолютным значением (модулем) именно уравновешенные интервалы имеют наибольшую 
ширину. И наоборот, среди интервалов фиксированной ширины\index{уравновешенный интервал} 
уравновешенные интервалы имеют наименьшее абсолютное значение. 

Интервал полностью определяется двумя своими концами и представляет 
собой объект, который не несет никакой дополнительной структуры. Все точки интервала 
равноценны (равнозначны, равновозможны), и для каждой из них интервал дает двустороннее 
приближение. 

В частности, интервал $\mbf{a}$ нельзя отождествлять с равномерным 
вероятностным распределением на $[\un{\mbf{a}}, \ov{\mbf{a}}]$ с плотностью 
$1/(\w\mbf{a})$, так как в пределах $\mbf{a}$ с тем же успехом может быть определено 
любое другое вероятностное распределение или даже какое-то распределение, меняющееся 
во времени, --- случайный процесс. 

{\bf Отношения между интервалами.}  
Интервалы являются множествами, составленными из вещественных чисел. Большую роль для них играют теоретико-множественные отношения и операции 
(объединение, пересечение и др.). Особенно важно \emph{отношение включения} одного интервала в другой:  \index{отношение включения}
\begin{equation} 
	\label{InclInteOrder} 
	\mbf{a}\subseteq\mbf{b} \  \text{ равносильно тому, что } \ 
	\un{\mbf{a}}\geq\un{\mbf{b}}\;\text{ и }\;\ov{\mbf{a}}\leq\ov{\mbf{b}}.  
\end{equation} 
Отношение включения является \emph{частичным порядком} и превращает множество интервалов в частично упорядоченное множество.
 \index{частичный порядок}

Помимо порядка по включению на множестве интервалов большую роль играют также другие 
отношения, которые обобщают линейный порядок <<$\leq$>> на вещественной 
оси $\mbb{R}$. \index{линейный порядок}
Порядок <<$\leq$>> между вещественными числами может быть обобщен на интервалы многими способами. Важную роль играет следующее упорядочение.
\begin{definition} 
	Для интервалов $\mbf{a}$, $\mbf{b}\in\mbb{IR}$ условимся считать, что
	\textsl{$\mbf{a}$ не превосходит $\mbf{b}$} и писать <<$\,\mbf{a}\leq 
	\mbf{b}$>> тогда и только тогда, когда $\,\un{\mbf{a}}\leq\un{\mbf{b}}\,$
	и $\,\ov{\mbf{a}}\leq\ov{\mbf{b}}$.  \par\noindent
	Интервал называется \textsl{неотрицательным}, т.\,е. <<$\,\geq 0$>>, если 
	неотрицательны оба его конца. Интервал называется \textsl{неположительным}, 
	т.\,е. <<$\,\leq 0$>>, если неположительны оба его конца. 
\end{definition}

{\bf Теоретико-множественные операции между интервалами.}    
Если интервалы $\mbf{a}$ и $\mbf{b}$ имеют непустое пересечение, т.\,е. $\mbf{a}
\cap \mbf{b} \neq\varnothing$, то можно дать простые выражения для результатов 
теоретико-множественных операций пересечения и объединения через концы этих интервалов 
\begin{align*} 
	\mbf{a}\cap\mbf{b} = 
	\bigl[\max\{\un{\mbf{a}}, \un{\mbf{b}}\}, \min\{\ov{\mbf{a}}, \ov{\mbf{b}}\}\bigr], 
%	\\[5pt] 
\quad
	\mbf{a}\cup\mbf{b} = 
	\bigl[\min\{\un{\mbf{a}}, \un{\mbf{b}}\}, \max\{\ov{\mbf{a}}, \ov{\mbf{b}}\}\bigr].  
\end{align*} 
Если же $\mbf{a}\cap\mbf{b} = \varnothing$, т.\,е. интервалы $\mbf{a}$ и $\mbf{b}$ 
не имеют общих точек, то эти равенства уже неверны. 

Обобщением операций пересечения и объединения являются операции взятия \emph{минимума} 
и \emph{максимума} относительно включения <<$\subseteq$>>:  \index{минимум и максимум относительно включения <<$\subseteq$>>}
\begin{align} 
	\mbf{a}\wedge\mbf{b} &= \label{InteMinExpr}
	\bigl[\max\{\un{\mbf{a}}, \un{\mbf{b}}\}, \min\{\ov{\mbf{a}}, \ov{\mbf{b}}\}\bigr], 
	\\[2mm]
	\mbf{a}\vee\mbf{b} &= \label{InteMaxExpr}
	\bigl[\min\{\un{\mbf{a}}, \un{\mbf{b}}\}, \max\{\ov{\mbf{a}}, \ov{\mbf{b}}\}\bigr].  
\end{align} 
Первая из этих операций, 
<<$\wedge$>>, не всегда выполнима во множестве обычных интервалов, но это затруднение 
преодолевается посредством расширения множества интервалов специальными элементами 
--- неправильными интервалами (см.  п.~\ref{KaucherArithmSect}). 


\section[Классическая интервальная арифметика]% 
{Классическая \\* интервальная арифметика} \label{ClassicArithmSect} 

Значения физических (и иных) величин входят в математические выражения для физических законов, в различные формулы, в которых используются арифметические операции.
После определения интервалов, приходим к необходимости введения операций и отношений между ними. 

Наиболее естественным является определение результата интервальной операции <<по представителям>>,
как множества всевозможных результатов этой же операции между числами из интервалов. 
Например, для двухместной операции <<$\star$>> можно 
считать, что 
\begin{equation} 
	\label{IAMainPrinciple} 
	\mbf{a}\star\mbf{b}\; = 
	\;\bigl\{\, a\star b \mid a\in\mbf{b}, \,b\in\mbf{b}\, \bigr\}.   
\end{equation} 
Аналогичным образом определяются интервальные аналоги для одноместных 
операций. 

Если рассматриваются арифметические операции, т.\,е. $\star\in\{ +, -, \cdot, / \}$, 
то нетрудно показать, что задаваемые правилом \eqref{IAMainPrinciple} множества являются интервалами, исключая 
случай деления на интервал 
$\mbf{b}$, который содержит нуль \cite{SSharyBook}. 

Конструктивные 
формулы, расшифровывающие этот общий принцип для отдельных арифметических операций, 
выглядят следующим образом: 
\begin{align}
	& \mbf{a} + \mbf{b} = \left[\,\un{\mbf{a}} + \un{\mbf{b}},\,\ov{\mbf{a}}
	+\ov{\mbf{b}}\,\right]; \label{Addition}\\[5pt]
	& \mbf{a} - \mbf{b} = \left[\,\un{\mbf{a}} - \ov{\mbf{b}},\,\ov{\mbf{a}}
	- \un{\mbf{b}}\,\right];  \label{Subtraction} %\\[5pt]
\end{align}	
\begin{align}
	& \mbf{a}\cdot\mbf{b} = \left[\,\min\{\un{\mbf{a}}\,\un{\mbf{b}},
	\un{\mbf{a}}\,\ov{\mbf{b}},\ov{\mbf{a}}\,\un{\mbf{b}},\ov{\mbf{a}}\,
	\ov{\mbf{b}}\}\right.,\left. \max \{\un{\mbf{a}}\,\un{\mbf{b}},
	\un{\mbf{a}}\,\ov{\mbf{b}},\ov{\mbf{a}}\,\un{\mbf{b}},\ov{\mbf{a}}\,
	\ov{\mbf{b}}\}\,\right];  \label{Multiplication}\\[5pt]
	& \mbf{a}/\mbf{b} = \mbf{a}\cdot\left[\,1/\ov{\mbf{b}},\,1/\un{\mbf{b}},
	\,\right]\qquad\mbox{ для } \ \mbf{b}\not\ni 0.  \label{Division}
\end{align}  

Множество всех интервалов вещественной оси с операциями сложения, вычитания, 
умножения и деления, которые определены посредством \eqref{Addition}--\eqref{Division},  \index{классическая интервальная арифметика} 
\index{интервальная арифметика классическая} 
называется \textit{классической интервальной арифметикой}, и часто его обозначают 
также $\mbb{IR}$.                

\begin{example}{Расчет силы тока}
	Пусть максимальное напряжение  в сети переменного тока находится 
	в пределах интервала $[220, 230]$ \,Вольт, а сопротивление нагревателя меняется, 
	% в зависимости при увеличении температуры 	(конкретно, на 65 $^{\circ}C$) 
	в пределах $[20, 25]$ Ом. 
	Каким будет ток в этом участке цепи? 
	
	Для расчета используем закон Ома, который дает выражение для тока в виде 
	\begin{equation*}
		I = \frac{U}{R},    
	\end{equation*}
	где $U$ --- напряжение на участке цепи; $R$ --- ее сопротивление. Подставляя вместо 
	этих переменных интервалы их изменения и заменяя операцию деления на интервальное 
	деление \eqref{Division}, получим интервал значений максимального тока через нагреватель
	\begin{equation*}
	\mbf{I}_{\tt max}	 = \frac{[220, 240]\;\text{Вольт}}{[20, 25]\;\text{Ом}} \ 
		= \  \bigr[\,\tfrac{220}{25}, \tfrac{240}{20}\,\bigr]\;\text{А} \  
		\approx \  [8,8, 12,0]\;\text{А}. 
	\end{equation*} 
\end{example} 	
	Это точный интервал значений тока, так как математическое выражение для него является  простым и позволяет точно оценивать свою область значений с помощью классической 
	интервальной арифметики. Для более сложных выражений оценка, полученная приведенным \index{внешняя оценка области значений}
	выше простым способом, может не быть равной области значений, но лишь содержит ее 
	или, как часто говорят, является ее \emph{внешней оценкой}.

Отметим важный частный случай интервального умножения, произведение числа на интервал: 
\begin{equation*} 
	\label{NumIntProduct}
	a\cdot\mbf{b} \ = \ 
	\left\{ 
	\begin{array}{ll}
		\bigl[\,a\un{\mbf{b}}, a\ov{\mbf{b}}\,\bigr], & \text{если} \  a\geq 0, \\[2mm] 
		\bigl[\,a\ov{\mbf{b}}, a\un{\mbf{b}}\,\bigr], & \text{если} \  a\leq 0. 
	\end{array} 
	\right. 
\end{equation*} 

Алгебраические свойства интервальной арифметики
являются необычными. 
Операции сложения и умножения не связаны друг с другом привычным соотношением 
дистрибутивности. Вместо него имеет место более слабая \textit{субдистрибутивность}: 
\begin{equation*} 
	\index{субдистрибутивность} 
	\mbf{a}\,(\mbf{b} + \mbf{c}) \,\subseteq\, \mbf{a}\mbf{b} + \mbf{a}\mbf{c}. 
\end{equation*} 
Например, $[0, 1] \cdot (1 - 1) = 0 \;\subset\; [-1, 1] = [0, 1]\cdot 1 + [0, 1]\cdot(-1)$. 
\index{субдистрибутивность}

\emph{Интервальный вектор} --- это упорядоченный набор интервалов. 
Множество интервальных $n$-векторов, 
компоненты которых принадлежат $\mbb{IR}$, обозначаем через $\mbb{IR}^n$. 
Интервальные векторы называют также \textit{брусами}, поскольку геометрическим 
образом векторов являются прямоугольные параллелепипеды с гранями, параллельными координатным 
осям в $\mbb{R}^n$.\index{брус} 

\emph{Интервальная матрица} --- это матрица с интервальными элементами, т.\,е. прямоугольная 
таблица, заполненная интервалами.\index{интервальная матрица} 

Интервальные векторы и матрицы являются специальным классом 
множеств в многомерных пространствах. С ними  удобно работать, с их 
помощью оценивают другие множества, возникающие при решении математических задач. 
В этой связи чрезвычайно важно следующее понятие. 
\begin{definition}
	\index{оболочка интервальная}\index{интервальная оболочка}  
	Если $S$ --- непустое ограниченное множество в $\mbb{R}^n$ или $\mbb{R}^{m\times n}$, 
	то его \textsl{интервальной оболочкой} $\ih S$ называется наименьший по включению 
	интервальный вектор (или матрица), содержащий $S$. 
\end{definition}

\section{Примеры интервальных расчетов} 


Обсуждение зависимости интервальных оценок областей значений выражений от их вида 
содержится в \cite{SSharyBook}.  Имеет место <<основная теорема>> интервальной арифметики.

	\addvspace{\bigskipamount}\noindent
\textbf{Теорема.}
\index{основная теорема интервальной арифметики}
{\sl Пусть $f(\,x_{1}, \ \ldots, \ x_{n})$ --- рациональная функция вещественных 
	аргументов $ \ x_{1}$, \ \ldots, \ $x_{n} \ $ и для нее определен результат  $\mbf{f}\NExt 
	(\,\mbf{x}_{1}, \ \ldots, \ \mbf{x}_{n})$ подстановки вместо аргументов интервалов их 
	изменения $ \mbf{x}_{1}$, $\mbf{x}_{2}$, \ \ldots, \ $\mbf{x}_{n}\in\mbb{IR} $ и выполнения 
	всех действий над ними по правилам интервальной арифметики. Тогда 
	\begin{equation}
	\label{MainIArInclu}
	\bigl\{ \ f( x_{1}, \ \ldots, \ x_{n}) \bigm| x_{1}\in\mbf{x}_{1}, \
	\ldots, \ x_{n}\in\mbf{x}_{n} \ \bigr\} \;\subseteq \  
	\mbf{f}\NExt(\mbf{x}_{1}, \ \ldots, \ \mbf{x}_{n}),
	\end{equation}
	т.\,е. $\mbf{f}\NExt(\mbf{x}_{1}, \ \ldots, \ \mbf{x}_{n})$ содержит множество значений 
	функции $f$ на брусе $(\mbf{x}_{1}, \ \ldots, \ \mbf{x}_{n})$. Если выражение для $f(x_{1}, \	\ldots, \ x_{n})$ содержит не более чем по одному вхождению каждой переменной в первой 
	степени, то в \eqref{MainIArInclu} вместо включения выполняется точное равенство.} 


Приведем пример интервальных расчетов с формулами, которые встречаются в естественнонаучных законах. 

%%%%%%%%%%%%%%%%%%%%%%%%%%%%%%%%%%%%%%%%%%%%%%%%%%%%%%%%%%%%%%%%%%%%%%%%%%%%%%%%%%%%%%%% 

\begin{example}{Уравнение катализа.} 
	Для описания зависимости скорости реакции, катализируемой ферментом, от концентрации субстрата, используется формула 	\cite{MichaelisMenten}:   
	\begin{equation} 
		\label{MichaelisMenten} 
		v \; = \; V_{\max}\, \frac{S}{S+K_M},
	\end{equation}
	где $v$ --- скорость реакции; $V_{\max} $ --- максимальная скорость реакции; 
	$K_M$ --- константа Михаэлиса; $S$ ---  концентрация субстрата. 
	
	Для иллюстративного расчета зададимся значением $V_{\max} =  1$.
	Возьмем  конкретную реакцию со справочным значением $K_M = 1,44 \ \cdot 10^{-4} $, а интервал концентрации 	$\mbf{S}$ равным интервалу с серединой $K_M$ и 10\%-ным радиусом, т.\,е. 
	\begin{equation*} 
		\mbf{S} = [1, 2959, 1, 5841] \cdot 10^{-4}. 
	\end{equation*} 
	
	Для вычисления  $\mbf{v}$ выражение \eqref{MichaelisMenten} представим двумя способами: 
	в исходном виде и с делением числителя и знаменателя на $S$. Во втором случае переменная 
	$\mbf{S}$ входит в выражение \eqref{v22} один раз, и согласно основной теореме интервальной 
	арифметики результат естественного интервального оценивания совпадает с точной областью 
	значений выражения:  
	\begin{align}
		\mbf{v}_1 &= V_{\max}\;\frac{\mbf{S}}{\mbf{S}+K_M} = [0, 42857, 0 ,57895], \label{v11}\\[2mm] 
		\mbf{v}_2 &= V_{\max}\;\frac{1}{1+K_M/\mbf{S}} = [0, 47368, 0, 52381]. \label{v22}
	\end{align} 
	
	Средние величины интервалов \eqref{v11} и \eqref{v22} отличаются незначительно: 
	\begin{equation*} 
		\m \mbf{v}_1 =  0,5038  \   \approx \m \mbf{v}_2 =  0, 4987. 
	\end{equation*} 
	В то же время радиусы результатов вычислений $ \mbf{v}$ для выражений \eqref{v11} 
	и \eqref{v22} существенно различны: 
	\begin{align*}
		\r \mbf{v}_1 =  0,075188, \quad 	\r \mbf{v}_2 = 0,025063.
	\end{align*} 
	При этом имеет место соотношение:
	\begin{equation*} 
		\r \mbf{v}_1 > \r \mbf{v}_2
	\end{equation*} 
	в силу неоднократного вхождения $\mbf{S}$ в выражение \eqref{v11}. 
\end{example} 

	
\section[Полная интервальная арифметика Каухера]%  
{Полная интервальная \\* арифметика Каухера} 
\label{KaucherArithmSect} 

Помимо классической интервальной арифметики часто возникает необходимость работать 
с полной интервальной арифметикой Каухера $\mbb{KR}$. Она является 
алгебраическим и порядковым пополнением арифметики $\mbb{IR}$, подобно тому, 
как множество целых чисел пополняет натуральный ряд. 

Элементами арифметики $\mbb{KR}$ являются пары чисел, взятые в квадратные скобки,  
вида $[\alpha, \beta]$, которые будем называть \textit{интервалами}. 
При этом возможны ситуации, когда $\alpha\leq\beta$ или $\alpha > \beta$. Если 
$\alpha\leq \beta$, то $[\alpha, \beta]$ обозначает обычный интервал вещественной оси, 
его называют \textit{правильным}. Если же $\alpha >  \beta$, то $[\alpha, \beta]$ 
--- \textit{неправильный интервал}. 

Таким образом, \index{интервальная арифметика Каухера} \index{правильный интервал}
\index{неправильный интервал} 
$$\mbb{IR}\subset\mbb{KR}.$$ 

Неправильные интервалы можно рассматривать как математические абстракции (аналогичные отрицательным или мнимым числам), которым могут быть даны осмысленные 
физические интерпретации. В данном учебном пособии полная интервальная арифметика 
Каухера $\mbb{KR}$ и неправильные интервалы, по существу, возникают при математической 
обработке интервальных результатов наблюдений и измерений. 

Правильные и неправильные интервалы переходят друг в друга 
в результате отображения \emph{дуализации} \index{отображение дуализации}  $\,\dual : \mbb{KR}\to \mbb{KR}$, меняющего местами  концы интервала, т.\,е. такого, что
\begin{equation*}
	\label{Dualization}
	\dual\mbf{a} := [\;\ov{\mbf{a}},\,\un{\mbf{a}}\;].
\end{equation*} 
\textit{Правильной проекцией} интервала \index{правильная проекция} $\mbf{a}$ из $\mbb{KR}$ называется интервал, 
обозначаемый $\pro\mbf{a}$ и такой, что 
\begin{equation*} 
	\pro\mbf{a} = 
	\left\{ \ 
	\begin{array}{cl}
		\mbf{a}, & \text{если $\mbf{a}$ --- правильный,} \\[1mm] 
		\dual\mbf{a}, & \text{если $\mbf{a}$ --- неправильный.} 
	\end{array} 
	\right. 
\end{equation*} 
С помощью правильной проекции из произвольного интервала получается его <<правильный 
образ>>, с которым можно обращаться как с обычным числовым интервалом в $\mbb{R}$.


Арифметические операции между интервалами в $\mbb{KR}$ продолжают операции в $\mbb{IR}$. 
В частности, формулы \eqref{Addition}, \eqref{Subtraction} для сложения и вычитания 
также определяют сложение и вычитание в $\mbb{KR}$. Умножение и деление между интервалами 
из $\mbb{KR}$ определяется более сложно, и их описание можно найти в \cite{SSharyBook}. 

Наиболее важным в интервальной арифметике Каухера является обратимость 
арифметических операций. В частности, для любого интервала имеется противоположный ему, 
т.\,е. обратный по сложению. Для интервалов, не содержащих нуль, имеются обратные к ним 
по умножению. Для сложения \eqref{Addition} обратной операцией является не операция 
интервального вычитания \eqref{Subtraction}, а операция <<алгебраическое вычитание>>, которую 
обозначают знаком <<$\ominus$>>: \index{алгебраическое вычитание}
\begin{equation}
	\label{AlgebrMinus} 
	\mbf{a}\ominus\mbf{b} 
	= [\un{\mbf{a}} - \un{\mbf{b}}, \ov{\mbf{a}} - \ov{\mbf{b}}].
\end{equation} 
Иногда в математических текстах тем же символом обозначается так называемая <<разность 
Хукухары>> двух множеств (Hukuhara difference), но она имеет  другие смысл 
и назначение. Нетрудно проверить, что для любых интервалов $\mbf{a}$, $\mbf{b}$ 
из $\mbb{KR}$ имеют место равенства 
\index{разность	Хукухары}
\begin{equation*} 
	\mbf{a}\ominus\mbf{a} = 0;  \hspace{15mm} 
	(\mbf{a} + \mbf{b})\ominus\mbf{b} = \mbf{a}; \hspace{15mm} 
	(\mbf{a}\ominus\mbf{b}) + \mbf{b} = \mbf{a}. 
\end{equation*} 
\begin{example}{Противоположный интервал}.
	Для интервала $[1, 2]$ противоположным по сложению является интервал $[-1, -2]$. 
	Это неправильный интервал   
	\begin{equation*} 
		[1, 2] + [-1, -2] = [1-1, 2-2] = [0, 0] = 0,   
	\end{equation*} 
	т.\,е. в сумме с исходным интервалом он дает нейтральный элемент $0$. Отметим, что 
	для обычного интервального вычитания 
	\begin{equation*} 
		[1, 2] - [1, 2] = [1-2, 2-1] = [-1, 1].   
	\end{equation*} 
	Это иллюстрирует отмеченный ранее факт, что обычное интервальное вычитание не является 
	операцией, обратной интервальному сложению. 
\end{example} 
\emph{Абсолютное значение} интервалов из $\mbb{KR}$ определяется как абсолютное 
значение их правильных проекций, т.\,е. 
\begin{equation*} 
	|\mbf{a}|\  = \;\max\,\{\,|\un{\mbf{a}}|, |\ov{\mbf{a}}|\,\}.  
\end{equation*} 
Полная интервальная арифметика Каухера $\mbb{KR}$ пополняет классическую интервальную 
арифметику $\mbb{IR}$ не только в алгебраическом смысле, но также и относительно 
естественного порядка по включению <<$\subseteq$>>. 
\begin{definition} 
%	\setcounter{IncluDefi}{\value{DefNum}} 
	Будем говорить, что для интервалов $\mbf{a}$, $\mbf{b}\in\mbb{KR}$ выполняется 
	включение $\mbf{a}\subseteq\mbf{b}$, если  
	\begin{equation*} 
		\un{\mbf{a}}\geq\un{\mbf{b}}\quad\text{ и }\quad\ov{\mbf{a}}\leq\ov{\mbf{b}}, 
	\end{equation*} 
	т.\,е. справедливы те же соотношения \eqref{InclInteOrder} между концами интервалов, 
	что и в случае классической интервальной арифметики. 
\end{definition} 
Относительно введенного таким образом отношения включения в $\mbb{KR}$ для любых двух 
интервалов существует минимальный и максимальный по включению, т.\,е. результаты 
операций $\mbf{a}\wedge\mbf{b}$ и $\mbf{a}\vee\mbf{b}$  всегда определены. 

\begin{example}{Минимум и максимум по включению в полной интервальной арифметике}:  
	\begin{equation*} 
		[1, 2]\wedge [3, 4] = [3, 2], \hspace{18mm} [1, 2]\vee [3, 4] = [1, 4]. 
	\end{equation*} 
\end{example} 
\index{минимум и максимум относительно включения <<$\subseteq$>>}

{\bf Расстояние на множестве интервалов.}  
Расстояние между интервалами $\mbf{a}$ и $\mbf{b}$ из $\mbb{IR}$ или $\mbb{KR}$ 
определяется как                            \index{расстояние} 
\begin{equation}
	\label{InteDist}
	\dist(\mbf{a}, \mbf{b}) \  = \  \max 
	\bigl\{|\un{\mbf{a}} - \un{\mbf{b}}|, 
	|\ov{\mbf{a}} - \ov{\mbf{b}}| \bigr\}.
\end{equation}
Расстояние \eqref{InteDist} обладает всеми свойствами абстрактного расстояния (метрики).
Легко убедиться, что 
\begin{equation*}
	\dist(\mbf{a}, \mbf{b}) \  = \  |\mbf{a}\ominus\mbf{b}|.
\end{equation*}
Эта формула является полным аналогом расстояния между точками вещественной оси, как 
модуля их разности, т.\,е. $|a - b|$. 

Справедливо также следующее равносильное представление расстояния 
\eqref{InteDist} между интервалами: 
\begin{equation*}
	\dist(\mbf{a}, \mbf{b}) \  = \  |\m\mbf{a} - \m\mbf{b}| + |\r\mbf{a} - \r\mbf{b}|.
\end{equation*}
\begin{example}{Расстояния между интервалами}.
	Рассмотрим интервал $[3, 7]$ и точку $4$ внутри него. Расстояние от этой точки,  
	отождествляемой с вырожденным интервалом $[4, 4]$, до данного интервала равно 
	\begin{equation*} 
		\dist(4, [3, 7]) = \max\bigl\{|4 - 3|, |4 - 7|\bigr\} = 3. 
	\end{equation*} 
	Рассмотрим дуальный интервал к интервалу $[3, 7]$. Это интервал $\dual[3, 7] = [7, 3]$.  
	Расстояние его до исходного интервала равно $\dist([3, 7], [7, 3]) = 4$. 
\end{example}
Расстояние важно для определения отклонения интервалов друг от друга и, как следствие, 
для определения погрешности интервальных измерений. Полная интервальная арифметика реализована С.\,И.\,Жилиным на языке {\tt Octave} \cite{OctaveKaucher}. \index{Octave}

%%%%%%%%%%%%%%%%%%%%%%%%%%%%%%%%%%%%%%%%%%%%%%%%%%%%%%%%%%%%%%%%%%%%%%%%%%%%%%%%%%%%%%%%

	\section{Оценки и погрешности измерений}\label{EstimatesMeasurements}

  \subsection{Оценки точечные и интервальные} 
\index{оценки точечные и интервальные}

Следуя \cite{MetodikaBook}, опишем два вида оценок в традиционной и интервальной статистиках. 	
Оценки величин могут быть \emph{точечными} или \emph{интервальными}. 

\emph{Точечные оценки}, т.\,е. оценки в виде точек --- чисел, векторов или матриц, --- 
соответствуют, как правило, тому типу данных, который используется в модели 
рассматриваемого объекта или явления, и могут непосредственно использоваться 
при его дальнейшем исследовании, прогнозировании его поведения и т.\,п. 

\emph{Интервальные оценки} дают области возможных значений точечных оценок и нужны 
для характеризации их возможного разброса и изменчивости (\emph{вариабельность},\index{вариабельность} см. пп.~\ref{ConstVariabSect} 
и \ref{VariabilitySect}). Так как в традиционной вероятностной статистике оценки 
параметров являются случайными величинами, а носители их вероятностных 
распределений могут быть неограниченными, то при определении интервальных оценок 
обычно задают некоторый \emph{уровень значимости} или \emph{доверительной 
	вероятности}, с помощью которых выполняют усечение вероятностного распределения. 
Тем самым всегда обеспечивается ограниченность интервальных оценок и их практичность. 

В интервальном анализе данных оценки величин также могут быть \emph{точечными} 
либо \emph{интервальными} или даже иметь форму других множеств. Точечная оценка 
несет тот же смысл, что и в традиционной статистике, а интервальная оценка тоже дает 
область возможных значений точечных оценок, характеризуя их возможный разброс и 
вариабельность. Многомерные интервальные оценки удобнее всего брать в форме брусов. 

Но есть и существенные отличия от вероятностной статистики. 
Во-первых, задание уровня  значимости не требуется, так как множества значений оценки, как правило, являются ограниченными. 
Во-вторых, интервальные оценки могут иметь различных смысл --- быть внутренними, 
внешними или какими-нибудь другими, сообразно чему их смысл различен. 
В-третьих, в пределах внутренней интервальной оценки все значения равноценны и тоже могут являться точечными оценками рассматриваемой величины. 
Напротив, в традиционной вероятностной статистике точечные значения внутри интервальной 
оценки не вполне равноценны друг другу. 

\subsection{Измерения и их результаты} 
\label{MeasuResultSect}

	Основным понятием теории обработки наблюдений является понятие \emph{<<измерения>>} (\emph{<<наблюдения>>}). Слово <<измерение>> имеет много значений. Оно может 
обозначать как процесс измерения или наблюдения, так и его результат. Из контекста обычно 
бывает ясно, какое значение слова имеется в виду \cite{MetodikaBook}.  
\vspace{-2mm}
\begin{definition}       
	\textsl{Измерением (замером, наблюдением)} будем называть измеренное значение величины. 
	\index{измерение}\index{замер} 
\end{definition}
\vspace{-2mm}
\noindent
По способу получения результата измерения все процессы измерения разделяются 
в \cite{Malikov} на \emph{прямые}, \emph{косвенные} и \emph{совокупные}. 

При прямых измерениях объект исследования приводят в непосредственное взаимодействие 
со средством измерений, которое выдает результат. \index{измерения прямые}  
При косвенных измерениях значение измеряемой величины находят на основании \index{измерения косвенные} 
известной зависимости между измеряемой величиной и искомой величинами. 

При совокупных измерениях значения искомых величин определяются из системы (совокупности)  \index{измерения совокупные} уравнений. 


Приведенная классификация весьма условна. Следует отметить, что результат измерения является \emph{итогом} какого-то физического эксперимента, в котором получаются первичные измерения, и \emph{последующего применения некоторого способа	математической обработки} первичных измерений. 

На практике измерение (замер, наблюдение) может представлять собой вещественное число или интервал или же составленные из них многомерные объекты (вектор, матрицу, интервальный вектор, 
интервальную матрицу и т.\,п.). 
Вещественный тип данных для измерений является традиционным. 
Каким образом в результате измерений могут быть получены интервалы? Приведем ряд примеров.

{\bf Погрешности квантования.}  
Это инструментальная погрешность, возникающая при преобразовании величины, принимающей 
непрерывный ряд значений, в цифровую форму, которая может принимать дискретный набор 
допустимых уровней. Значение преобразуемого аналогового сигнала 
заменяется ближайшим разрешенным уровнем цифрового сигнала, что дает  погрешность квантования. \index{погрешность квантования} 
Ее часто называют также \emph{погрешностью оцифровки}. 
\index{погрешность оцифровки} 

Погрешности квантования присущи всем аналого-цифровым преобразователям. Если мы используем интервальный тип данных, интервальные результаты измерений, то результат представления непрерывного сигнала $t$, 
не равный точно какому-либо допустимому уровню $t_0$, $t_1$, \ \ldots, \ $t_p$, может быть 
записан как интервал $[t_{i}, t_{i \, + \, 1}]\ni t$,  и такое представление точное (см. п.~\ref{NonCoverSampleSect}).

{\bf Неопределенность измерения нуля.}
Согласно \cite{RMG29-2013} \emph{погрешностью нуля} называется погрешность средства 
измерений в контрольной точке, когда заданное значение измеряемой величины равно нулю. 
\index{погрешность нуля} Следовательно, неопределенность измерений нуля --- это 
неопределенность измерений, когда заданное значение измеряемой величины равно нулю (см. п.~\ref{NonCoverSampleSect}). 

{\bf Агрегирование результатов многократных наблюдений.} 
Во многих практических ситуациях измерение интересующей величины выполняется 
для надежности многократно. Тем не менее повторные измерения одних и тех же 
явлений не показывает  в пределах точности измерений совпадений результатов. 
В  данном случае результатом серии повторяющихся измерений можно взять интервал 
от минимального до максимального из полученных результатов, т.\,е. агрегировать 
(объединить) результаты отдельных измерений. \index{агрегирование результатов} 
Математически, если результаты повторных измерений величины равны $x_1$, $x_2$, \
\ldots, \ $x_n$, то интервальным результатом следует взять 
\begin{equation*} 
	\mbf{x} = \bigl[\,\min_{1\leq \,i \,\leq n} x_{i}, \max_{1\leq \,i \,\leq n} x_{i}\,\bigr].  
\end{equation*} 
Будем называть этот способ получения интервального результата измерения 
\emph{агрегированием}. 

Используя введенные выше (см. п.~\ref{ClassicArithmSect}) операции взятия интервальной 
оболочки множества и максимума по включению \eqref{InteMaxExpr}, этот результат 
можно записать следующим равносильным образом: 
\begin{equation*} 
	\mbf{x} = \ih\{ x_1, x_2, \ \ldots, \ x_n\} 
\end{equation*} 
или 
\begin{equation*} 
	\mbf{x} = \bigvee_{1\leq i\leq n} x_{i}.  
\end{equation*} 
Эти представления хороши тем, что могут быть обобщены на более сложные случаи 
(см. п.~\ref{NonCoverSampleSect}).  

{\bf Погрешности измерений и наблюдений.} \label{ParErrorModel}
Интервалы в результатах измерений могут возникать различным способом. Они могут 
получаться сразу в виде готовых интервалов, но могут возникать в результате коррекции 
точечных результатов. 

Один из распространенных способов получения интервальных результатов в первичных измерениях --- это обинтерваливание точечных значений, когда к точечному \emph{базовому значению} 
$\mathring{x}$, которое считывается по показаниям измерительного прибора, прибавляется 
\emph{интервал погрешности} $\mbf{\epsilon}$: 
\begin{equation} 
	\label{GeneralErrorModel} 
	\index{базовое значение}
	\index{интервал погрешности} 
	\mbf{x} = \mathring{x} + \mbf{\epsilon}.  
\end{equation} 
Интервал погрешности, вообще говоря, может быть произвольным, но если он уравновешен, 
т.\,е. 
\begin{equation*} 
	\mbf{\epsilon} = [-\epsilon, \epsilon] \quad\text{ для некоторого } \epsilon > 0, 
\end{equation*} 
то иногда для прямых измерений это можно трактовать, как отсутствие систематических 
погрешностей. 

\vspace{-2mm}   
\begin{example}{Интервал показаний измерителя силы тока}. 
	Предположим, что в процессе измерения силы тока мы смотрим на шкалу амперметра 
	и считываем измеренное значение --- 5,4 А. Класс точности используемого прибора --- 2, и это, по определению, 	максимально допустимое значение основной приведенной погрешности, выраженной 	в процентах. Следовательно, истинное значение измеряемого тока должно лежать 
	в интервале 
	\begin{equation*} 
		\bigl[ 5, 4 - 0, 02\cdot 5, 4, 5, 4 + 0, 02\cdot5,4\bigr] \ \text{А} \ 
		= \  [ 5, 292, 5, 508] \  \text{А}. 
	\end{equation*} 
\end{example} 
Интервал измерения строится с целью оценить истинное значение измеряемой величины, 
и получаемые при этом приближения могут быть  качественно различными. Они могут 
включать ( накрывать ) истинное значение, но они также могут его и не содержать. 

Выборка в вероятностной статистике --- это часть генеральной совокупности элементов, которая охватывается экспериментом (наблюдением, опросом). 
В данной работе будем называть \textit{выборкой}  
совокупность результатов измерений. Абстрактное понятие \emph{генеральной совокупности}, 
которая представляет собой совокупность всех мыслимых (но реально не существующих) 
наблюдений интересующей нас величины при заданных условиях эксперимента, в анализе 
интервальных данных не используется. \index{выборка}\index{генеральная совокупность} 

Каждое измерение из выборки в случае неточности описывается своим интервалом 
неопределенности. Погрешности и неопределенности многомерных величин могут описываться 
интервальными векторами, которые обычно называют брусами. 

Существуют также другие подходы к классификации интервальных данных, что может использоваться при выборе способа их обработки. 

{\bf Классификация по ширине интервалов.}  \label{InteWidClass}
Прежде всего, имеет смысл различать интервальные данные по ширине интервалов, 
т.\,е. по величине имеющейся у них неопределенности. 

Если интервальные данные являются узкими, почти совпадая 
с точечными величинами, то для них интервальная специфика выражена слабо или вовсе 
не выражена. В некоторых ситуациях для их обработки можно даже применять те подходы 
и алгоритмы, которые используются для неинтервальных (точечных) данных. При этом 
<<небольшая интервальность>> узких интервалов позволяет выполнять с ними упрощенные, но 
достаточно точные приемы обработки, основанные на асимптотических разложениях, пренебрежении 
членами высокого порядка и т.\,п. 

При увеличении ширины интервальных данных их  уже нельзя рассматривать как 
<<приближенно точечные>>, они становятся <<существенно интервальными>>,  
но несут некоторые черты, присущие точечным данным. Отсутствие пересечений интервальных измерений выборки является признаком того, что интервальные данные все еще <<не слишком широки>> и не слишком сильно отличаются от точечных данных. 

Наконец, при дальнейшем увеличении ширины интервальных измерений в выборке они начинают 
пересекаться друг с другом, и это служит признаком следующего качественного состояния, 
--- когда интервальные данные являются <<широкими интервальными>>, т.\,е. интервальная 
неопределенность велика. Этот случай является специфически интервальным, к которому 
точечные методы обработки данных уже принципиально неприменимы. 

{\bf Классификация по способу измерения.}  
В \cite{NguyenKreinWuXiang} вводится классификация интервальных данных по способу 
их получения --- с помощью одного или нескольких измерительных устройств. 

С учетом различных способов измерений и различной точности измерительных инструментов 
нужно по-разному применять прием варьирования неопределенности в выборке 
(см. п.~\ref{ConstVariabSect}). 

\subsection{Более сложные типы интервалов}
\label{CompositeIntervalTypes}

Для описания данных разработана теория и более сложных типов интервалов.
Популярное описание двух таких типов, твинов и мультиинтервалов, дано в \cite{SPbSTU2021}.

{\bf Твины --- интервалы интервалов.} 
На практике концы интервалов, представляющие результаты измерений, могут быть 
известны неточно, поэтому  возникает необходимость работы с интервалами, имеющими  
интервальные концы. В интервальном анализе такие объекты называются 
\textit{твинами} (по-англ. twin, сокращение фразы \un{tw}ice \un{in}terval, 
<<двойной интервал>>).\index{твин} 
Развернутое изложении теории твинов дано в диссертации \cite{Nesterov1999}. 

Tвин, как <<интервал интервалов>>  или интервал с интервальными концами, можно 
представить как 
\begin{equation} 
	\label{Twin}
	\mbf{X} = 
	[\mbf{a}, \mbf{b}] = \bigl[\,[\un{\mbf{a}}, \ov{\mbf{a}}], [\un{\mbf{b}}, \ov{\mbf{b}}]\,\bigr].
\end{equation}

\begin{figure}[hbt]
	\centering\small 
	\setlength{\unitlength}{1mm}
	\begin{picture}(70,10)
		\put(0,0){\includegraphics[width=70mm]{Twinfig}}
		\put(-5,6.6){\vector(1,0){80}} \put(71.5,7.6){$\mbb{R}$} 
		\put(21,10){$\un{\mbf{a}}$} \put(30,10){$\ov{\mbf{a}}$} 
		\put(41,10){$\un{\mbf{b}}$} \put(50,10){$\ov{\mbf{b}}$} 
		\put(35,1){$\mbf{X}$}  
	\end{picture}
	\caption{Твины на вещественной оси} 
	\label{TwinsPic2} 
\end{figure}
На рис. \ref{TwinsPic2} твин $\mbf{X}$ представлен в графической форме. Концы твина, 
т.\,е. интервалы $\mbf{a}$ и $\mbf{b}$, представлены более темной заливкой, чем остальная часть 
твина. 

Твин является множеством всех интервалов, больших или равных $[\un{a}, \ov{a}]$ и меньших
или равных $[\un{b}, \ov{b}]$, и точное определение зависит от смысла, который вкладывается 
в понятия <<больше или равно>>, <<меньше или равно>>. 
Поскольку интервалы могут быть упорядочены различными способами, то существуют 
различные виды твинов. Двум основным частичным порядкам на $\mathbb{IR}$ и $\mathbb{KR}$, 
<<$\subseteq$ >> и <<$\leq$>>,  соответствуют два основных типа твинов. Разработаны 
различные операции с твинами, а также способы оценок значений функций от них. 

\begin{example}{Измерение температуры термометром сопротивления  в виде твина}.	
	В повседневной лабораторной и промышленной практике широко применяются 	термометры 
	сопротивления. Один из типов таких датчиков, платиновый термометр Pt100, имеет номинальное 
	сопротивление 100 Ом при температуре $ 0 \ ^{\circ}$ C и систематическую погрешность 
	\begin{equation*} 
		\Delta t =\pm 0, 35 \ ^{\circ}\text{ C}.
	\end{equation*} 
	Пусть измеряемая  температура находится в диапазоне [19, 5, 20, 5] $^{\circ}$ C, которую 
	представим как интервал  $\mbf{t}$: 
	\begin{equation}\label{temp}
		\mbf{t}= [19,5, 20,5] \ ^{\circ} \text{ C}.
	\end{equation}
	Представим границы $\un{\mbf{t}}, \ \ov{\mbf{t}}$ 
	интервала $\mbf{t}$ как интервалы. С учетом систематической погрешности твин температур 
	$\mbf{T}$, даваемый датчиком, составит 
	\begin{equation}
		\label{TwinTemp}
		\mbf{T} = \bigl[\,[19, 15, 19, 85], \ [20, 15, 20, 85] \,\bigr] \ ^{\circ}\text{ C}.
	\end{equation}
	
	Графическое представление твина $\mbf{T}$ \eqref{TwinTemp} дано на рис. \ref{TwinsTemp}. 
	\begin{figure}[ht]
		\centering\small 
		\setlength{\unitlength}{1mm}
		\begin{picture}(70,17)
			\put(0,0){\includegraphics[width=70mm]{TwinTemp}}
			\put(-5,6.6){\vector(1,0){80}} \put(71.5,7.6){$\mbb{R}$} 
			\put(18,10){{\footnotesize 19.15}} 
			\put(28,10){{\footnotesize 19.85}} 
			\put(37,12){{\footnotesize 20.15}} 
			\put(47,12){{\footnotesize 20.85}} 	
			\put(35,1){$\mbf{T}$}  
		\end{picture}
		\caption{Температура как твин.} 
		\label{TwinsTemp} 
	\end{figure}
	Форма записи температуры в виде твина $\mbf{T}$ \eqref{TwinTemp} четко
	и полно представляет информацию об измеряемых данных. В случае если концы интервала 
	в выражении \eqref{temp} могут меняться независимо, возможны различные ситуации. 
	Например, может оказаться, что значения температур для левого конца будут 
	выше, чем для правого. 
\end{example}
	
{\bf Мультиинтервалы.} \label{MultiIntervalSect} 	
В ряде разделов науки и техники имеют место ситуации, когда исследуемая величина  
содержится в неодносвязной области. 

Согласно определению, приведенному в \cite{SSharyBook}, \emph{мультиинтервал} 
--- это объединение конечного числа несвязных интервалов числовой оси 
(рис. \ref{MultiInterval}). \index{мультиинтервал}

\begin{figure}[ht]
	\centering
	\begin{picture}(70,13)
		\put(-90,-10){\includegraphics[width=0.7\textwidth]{Multifig.eps}}
		\put(-100,12){\vector(1,0){260}} \put(155,15){$\mbb{R}$} 
	\end{picture}	
	\caption{Мультиинтервал в $\mathbb{R}$.} %Рис. 1.11 из \cite{SharyBook}.}
\label{MultiInterval} 
\end{figure}

Между мультиинтервалами также могут быть определены арифметические операции 
<<по представителям>> аналогично тому, как это делается на множестве интервалов. 
Мультиинтервалы можно получать при решении уравнений и систем уравнений.

\vspace{-2mm}
\begin{example}{Пример мультиинтервалов и их преобразований}
Приведем модельный пример, в котором появляются неодносвязные интервалы. Рассмотрим задачу нахождения корня уравнения второй степени с различными значениями параметра $a$.
\begin{equation}\label{ParabolaMulti}
a \cdot x^2 =[0, 5, 1, 5]
\end{equation}


\begin{figure}[ht]
\centering
{\includegraphics[width=0.5\textwidth]{MultiIntervalParabola.png}}
\caption{Решение уравнения \eqref{ParabolaMulti}} с разными значениями параметра $a$.
\label{f:MultiIntervalParabola} 
\end{figure}

Возьмем для определенности значения $a=0, 6$ и $a=1, 2$. Получим решения 
уравнения \eqref{ParabolaMulti}: 
\begin{align*}
a=0, 6: & \quad \mbf{X}_1= \left[ \  [-2, 04,  -1, 58], \ [1, 58,   2, 04] \ \right], \\	
a=1, 2: & \quad \mbf{X}_2= \left[ \  [-1, 44, -1, 12], \ [1, 12,   1, 44] \ \right].
\end{align*}

Мультиинтервалы $\mbf{X}_1, \mbf{X}_2$ показаны на рис.~\ref{f:MultiIntervalParabola} 
соответственно парами отрезков синего и красного цвета. При изменении коэффициента 
при старшей степени полинома компоненты мультиинтервалов --- решений уравнения 
\eqref{ParabolaMulti} меняют и размер, и положение на вещественной оси. 
\end{example}	
	
	\section{Описание измерений}
	
\subsection{Накрывающие и ненакрывающие \\* измерения} 
\label{CoverMeasrSect} 

Результат измерения интересующей нас величины может получиться либо равным, либо не равным ее истинному значению.
В случае измерения непрерывных физических величин, принадлежащих вещественному типу данных, равенство 
является исключительным событием,  неустойчивым к сколь угодно малым возмущениям или погрешностям в вычислительных алгоритмах. 

Принципиально другая ситуация возникает, если результат измерения может быть интервалом. 
Невырожденный интервал по своей сути является  представительным множеством на  
вещественной оси (имеющим ненулевую меру), и оно, как правило, устойчиво к малым возмущениям и 
погрешностям вычислений. Для обработки интервальных данных 
фундаментальный характер имеет следующее определение (\cite{MetodikaBook}, \cite{Enclosing2022}):
\begin{definition}
	\textsl{Накрывающее измерение} (накрывающий замер) --- это интервальный результат 
	измерения, который гарантированно содержит истинное значение измеряемой величины. 
	Измерение, для которого нельзя утверждать, что оно содержат истинное значение 
	измеряемой величины, будем называть \textsl{ненакрывающим} (рис.~\ref{PCoverMeasurPic} 
	и \ref{ICoverMeasurPic}). \index{накрывающее измерение} 
	\index{ненакрывающее измерение} 
\end{definition}

Отметим, что с точки зрения формальной логики понятия накрывающего и ненакрывающего 
измерений являются противоположными, но при этом не противоречащими. 



%%%%%%%%%%%%%%%%%%%%%%%%%%%%%%%%%%%%%%%%%%%%%%%%%%%%%%%%%%%%%%%%%%%%%%%%%%%%%%%%%%%%%

\begin{figure}[!ht]
	\unitlength=1mm
	\centering\small 
	\begin{picture}(110,27) 
	\put(0,7){\includegraphics[width=60mm]{Intefig.eps}} 
	\put(62,7){\includegraphics[width=60mm]{Intefig.eps}} 
	\put(4,12.5){\vector(1,0){40}}
	\put(65,12.5){\vector(1,0){40}}
	  \linethickness{.5mm}
	\put(25,9){\color{red}\line(0,1){13}}
	\put(95,9){\color{red}\line(0,1){13}}
	\put(72,5){\mbox{\small\begin{tabular}{c}Интервал \\[-1pt] измерения\end{tabular}}}
	\put(11,5){\mbox{\small\begin{tabular}{c}Интервал \\[-1pt] измерения\end{tabular}}} 
	\put(42.4,14.5){\mbox{$\mbb{R}$}}  \put(104.4,14.5){\mbox{$\mbb{R}$}}        
	\put(14,24){\small\mbox{Истинное значение}}    
	\put(82,24){\small\mbox{Истинное значение}}    
	\end{picture}
	\caption{Накрывающее (слева) и ненакрывающее (справа)	измерения точечного истинного значения величины }
	\label{PCoverMeasurPic} 
\end{figure} 

%%%%%%%%%%%%%%%%%%%%%%%%%%%%%%%%%%%%%%%%%%%%%%%%%%%%%%%%%%%%%%%%%%%%%%%%%%%%%%%%%%%%%%  

Накрывающее измерение является гарантированной двусторонней вилкой значений 
измеряемой величины, тогда как для ненакрывающего измерения  подобное утверждать 
нельзя. 
При перенесении свойства накрытия истинного значения на выборки простейший путь --- объявить накрывающей выборкой совокупность накрывающих измерений, тогда как выборки, в которых присутствует хотя бы 
одно ненакрывающее измерение, станут ненакрывающими. 
Погрешности 
и выбросы (промахи) неотъемлемо присутствуют в данных, и проверка свойства <<накрытия истинного значения>> является нетривиальной. 

%%%%%%%%%%%%%%%%%%%%%%%%%%%%%%%%%%%%%%%%%%%%%%%%%%%%%%%%%%%%%%%%%%%%%%%%%%%%%%%%%%%%%

\begin{figure}[!ht]
	\unitlength=1mm
	\centering\small 
	\begin{picture}(110,27) 
	\put(0,7){\includegraphics[width=60mm]{InteInside.eps}} 
	\put(60,7){\includegraphics[width=60mm]{InteInsidePart.eps}} 
	\put(4,12.5){\vector(1,0){40}}
	\put(65,12.5){\vector(1,0){40}}
	\linethickness{.1mm}
	\put(29,15){\line(0,1){8}}
	\put(92,15){\line(0,1){8}}
	\put(78,5){\mbox{\small\begin{tabular}{c}Интервал \\[-1pt] измерения\end{tabular}}}
	\put(18,5){\mbox{\small\begin{tabular}{c}Интервал \\[-1pt] измерения\end{tabular}}} 
	\put(42.4,14.5){\mbox{$\mbb{R}$}}  \put(104.4,14.5){\mbox{$\mbb{R}$}}        
	\put(14,24){\small\mbox{Истинное значение}}    
	\put(72,24){\small\mbox{Истинное значение}}    
	\end{picture}
	\caption{Накрывающее (слева) и ненакрывающее (справа)  	измерения
	интервального истинного значения величины }
\label{ICoverMeasurPic} 
\end{figure} 
%%%%%%%%%%%%%%%%%%%%%%%%%%%%%%%%%%%%%%%%%%%%%%%%%%%%%%%%%%%%%%%%%%%%%%%%%%%%%%%%%%%%%
Далее мы будем называть \textit{накрывающей выборкой}\index{накрывающая выборка} 
совокупность измерений, в которой доминирующая часть (большинство и т.\,п.) 
измерений (наблюдений) являются накрывающими. \index{включающее  измерение} 
\index{охватывающее измерение}

Напротив, выборка называется \textsl{ненакрывающей},\index{ненакрывающая выборка} если преобладающая часть 
входящих в нее измерений ненакрывающие. 

Данное определение нестрогое и использует расплывчатые понятия <<большинство>>, 
<<доминирующая часть>> и т.\,п., которые должны уточняться каждый раз в процессе 
применения. 

Важность введенных понятий обусловлена тем, что накрывающее (охватывающее) измерение 
дает не только приближение к интересующему истинному значению физической 
величины, но и двустороннюю оценку этого значения, т.\,е. его гарантированные  
оценки снизу и сверху. Это обстоятельство позволяет привлечь для обработки накрывающих 
измерений более сильные средства, качественно другой математический аппарат (в частности, 
некоторые специфичные методы интервального анализа) и получить в результате уточненные 
оценки для истинного значения также в виде двусторонней оценки. Для ненакрывающих 
измерений и выборок это не всегда достижимо. 

Тот факт, что интервальный результат измерения не является накрывающим, может быть 
вызван различными причинами:  
измерение может оказаться выбросом, погрешность измерения недооценена, неадекватность выбранной 
модели объекта (непостоянство во времени, выбор неверной функциональной зависимости). 

Проверка того, является ли данное измерение или выборка накрывающими, находится  
вне рамок математической теории интервальных измерений, и решается в каждом случае конкретно. 
Для традиционных точечных измерений аналога введенных понятий не существует, так как 
все точечные измерения, как правило, ненакрывающие.  

Для достижения свойства накрытия нередко прибегают 
к специальным приемам в процессе предобработки данных 
(см. пп.~\ref{UncertAlterSect} и \ref{VaryUncertSect}). 




 \subsection{Информационное множество} 
\label{InfoSetSect} 

Данные измерений, которые были описаны ранее, можно называть 
\emph{первичными}, так как чаще всего они подвергаются дальнейшей обработке. Таким образом, 
для определения окончательного результата измерения необходимо дополнить наши конструкции 
моделью обработки данных (способом обработки данных). Это математическая 
модель, формализующая требования к результату обработки измерения и оформленная в виде 
системы уравнений,  задачи оптимизации и т.\,п., которая определяет то, 
что должно считаться результатом обработки измерений. 

\emph{Информационное множество} для интервальных данных  --- это множество значений параметров, 
удовлетворяющих математической системе отношений, полученной в результате агрегирования информации 
о  математической модели объекта, первичных данных измерений и модели их обработки. 
Информационное множество  зависит от выбранной 
модели обработки данных, и потому даже для одних и тех же данных может быть определено 
неединственным образом в зависимости от того, как эти данные обрабатываются и 
интерпретируются.  \index{информационное множество} 


\begin{example}{ Различные походы к задаче восстановления зависимости.}
	Предположим, что мы решаем задачу восстановления зависимости некоторого заданного 
	вида по данным измерений. Эта зависимость может восстанавливаться, например, 
	методом наименьших квадратов (МНК), методом наименьших модулей (МНМ) или с помощью чебышевского 
	(минимаксного) сглаживания. Перечисленные методы представляют собой разные модели обработки 
	данных. \index{метод наименьших квадратов}\index{метод наименьших модулей}  
	\index{чебышевское сглаживание} 
	
	Для одних и тех же данных измерений, т.\,е. первичных данных, итоговый результат 
	измерения будет разным в зависимости от того, какая именно методика их 
	обработки применяется --- МНК, МНМ	или минимаксное приближение. Следовательно, мы получим три различных 	информационных множества, которые в обычном случае неинтервальных данных, скорее 	всего, будут одноточечными множествами. (См. п.~\ref{PointRegressionEstimate}). 
\end{example}  
Неформально говоря, \emph{информационное множество --- это множество параметров задачи, 
которые совместны с данными измерений в рамках выбранной модели их обработки.} 
В гл.~\ref{MeasrConstChap} и \ref{FuncFitChap} приведены конкретные определения 
информационных множеств, возникающих в задаче оценивания постоянной величины и в задаче 
восстановления линейной зависимости. 

Аналогом информационного множества может отчасти служить понятие доверительного 
интервала  оцениваемой случайной величины в традиционной вероятностной статистике. 
В определение доверительного интервала входит дополнительный параметр --- \emph{уровень 
статистической значимости}, без которого понятие становится бессодержательным из-за 
неограниченности носителей большинства вероятностных распределений, но смысл 
доверительного интервала примерно соответствует информационному множеству. 

%%%%%%%%%%%%%%%%%%%%%%%%%%%%%%%%%%%%%%%%%%%%%%%%%%%%%%%%%%%%%%%%%%%%%%%%%%%%%%%%%%%

\section{Выбросы и промахи} 
\label{OutlierSect}

\textit{Выбросами} (или \textit{промахами}) в метрологии называются такие измерения, 
результаты которых не привносят информацию об исследуемом объекте в рамках его 
принятой модели. \index{выброс}\index{промах}

Другое популярное определение выбросов (промахов) состоит в том, что это результаты 
измерений, которые для данных условий резко отличаются от остальных результатов общей 
выборки. Выбросы нарушают некоторую однородность (согласованность, непротиворечивость), 
характерную для большинства наблюдений выборки по отношению к заданной математической 
модели (см. п.~\ref{RegrOutlSect}). 

Оба приведенных определения неформальны, так как, по-видимому, одно формальное 
определение для данного важнейного понятия дать нельзя.  

Как правило, выбросы стремятся удалить из выборки на этапе ее предварительной 
обработки (предобработки), т.е. перед применением \index{предобработка} формальных 
математических методов, так как присутствие выбросов существенно искажает оценки 
истинных значений параметров. Выявление выбросов является нетривиальной и, как правило, 
трудноформализуемой процедурой, которая опирается на опыт и т.\,п. Для вероятностной 
статистики выявление выбросов входит необходимой составной частью в обработку данных, 
а некоторые процедуры даже рекомендованы в стандартах \cite{GOSTDirect}. 

\emph{Что считать выбросом} в случае интервальных результатов измерений? 
Из того, что интервальное измерение не является 
накрывающим, не следует, что оно представляет выброс или промах. 
Отождествление выбросов (промахов) со свойством ненакрывания противоречит 
принципу соответствия, сформулированному в п.~\ref{InteStatistiSect}. При стремлении ширины интервальных измерений к нулю они переходят в точечные 
измерения, которые, как правило, всегда ненакрывающие. 

Если априори известно, что измерение, производимое данным инструментом с помощью
некоторой определенной методики должно быть накрывающим, то получение 
ненакрывающего результата является признаком выброса (промаха). 
 
Более подробное выбросы и промахи, а также методики их выявления будут 
подробнее  в гл.~3 и 4.



	\chapter{Измерение постоянной величины} 
\label{MeasrConstChap}
%\input{Chapter3old.txt}
\input{Chapter3.txt}
%%%%%%%%%%%%%%%%%%%%%%%%%%%%%%%%%%%%%%%%%%%%%%%%%%%%%%%%%%%%%%%%%%%%%%%%%%%%%%%%%%%%%%%%  


	\chapter{Задача восстановления зависимостей} 
\label{FuncFitChap}
\input{Chapter4.txt}




\newpage
\renewcommand{\bibname}{\centering \normalsize БИБЛИОГРАФИЧЕСКИЙ СПИСОК}  
% for the report or book class 
\input{Biblio.txt}
   
	
%	\addcontentsline{toc}{chapter}{Предметный указатель}
\raggedright\small\printindex   

% 1 стр пустая
%\blankpage
\thispagestyle{empty}

\begin{center}
	\hfill \break
	Министерство науки и высшего образования  Российской Федерации\\
	%	\hfill \break
	$\ov{~~~~~~~~~~~~~~~~~~~~~~~}$\\
	\normalsize{	САНКТ-ПЕТЕРБУРГСКИЙ \\
		ПОЛИТЕХНИЧЕСКИЙ УНИВЕРСИТЕТ ПЕТРА ВЕЛИКОГО}\\ 
	$\ov{~~~~~~~~~~~~~~~~~~~~~~~~~~~~~~~~~~~~~~~~~~~~~~~~~~~~~~~~~~~~~~~~~~~~~~~~~~~~~~~~~~~~~~~~~~~~~~}$\\	
	Физико-механический институт\\
	Высшая школа прикладной математики и вычислительной физики\\
	%\hfill \break
	%\hfill \break
	%	\large{Институт прикладной математики и механики}\\
	%\hfill \break		
	%\hfill \break
	%	\large{Кафедра «Прикладная математика»}\\
	%\hfill \break
	%\hfill \break
	\hfill \break
	
	
	
	
	
	\Large{\it А.\,Н.\,Баженов\\
		\hfill \break		\hfill \break		}
	{\Large	ВВЕДЕНИЕ В АНАЛИЗ ДАННЫХ\\
		С ИНТЕРВАЛЬНОЙ НЕОПРЕДЕЛЕННОСТЬЮ}\\
	\hfill \break 	\hfill \break	
	\Large{	Учебное пособие	
	}\\
\end{center}

\hfill \break		\hfill \break	
\begin{figure}[h]
	\centering
	\includegraphics[width=60mm]{PolytechPressRu.png}
	%	\label{f:cover}	
\end{figure}
%\hfill \break
%\hfill \break
\begin{center}\Large{Санкт-Петербург \\
		\hfill \break
		2022} \end{center}
\thispagestyle{empty} % выключаем отображение номера для этой страницы


\end{document}

	%%%%%%%%%%%%%%%%%%%%%%%%%%%%%%%%%%%%%%%%%%%%%%%%%%%%%%%%%%%%%%%%%%%  

	\begin{table}[h!]
	\centering
	\caption{Стабильные изотопы ртути.} 
	\medskip 
	\begin{tabular}{cc}
		Изотоп & Распространенность \\
		\hline
		$^{196}$Hg &  0,155 \% \\
		$^{198}$Hg & 10,04 \% \\
		$^{199}$Hg  & 16,94 \% \\
		$^{200}$Hg  & 23,14 \% \\
		$^{201}$Hg  &  13,17 \% \\
		$^{202}$Hg  &  29,74 \% \\
		$^{204}$Hg  &  6,82 \% \\
		\hline
	\end{tabular} 
	\label{Sulfur}
\end{table}		

%	Приведенные в таблице \ref{Sulfur} величины распространенности служат исходными данными для построения гистограммы частот. %, схематично представленной на Рис.~\ref{f:HistAtom}.
Конкретно для атомов ртути этот рисунок показан на Рис.~\ref{f:HistHg}.

	%%%%%%%%%%%%%%%%%%%%%%%%%%%%%%%%%%%%%%%%%%%%%%%%%%%%%%%%%%%%%%%%%%%  


\begin{example}{Полное сопротивление резонансного контура.}
	Полное сопротивление $Z$ цепи переменного тока складывается из активной и реактивной 
	составляющих, общая формула 
	\begin{equation}
	Z = \sqrt{ R^2 + (X_L - X_C)^2 }, \label{Zseq}
	\end{equation}
	где  $R$ --- активное сопротивление, $X_L = \omega\cdot L$ --- индуктивное 
	сопротивление, $X_C = 1/\omega C$ --- емкостное сопротивление (см., к примеру, 
	\cite{YavorskiDetlaf}). Как будет изменяться ток в цепи переменного тока с заданными 
	$R$, $L$, $C$, когда круговая частота тока меняется в интервале $[\,\omega_1, \omega_2 ]$? 
	
	Для ответа на вопрос можно воспользоваться законом Ома для цепи переменного тока
	\begin{equation*}
	I = U/Z, 
	\end{equation*} 
	и здесь нужно найти, прежде всего, интервал изменения $Z$. Выражение для полного 
	сопротивления $Z$ характерно тем, что индуктивное и емкостное сопротивления зависят 
	от частоты противоположным образом. 
	
	В качестве середины интервала частот возьмем $f= 13.56$ МГц. Диапазон вокруг этой частоты 
	может использоваться во всем мире без лицензий, он имеет специальное обозначение 
	--- ISM (Industrial, Scientific, Medical). 
	
	Рассмотрим практическое применение оборудования в диапазоне ISM: генератор 
	плазменного источника возбуждения-ионизации пробы для элементного анализа 
	с индуктивно-связанной плазмой (ИСП). 
	
	Индуктор плазмотрона ИСП (диаметр и длина 20 мм, три витка) имеет индуктивность 
	порядка $L=100$ нГ и реактивное сопротивление 8.52 Ом на частоте ISM. В таком случае 
	емкость резонансного контура имеет такое же реактивное сопротивление при $C=1.38$ нФ.  
	
	Пусть круговая частота $ \omega$ меняется в интервале $\ \pm 10 \%$ относительно 
	центра  диапазона ISM. 
	\begin{equation}
	\label{OmegaInt}
	\mbf{\omega} = [\,\omega_1, \omega_2 ] = [7.668, 9.372] \cdot 10^7 \ \text{Гц}.
	\end{equation}
	Напомним, что круговая частота связана с частотой колебаний $f$ соотношением 
	$\omega = 2 \pi f$. Активное сопротивление примем равным 1 Ом. Такой порядок имеет 
	сопротивление частично ионизированного аргона в плазмотроне. 
	
	Построим зависимости $X_L, X_C$ и $Z$ от частоты в интервале 
	$\mbf{\omega}$ \eqref{OmegaInt}. 
	
	%%%%%%%%%%%%%%%%%%%%%%%%%%%%%%%%%%%%%%%%%%%%%%%%%%%%%%%%%%%%%%%%%%%  
	
	\begin{figure}[ht]
		\centering\small  
		\setlength{\unitlength}{1mm} 
		\begin{picture}(80,55) 
		%\put(6,0){\includegraphics[width=75mm]{ZvalueNoLabel.png}} 
		\put(6,0){\includegraphics[width=75mm]{ZvalueNoLabelText.png}} 
		\put(40,-1){$\omega, 10^{7}$ Гц}
		\put(-10,25){$X_L, X_C, Z,$ Ом}
		\end{picture} 
		\caption{Зависимости $X_L, X_C$ и $Z$ от частоты.} 
		\label{Xresonance}  
	\end{figure} 
	
	
	%%%%%%%%%%%%%%%%%%%%%%%%%%%%%%%%%%%%%%%%%%%%%%%%%%%%%%%%%%%%%%%%%%%
	
	Определим границы изменения величины $Z$. В точке резонанса  $X_L = X_C$ и суммарное 
	сопротивление $LC$-части схемы равно нулю, а полное сопротивление равно активному, 
	т.\,е. $Z=R$. Таким образом, 
	\begin{equation*} 
	\min_{\omega} Z = R =1. 
	\end{equation*} 
	
	Величины индуктивного $X_L$ и емкостного $X_C$ сопротивлений зависят от частоты 
	противоположным образом и монотонны. Поэтому максимальные значения величины 
	$X_{LC} = | X_L - X_C | $ достигаются на одном из краев диапазона $\mbf{\omega}$: 
	\begin{equation*} 
	\max_{\omega} X_{LC} = \max \left\lbrace X_{LC} ( \un {\mbf{\omega}}), X_{LC} ( \ov {\mbf{\omega}}) 
	\right\rbrace =  \max \left\lbrace 1.6401, 1.7822 \right\rbrace = 1.7822. 
	\end{equation*} 
	Соответственно, получаем значение
	\begin{equation*} 
	\max Z = \sqrt{R^2 + \max_{\omega} X^2_{LC}} =2.0436.
	\end{equation*} 
	%		{\color{red}$\Delta$ - заменить} 
	
	Окончательно имеем точный интервал величины полного сопротивления в интервале 
	$\mbf{\omega}$ 
	\begin{equation*} 
	\mbf{Z} = [1, 2.0436].
	\end{equation*} 
	
	
	Вычислим также интервал изменения полного сопротивления $\mbf{Z}$ по формуле \eqref{Zseq}, 
	представленной двумя способами: 
	\begin{align}
	\mbf{Z}_1 &= \sqrt{R^2 + \left( \mbf{\omega} L - \frac{1}{\mbf{\omega} C} \right) ^2}, 
	\label{Z1} \\[3mm] 
	\mbf{Z}_2 &= \sqrt{R^2 + \mbf{\omega}^2\left( L - \frac{1}{\mbf{\omega}^2 C} \right)^2}. 
	\label{Z2}
	\end{align} 
	Вычисления дают величины сопротивлений соответственно
	\begin{align*}
	\mbf{Z}_1 &= [1, 2.0436] \ \text{Ом}, \\[1mm]
	\mbf{Z}_2 &= [1, 2.3968] \ \text{Ом}.
	\end{align*} 
	Нижние величины $Z_1, \ Z_2$ соответствуют частотному резонансу $X_L=X_C$ в середине 
	диапазона, верхние относятся к краям выбранного диапазона, иначе --- расстройке резонанса. 
	
	В выражении \eqref{Z2} радиус выражения больше, чем в \eqref{Z1} в связи с дополнительным умножением на величину $\mbf{\omega}^2$.
	
	%Для последовательного колебательного контура в $RLC$-цепях, в котором все три элемента включены последовательно добротность 
	%\begin{equation}
	%Q = \frac{1}{R} \sqrt{\frac{L}{C}}
	%\end{equation}
\end{example}


\begin{example}{Твины для описания составных ошибок измерений.}
	Рассмотрим измерение так называемых осцилляций нейтрино, результаты измерений которых 
	удобно представить в виде твинов. 
	
	При измерении осцилляций нейтрино в атмосфере Земли экспериментаторы традиционно 
	используют безразмерную величину $R$, характеризующую отношение числа разных сортов 
	нейтрино. Подборка значений $R$ из разных экспериментов приведена на стр.~872 
	в публикации \cite{UFN1997}. Приведем часть данных из этой публикации: 
	\begin{align}
	R_1 &=  0.60^{+0.07}_{-0.06} \pm 0.05, && \text{<<Kamiokande>>}, \label{Kamiokande}\\[4pt]  
	R_2 &=  0.54 \pm 0.05 \pm 0.12, &&   \text{<<IMB>>}. \label{IMB}
	\end{align} 
	Результаты измерений даны  в форме <<базовое значение, статистическая погрешность, 
	систематическая погрешность>>, а после числовых данных приводится название проекта, 
	в ходе которого они были получены.                                
	В первом примере (<<Kamiokande>>) статистическая погрешность дана в виде границ, 
	несимметричных относительно среднего значения. Такая ситуация возникает при оценке 
	значения величины, входящей в нелинейную функцию. Она соответствует 
	общему случаю модели погрешности \eqref{GeneralErrorModel}, рассмотренной выше, 
	в разделе \ref{MeasuResultSect}. 
	
	В виде твинов данные \eqref{Kamiokande} и \eqref{IMB} можно представить следующим 
	образом. В качеcтве первого шага выразим результаты в виде обычных интервалов $\mbf{r}_1$ 
	и $\mbf{r}_2$, с учетом только статистических погрешностей: 
	\begin{align}
	\mbf{r}_1  &= [\; 0.6-0.06, \  0.6+0.07 \;] \  = \  [\; 0.54, 0.67 \;], \label{r1} \\[3pt]  
	\mbf{r}_2  &= [\; 0.54-0.05,  0.54+0.05 \;] = [\; 0.49, 0.59 \;]. \label{r2} 
	\end{align}	
	При этом $\w{\mbf{r}_1} = 0.13 > \w{\mbf{r}_2} = 0.1$ ввиду более широких статистических 
	оценок для величины $R_1$.	
	
	Далее, произведем учет систематической погрешности, произведя <<интервализацию>> концов 
	интервалов $\mbf{r}_1$ и  $\mbf{r}_2,$ вычитая и добавляя величины систематических 
	погрешностей к величинам $\un{\mbf{r}}_1$, $\ov{\mbf{r}}_1$, $\un{\mbf{r}}_2$, 
	$\ov{\mbf{r}}_2$. 
	
	Обозначим получившиеся твины как $\mbf{R}_1$ и $\mbf{R}_2$:  
	\begin{align}
	\mbf{R}_1  
	&= \bigl[\;[ 0.54-0.05,  0.54+0.05], \  [0.67-0.05,  0.67+0.05]\;\bigr] \notag\\[3pt] 
	&= \bigl[\;[ 0.49,  0.59],  \ [0.62,  0.72 ]\;\bigr], \label{R1}     \\[3mm] 
	\mbf{R}_2  
	&= \bigl[\;[ 0.49-0.12,  0.49+0.12], \  [0.59-0.12,  0.59+0.12]\;\bigr] \notag\\[3pt] 
	&= \bigl[\;[ 0.37,  0.61], \  [0.47,  0.71 ]\;\bigr]. \label{R2} 
	\end{align}
	
	На рисунке \ref{TwinsRnu2} графически представлены твины $\mbf{R}_1$ и $\mbf{R}_2$. 
	Численные значения концов правого интервала смещены вверх. На сей раз твин $\mbf{R}_1$  
	<<\'{у}же>>, чем  твин $\mbf{R}_2$, ввиду более широких систематических погрешностей 
	для величины $R_2$. Следует заметить также, интервалы  $\un{\mbf{R}}_2$, $\ov{\mbf{R}}_2$  
	в форме \eqref{Twin} имеют ненулевое пересечение. Это пересечение дано более темной 
	заливкой, чем концы твина. 
	
	\begin{figure}[hbt]
		\centering\small 
		\setlength{\unitlength}{1mm}
		\begin{picture}(70,17)
		\put(0,0){\includegraphics[width=70mm]{TwinRnu}}
		\put(-5,6.6){\vector(1,0){80}} \put(71.5,7.6){$\mbb{R}$} 
		\put(18,10){{\footnotesize 0.49}} 
		\put(30,10){{\footnotesize 0.59}} 
		\put(37,12){{\footnotesize 0.62}} 
		\put(47,12){{\footnotesize 0.72}}  
		\put(35,1){$\mbf{R}_1$}  
		\end{picture}
		%\caption{Данные по массе нейтрино как твин.} 
		%	\label{TwinsRnu} 
	\end{figure}
	%	\vspace{-10mm}
	\begin{figure}[hbt]
		\centering\small 
		\setlength{\unitlength}{1mm}
		\begin{picture}(70,17)
		\put(0,0){\includegraphics[width=70mm]{TwinR2nu}}
		\put(-5,6.6){\vector(1,0){80}} \put(71.5,7.6){$\mbb{R}$} 
		\put(3,10){{\footnotesize 0.37}} 
		\put(16,12){{\footnotesize 0.47}} 
		\put(33,10){{\footnotesize 0.61}} 
		\put(47,12){{\footnotesize 0.71}} 
		\put(30,1){$\mbf{R}_2$}  
		\end{picture}
		\caption{Данные по физике нейтрино в форме твинов типа <<$\leq$>>.} 		
		\label{TwinsRnu2} 
	\end{figure} 
	
	%%%%%%%%%%%%%%%%%%%%%%%%%%%%%%%%%%%%%%%%%%%%%%%%%%%%%%%%%%%%%%%%%%%%%%%%%%%%%%%%%%%%
	
	Далее, можно проводить различный содержательный анализ величин $\mbf{R}_1$  и $\mbf{R}_2$. 
\end{example}  

	%%%%%%%%%%%%%%%%%%%%%%%%%%%%%%%%%%%%%%%%%%%%%%%%%%%%%%%%%%%%%%%%%%%%%%%%%%%%%%%%%%%%
Эти данные приводятся в Табл.~\ref{TableDataCover2}.

\begin{table}[h!]
	\begin{center}
		\begin{tabular}{| c | c c | }
			\hline
			Номер замера & Peak &   {\tt std}  Peak \\ % &  BG &   {\tt std} BG \\
			\hline
			1 &	-4.4 & 2.7 \\ % & 4.2 & 6.7 \\
			2 & -3.4 & 1.9 \\ % & -3.2 &	4.8 \\
			3 & -6.9 & 2.4 \\ % & 12.1 &	9 \\
			%				4 &	-1.2 & 2.4 \\ % & 12.4 &	7.2 \\
			%				5 &	-1.0 & 2.7 \\ % & 9.4 & 5.1 \\
			%				6 &	-10.8 &	3.5 \\ % &1	& 12.4 \\
			%				7 &	-10.2 &	2.8 \\ % &-0.6 &	6.1 \\
			8 &	-6.3 &	2 \\ % &	3.9 &	4.3\\
			%				9 &	-10.4 &	4.1 \\ % &	10.3 &	10\\
			%				10 & 0.6& 3.4 \\ % & -4.8 & 10.6\\
			%				11 &-1.8 &	2 \\ % &	4.6&	4.2\\
			12 &-6.6 & 2.1	\\ % &-5.7&4.6\\
			13 &-4.9 &2.1 \\ % &	13 &3 \\
			14 &-6.0 &	2.4 \\ % & 8.4	&4.6\\
			15 &-4.0 & 2.7 \\ %	& 10.6 &5.5\\
			\hline	
		\end{tabular}
	\end{center}
	\caption{Накрывающая подвыборка данных таблицы 1.}
	\label{TableDataCover2}
\end{table}
	%%%%%%%%%%%%%%%%%%%%%%%%%%%%%%%%%%%%%%%%%%%%%%%%%%%%%%%%%%%%%%%%%%%%%%%%%%%%%%%%%%%%
Рассмотрим теперь, что было бы при более низкой точности цифрового измерителя, чем в только что рассмотренном примере.  Пусть она равна 9-ти двоичным разрядам. 
На Рис.~\ref{DRS4ZeroLine100cell2} представлены 
интервальные результаты измерения нуля %цифрового измерителя напряжения 
%данные для 100 измерений базовой линии 
с теми же  данными $\left\lbrace \mathring{x}_k\right\rbrace _{k=1}^{100}$. 
\begin{figure}[htb]
	\centering\small 
	\unitlength=1mm
	\begin{picture}(100,58)
	\put(-10,53){\mbox{\small Данные}} 
	\put(-10,50){\mbox{\small измерений, В}}	
	\put(90,50){\mbox{\small $ \max_{1\leq k\leq n} \ov{\mbf{x}}_{k}$}} 
	\put(90,30){\mbox{\small $x_\text{c}  $}}	
	\put(10,0){\includegraphics[width=0.7\textwidth]{ZeroLineCh=1cell=1resolution=9.png}}
	\put(90,11){\mbox{\small $ \min_{1\leq k\leq n} \un{\mbf{x}}_{k}$}} 
	\put(87,5){\mbox{\small Номер}} 
	\put(87,2){\mbox{\small измерения}} 
	\end{picture}
	\caption{Диаграмма рассеяния интервальных   измерений}
	неопределенности нуля.
	Разрядность измерителя $\mathit{N \!O\!B} = 9$.
	\label{DRS4ZeroLine100cell2} 
\end{figure}  

В  случае более грубых измерений, характер совместности отдельных измерений существенно 
изменился. Для многих пар замеров $(i, j)$ имеет место непустота пересечения интервалов:  $ \mbf{x}_i \cap \mbf{x}_j \neq\varnothing$. 

%%%%%%%%%%%%%%%%%%%%%%%%%%%%%%%%%%%%%%%%%%%%%%%%%%%%%%%%%%%%%%%%%%%%%%%%%%%%%%%%%%%%%%%%  

\begin{figure}[htb]
	\centering\small 
	\unitlength=1mm
	\begin{picture}(100,58)
	\put(-10,53){\mbox{\small Число}} 
	\put(-10,50){\mbox{\small измерений}}
	\put(-10,47){\mbox{\small в столбце}}
	\put(-10,44){\mbox{\small гистограммы}}
	\put(10,0){\includegraphics[width=0.7\textwidth]{HISTZeroLineResolution=9.png}}
	\put(87,5){\mbox{\small Диапазон}} 
	\put(87,2){\mbox{\small измерений, В}} 
	\end{picture}
	\caption{Гистограмма данных $\left\lbrace \mathring{x}_k\right\rbrace _{k=1}^{100}$ 
		интервальных   измерений } неопределенности нуля.
	Разрядность измерителя $\mathit{N\!O\!B} = 9$.
	\label{HISTZeroLine2} 
\end{figure}  

%%%%%%%%%%%%%%%%%%%%%%%%%%%%%%%%%%%%%%%%%%%%%%%%%%%%%%%%%%%%%%%%%%%%%%%%%%%%%%%%%%%%%%%%%

Характер изменения статуса пересечения данных ярко демонстрируется гистограммой, 
представленной на Рис.~\ref{HISTZeroLine2}. Ширина столбца соответствует точности 
измерения. Распределение результатов измерения неопределенности нуля цифрового измерителя напряжения
%амплитуд смещений базовой линии 
весьма близко к равномерному. 
% неопределенность измерения нуля цифрового измерителя напряжения
Согласно выражению \eqref{UNInterval} имеем оценку информационного множества неопределенности нуля
\begin{equation*} 
%\label{UNInterval} 
\mbf{J}\; = \;\bigvee_{1\leq k\leq n} \mbf{x}_{k} \ 
= \  \Bigl[\,\min_{1\leq k\leq n} \un{\mbf{x}}_{k}, 
\max_{1\leq k\leq n} \ov{\mbf{x}}_{k}\,\Bigr] = \left[ -1.16 \cdot 10^{-3}, 8.79 \cdot 10^{-3} \right].
\end{equation*} 
Точечная оценка измеряемой величины \eqref{midUNInterval} не изменилась:
\begin{equation*}
x_\text{c} \  = \  \m\mbf{J} \   
= \  \tfrac{1}{2} \Bigl(\,\min_{1\leq k\leq n} \un{\mbf{x}}_{k} + 
\max_{1\leq k\leq n} \ov{\mbf{x}}_{k}\,\Bigr) = 3.82 \cdot 10^{-3}. 
\end{equation*} 

Результат более грубыx измерений выглядит, на первый взгляд более надежным, 
чем более точных. Гистограмма на Рис.~\ref{HISTZeroLine2} демонстрирует гораздо 
более высокую степень однородность представителей интервальной выборки Рис.~\ref{DRS4ZeroLine100cell2}. 

В случае решения задач восстановления зависимостей по интервальным  данным, см. \S\ref{FuncFitChap}, подобное  
явление носит название парадокса Е.З.\,Демиденко, см. \S\ref{DemidParadoxSect}, 
суть которого может быть кратко выражена фразами <<Чем грубее --- тем лучше>> или 
<<Чем точнее --- тем хуже>>.  \index{парадокс Демиденко} 

Это, тем не менее, неточно. Оценка информационного множества  $\mbf{J}$
заметно расширилась для более грубых измерений. В определенность смысле можно сказать, что огрубленная дискретность измерений поглотила в существенной степени  неучтенную систематическую 
погрешность, сделала ее менее заметной. При этом сами результаты не стали более качественными. 

Подводя итог, можно заметить, что более грубое измерение или, что эквивалентно, 
отбрасывание некоторого количества младших разрядов, не улучшает интерпретацию выборки 
интервальных данных с большим числом пустых множеств пересечения интервалов замеров. 
При этой огрублении утрачивается часть информации, объективно содержащейся в выборке $\left\lbrace \mathring{x}_k\right\rbrace _{k=1}^{100}$. 
Процедура варьирования неопределенности  по методике \S\ref{UncertAlterSect} для достижения совместности данных выглядит более содержательной, поскольку дает как достижение свойства накрытия модифицированной выборки, так и количественную оценку увеличения величины радиусов отдельных замеров.
	
	%%%%%%%%%%%%%%%%%%%%%%%%%%%%%%%%%%%%%%%%%%%%%%%%%%%%%%%%%%%%%%%%%%%%%%%%%%%%%%%%%%%%
	
		%%%%%%%%%%%%%%%%%%%%%%%%%%%%%%%%%%%%%%%%%%%%%%%%%%%%%%%%%%%%%%%%%%%%%%%%%%%%%%%%%%%%%%%
	\begin{figure}[htb]
		\centering\small 
		\unitlength=1mm
		%	\begin{picture}(90,67)
		%	\put(0,0){\includegraphics[width=90mm]{PgammaPhMean.png}}
		\includegraphics[width=0.7\textwidth]{PgammaPhMean.png}
		%	\put(85,2){\mbox{\small номер измерения}} 
		%	\end{picture}
		\caption{Диаграмма рассеяния интервальных измерений  и точечная оценка \eqref{xc}.} 
		\label{IMeanNonCover} 
	\end{figure} 
	%%%%%%%%%%%%%%%%%%%%%%%%%%%%%%%%%%%%%%%%%%%%%%%%%%%%%%%%%%%%%%%%%%%%%%%%%%%%%%%%%%%%%%%
	
	
		
	\begin{table}[h!tb]
		\begin{center}
			\begin{tabular}{| c | c c | }
				\hline
				Номер замера & Peak &   {\tt std}  Peak \\ % &  BG &   {\tt std} BG \\
				\hline
				1 &	-4.4 & 2.7 \\ % & 4.2 & 6.7 \\
				2 & -3.4 & 1.9 \\ % & -3.2 &	4.8 \\
				3 & -6.9 & 2.4 \\ % & 12.1 &	9 \\
				4 &	-1.2 & 2.4 \\ % & 12.4 &	7.2 \\
				5 &	-1.0 & 2.7 \\ % & 9.4 & 5.1 \\
				6 &	-10.8 &	3.5 \\ % &1	& 12.4 \\
				7 &	-10.2 &	2.8 \\ % &-0.6 &	6.1 \\
				8 &	-6.3 &	2 \\ % &	3.9 &	4.3\\
				9 &	-10.4 &	4.1 \\ % &	10.3 &	10\\
				10 & 0.6& 3.4 \\ % & -4.8 & 10.6\\
				11 &-1.8 &	2 \\ % &	4.6&	4.2\\
				12 &-6.6 & 2.1	\\ % &-5.7&4.6\\
				13 &-4.9 &2.1 \\ % &	13 &3 \\
				14 &-6.0 &	2.4 \\ % & 8.4	&4.6\\
				15 &-4.0 & 2.7 \\ %	& 10.6 &5.5\\
				\hline	
			\end{tabular}
		\end{center}
		\caption{Данные таблицы 1  для величины $\delta \times 10^{5}$ \cite{Pgamma1992}.}
		\label{TableDataV}
	\end{table}

	\begin{figure}[h!] 
	\centering\small 
	\unitlength=1mm
	\begin{picture}(100,48)
	\put(5,39){\mbox{\small $w$}} 
	\put(-5,42){\mbox{\small Значения}}
	\put(10,0){\includegraphics[width=70mm]{weightL1.png}}
	\put(87,5){\mbox{\small номер}} 
	\put(87,2){\mbox{\small измерения}} 
	\end{picture}
	\caption{Значения весов в задаче оптимизации \eqref{L1opt}.}
	\label{weightL1fig}
\end{figure}

	\begin{table}[h!]
	\begin{center}
		\begin{tabular}{|c|c|}
			\hline
			Номер измерения & Данные энкодера\\	
			\hline
			1 & 388 \\
			2 & 737 \\
			3 & 951 \\
			4 & 1354 \\
			5 & 1756 \\
			6 & 1970 \\
			7 & 2399 \\
			8 & 2801 \\
			9&  3204 \\
			10& 3606 \\
			\hline
		\end{tabular}
		\caption{Подвыборка из данных на Рис.~\ref{EncoderStepData}.  }
		\label{TableDataEncoderpart2}
	\end{center}
\end{table}

\begin{equation} 
\label{InLinEqSys} 
\arraycolsep=2pt 
\left\{ \ 
\begin{array}{ccccccccccc}
\beta_0 &+& \mbf{x}_{11}\beta_1 &+& 
\mbf{x}_{12} \beta_2 &+& \ldots &+& \mbf{x}_{1m}\beta_m &=& \mbf{y}_{1}, \\[3pt] 
\beta_0 &+& \mbf{x}_{21}\beta_1 &+& 
\mbf{x}_{22} \beta_2 &+& \ldots &+& \mbf{x}_{2m}\beta_m &=& \mbf{y}_{2}, \\[3pt] 
\vdots &&  \vdots && \vdots && \ddots && \vdots && \vdots                  \\[3pt]  
\beta_0 &+& \mbf{x}_{n1}\beta_1 &+& 
\mbf{x}_{n2} \beta_2 &+& \ldots &+& \mbf{x}_{nm}\beta_m &=& \mbf{y}_{n}. 
\end{array} 
\right. 
\end{equation} 