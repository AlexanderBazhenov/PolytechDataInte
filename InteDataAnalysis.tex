\documentclass[a5paper,openany]{book}

\usepackage{cmap}  
\usepackage[utf8]{inputenc}
\usepackage[T2A]{fontenc} 
\usepackage{index} 
\usepackage[russian]{babel} 
\usepackage{amsmath,amssymb} 
\usepackage{euscript,upref}  
\usepackage{array,longtable}
\usepackage{indentfirst} 
\usepackage{graphicx} 
%\usepackage{caption} 
\usepackage[justification=centering]{caption}
\usepackage{calrsfs} 
\usepackage{url}
\usepackage{multirow,makecell,array}
%\usepackage{setspace} 
\usepackage{todonotes}
%\usepackage{calligra}
%\usepackage{makeidx}
%%%%%%%%%%%%%%%%%%%%%%%%%%%%%%%%%%%%%%%%%%%%%%%%%%%%%%%%%%%%%%%%%%%%%%%%%%%%%%%%%%%%%%%%
%
%           Определения новых команд и макросов
%    
%\DeclareMathAlphabet{\mathcalligra}{T1}{calligra}{m}{n}
%\DeclareFontShape{T1}{calligra}{m}{n}{<->s*[1.8]callig15}{}
\newcommand{\mbf}[1]{\protect\text{\boldmath$#1$}}
\newcommand{\mbb}{\mathbb}
\newcommand{\mrm}{\mathrm}
\newcommand{\mcl}{\mathcal}
\newcommand{\msf}{\mathsf}
\newcommand{\eus}{\EuScript}
\newcommand{\ov}{\overline}
\newcommand{\un}{\underline}
\newcommand{\m}{\mathrm{mid}\;}
\newcommand{\w}{\mathrm{wid}\;}
\newcommand{\Uni}{\mathrm{Uni}\,}
\newcommand{\Tol}{\mathrm{Tol}\,} 
\newcommand{\Uss}{\mathrm{Uss}\,} 
\newcommand{\Ab}{(\mbf{A}, \mbf{b})}
\newcommand{\Arg}{\mathrm{Arg}\;} 
\newcommand{\sgn}{\mathrm{sgn}\;} 
\newcommand{\ran}{\mathrm{ran}\,} 
\newcommand{\pro}{\mathrm{pro}\,} 
\newcommand{\dom}{\mathrm{dom}\,} 
\newcommand{\IVE}{\mathrm{IVE}\,} 
\newcommand{\IED}{\mathrm{IED}\,} 
\newcommand{\calX}{\mathrsfs{X}} 
\newcommand{\cond}{\mathrm{cond}} 
\newcommand{\mode}{\mathrm{mode}\,} 
\newcommand{\dual}{\mathrm{dual}\,} 
\newcommand{\dist}{\mathrm{dist}\,} 
\newcommand{\Dist}{\mathrm{Dist}\,} 
\newcommand{\const}{\mathrm{const}} 
\newcommand{\USS}{\varXi_{\hspace{-0.5pt}uni}} 
\newcommand{\TSS}{\varXi_{\hspace{-0.5pt}tol}} 
\newcommand{\NExt}{_{\scalebox{0.57}{$\natural$}}}
\newcommand{\ih}{\scalebox{0.67}[0.87]{$\Box$\hspace*{1pt}}}

\renewcommand{\r}{\mathrm{rad}\;} 
\renewcommand{\vert}{\mathrm{vert}\,} 
\newcommand{\md}{\operatorname{med}}

%%%%%%%%%%%%%%%%%%%%%%%%%%%%%%%%%%%%%%%%%%%%%%%%%%%%%%%%%%%%%%%%%%%%%%%%%%%%%%%%%%%%%%%

\textwidth=114truemm
\textheight=165truemm
\oddsidemargin=-1cm
\evensidemargin=\oddsidemargin
\topmargin=-1cm
\sloppy

\pagestyle{plain}
%\mathsurround=1pt
%\tolerance=400
%\hfuzz=2pt
\makeindex

\captionsetup{font=small,labelsep=period,margin=7mm} 
%%% ToC (table of contents) APPEARANCE
\usepackage[nottoc,notlof,notlot]{tocbibind} % Put the bibliography in the ToC
\usepackage{tocloft} % Alter the style of the Table of Contents
\renewcommand{\cftsecfont}{\rmfamily\mdseries\upshape}
\renewcommand{\cftsecpagefont}{\rmfamily\mdseries\upshape} % No bold!

\newtheorem{definition}{Определение}[section] 


%\renewcommand{\bibname}{References} 
%\addto{\captionsenglish}{\renewcommand{\bibname}{References}}
%\renewcommand\bibname{Библиографический список}

% переименовываем  список литературы в "список используемой литературы"
%\addto\captionsrussian{\def\refname{Список используемой литературы}}

%  в Литературе ссылки БЕЗ квадратных скобок
\makeatletter
\renewcommand\@biblabel[1]{#1.}
\makeatother
%%% Создание списка собственной переменной окружения

\newcommand{\listOfExamples}{Список примеров}
\newlistof{example}{exp}{\listOfExamples} 
\newcounter{Examp}
\setcounter{Examp}{0}
\renewcommand\theExamp{\thesection.\arabic{Examp}} %Эта команда - макрос для счетчика Examp, чтобы каждый раз не писать длинное выражение 
%Если нужна сквозная нумерация примеров:
%\renewcommand\theExamp{\arabic{Examp}}

\newenvironment{example}[1]
{	\par\vspace{\baselineskip}
	\refstepcounter{Examp}	
	{\noindent\textbf{Пример~\theExamp~(#1)} 			
		\protect\addcontentsline{exp}{example}{\protect\numberline{\theExamp}\hspace{10pt}~#1}}
	%{\protect\numberline{\listOfExamples}}	
	{\noindent\ignorespaces}
}
%{\hfill$\blacksquare$\par\addvspace{3ex}}
{\hfill\par\addvspace{3ex}}
%

\newcounter{DefNum}
\setcounter{DefNum}{1}






\begin{document}
%\maketitle

\begin{center}
	\hfill \break
Министерство науки и высшего образования  Российской Федерации\\
%	\hfill \break
$\ov{~~~~~~~~~~~~~~~~~~~~~~~}$\\
	\normalsize{	САНКТ-ПЕТЕРБУРГСКИЙ \\
		ПОЛИТЕХНИЧЕСКИЙ УНИВЕРСИТЕТ ПЕТРА ВЕЛИКОГО}\\ 
$\ov{~~~~~~~~~~~~~~~~~~~~~~~~~~~~~~~~~~~~~~~~~~~~~~~~~~~~~~~~~~~~~~~~~~~~~~~~~~~~~~~~~~~~~~~~~~~~~~}$\\	
	Физико-механический институт\\
	Высшая школа прикладной математики и вычислительной физики\\
	%\hfill \break
	%\hfill \break
	%	\large{Институт прикладной математики и механики}\\
	%\hfill \break		
	%\hfill \break
	%	\large{Кафедра «Прикладная математика»}\\
	%\hfill \break
	%\hfill \break
	\hfill \break
	
	
	
	
	
	\Large{\it А.\,Н.\,Баженов\\
		\hfill \break		\hfill \break		}
	{\Large	ВВЕДЕНИЕ В АНАЛИЗ ДАННЫХ\\
		С ИНТЕРВАЛЬНОЙ НЕОПРЕДЕЛЕННОСТЬЮ}\\
	\hfill \break 	\hfill \break	
	\Large{	Учебное пособие	
	}\\
\end{center}

		\hfill \break		\hfill \break	
\begin{figure}[h]
	\centering
	\includegraphics[width=60mm]{PolytechPressRu.png}
	%	\label{f:cover}	
\end{figure}
%\hfill \break
%\hfill \break
\begin{center}\Large{Санкт-Петербург \\
		\hfill \break
		2022} \end{center}
\thispagestyle{empty} % выключаем отображение номера для этой страницы


\newpage
УДК 519.9

ББК

~~~Б


\begin{center}
Р е ц е н з е н т ы:\\

Кандидат физико-математических наук, научный сотрудник Физико-технического института им. А.\,Ф.\,Иоффе
{\it А.\,А.\,Красилин}\\
Кандидат физико-математических наук, доцент Санкт-Петербургского политехнического  университета Петра Великого {\it Л.\,В.\,Павлова}
 \end{center}

%Доктор физико-математических наук, профессор НГУ С.П.Шарый
{\it Баженов\,А.\,Н.}
{\bf Введение в анализ данных с интервальной неопределенностью} : учеб. пособие /  А.\,Н.\,Баженов.
--- СПб. : ПОЛИТЕХ-ПРЕСС, 2022. --- XXX с.
\hfill \break

{\small 
Учебное пособие соответствует образовательному стандарту высшего
образования Федерального государственного автономного образовательного учреждения высшего образования «Санкт-Петербургский политехнический университет Петра Великого» по направлению подготовки бакалавров 01.03.02 <<Прикладная математика и информатика>>, по дисциплине «Интервальный анализ».


Пособие посвящено введению в анализ данных с интервальной неопределенностью
и демонстрации его применения в различных задачах.  
Важной частью пособия являются примеры, от самых простых, иллюстрирующих базовые конструкции и операции, до более сложных, находящихся на переднем крае методов интервального анализа данных. 


\hfill \break
Табл.~. Ил.~. Библиогр.: 48 назв.
\hfill \break
\hfill \break

 \begin{center}
{\small  	
Печатается по решению\\
Совета по издательской деятельности Ученого совета\\
Санкт-Петербургского политехнического  университета Петра Великого. }
 \end{center}

\hfill \break
\begin{tabular}{ll}
	~ & \copyright  \ Баженов\,А.\,Н., 2022 \\
{\bf ISBN 978-5-7422-\ldots} & \copyright \
Санкт-Петербургский политехнический \\
doi:10.18720/SPBPU/2/id22-\ldots & университет Петра Великого, 2022
\end{tabular}


\thispagestyle{empty}

% 2 стр пустые
\begin{center}
	\hfill \break
	Министерство науки и высшего образования  Российской Федерации\\
	%	\hfill \break
	$\ov{~~~~~~~~~~~~~~~~~~~~~~~}$\\
	\normalsize{	САНКТ-ПЕТЕРБУРГСКИЙ \\
		ПОЛИТЕХНИЧЕСКИЙ УНИВЕРСИТЕТ ПЕТРА ВЕЛИКОГО}\\ 
	$\ov{~~~~~~~~~~~~~~~~~~~~~~~~~~~~~~~~~~~~~~~~~~~~~~~~~~~~~~~~~~~~~~~~~~~~~~~~~~~~~~~~~~~~~~~~~~~~~~}$\\	
	Физико-механический институт\\
	Высшая школа прикладной математики и вычислительной физики\\
	%\hfill \break
	%\hfill \break
	%	\large{Институт прикладной математики и механики}\\
	%\hfill \break		
	%\hfill \break
	%	\large{Кафедра «Прикладная математика»}\\
	%\hfill \break
	%\hfill \break
	\hfill \break
	
	
	
	
	
	\Large{\it А.\,Н.\,Баженов\\
		\hfill \break		\hfill \break		}
	{\Large	ВВЕДЕНИЕ В АНАЛИЗ ДАННЫХ\\
		С ИНТЕРВАЛЬНОЙ НЕОПРЕДЕЛЕННОСТЬЮ}\\
	\hfill \break 	\hfill \break	
	\Large{	Учебное пособие	
	}\\
\end{center}

\hfill \break		\hfill \break	
\begin{figure}[h]
	\centering
	\includegraphics[width=60mm]{PolytechPressRu.png}
	%	\label{f:cover}	
\end{figure}
%\hfill \break
%\hfill \break
\begin{center}\Large{Санкт-Петербург \\
		\hfill \break
		2022} \end{center}
\thispagestyle{empty} % выключаем отображение номера для этой страницы


\newpage
УДК 519.9

ББК

~~~Б


\begin{center}
	Р е ц е н з е н т ы:\\
	
	Кандидат физико-математических наук, научный сотрудник Физико-технического института им. А.\,Ф.\,Иоффе
	{\it А.\,А.\,Красилин}\\
	Кандидат физико-математических наук, доцент Санкт-Петербургского политехнического  университета Петра Великого {\it Л.\,В.\,Павлова}
\end{center}

%Доктор физико-математических наук, профессор НГУ С.П.Шарый
{\it Баженов\,А.\,Н.}
{\bf Введение в анализ данных с интервальной неопределенностью} : учеб. пособие /  А.\,Н.\,Баженов.
--- СПб. : ПОЛИТЕХ-ПРЕСС, 2022. --- XXX с.
\hfill \break

{\small 
	Учебное пособие соответствует образовательному стандарту высшего
	образования Федерального государственного автономного образовательного учреждения высшего образования «Санкт-Петербургский политехнический университет Петра Великого» по направлению подготовки бакалавров 01.03.02 <<Прикладная математика и информатика>>, по дисциплине «Интервальный анализ».
	
	
	Пособие посвящено введению в анализ данных с интервальной неопределенностью
	и демонстрации его применения в различных задачах.  
	Важной частью пособия являются примеры, от самых простых, иллюстрирующих базовые конструкции и операции, до более сложных, находящихся на переднем крае методов интервального анализа данных. 
	
	
	\hfill \break
	Табл.~. Ил.~. Библиогр.: 48 назв.
	\hfill \break
	\hfill \break
	
	\begin{center}
		{\small  	
			Печатается по решению\\
			Совета по издательской деятельности Ученого совета\\
			Санкт-Петербургского политехнического  университета Петра Великого. }
	\end{center}
	
	\hfill \break
	\begin{tabular}{ll}
		~ & \copyright  \ Баженов\,А.\,Н., 2022 \\
		{\bf ISBN 978-5-7422-\ldots} & \copyright \
		Санкт-Петербургский политехнический \\
		doi:10.18720/SPBPU/2/id22-\ldots & университет Петра Великого, 2022
	\end{tabular}
	
	
	\thispagestyle{empty}

\newpage
%\tableofcontents

\newpage

%\listofexample 

%\cleardoublepage
\newpage

%\listoffigures

\newpage

	\chapter*{Введение}
\addcontentsline{toc}{chapter}{Введение}   



Учебное пособие является дополнением к книге коллектива авторов Баженова\,А.\,Н., Жилина\,С.\,И., Кумкова\,С.\,И., Шарого\,С.\,П.
<<Обработка и анализ данных с интервальной неопределенностью>> \cite{MetodikaBook}, изложение которой носит методически основательный характер.
В пособии представлена согласованная система понятий и терминов, относящихся к обработке данных, имеющих интервальную  неопределенность, а также
дан краткий обзор основных и наиболее значимых результатов научного направления, которое можно назвать <<статистикой интервальных данных>>, или же <<анализом интервальных 
данных>>. В пособии будет много ссылок на  \cite{MetodikaBook}, в которой теоретические аспекты изложены обстоятельно и подробно.
Задача пособия --- дать обучащимся краткие сведения о теории и рассмотреть ряд примеров, иллюстрирующих интервальный подход. %, и разобрать их с большей подробностью, чем это сделано в \cite{MetodikaBook}. 

Обрисуем общую ситуацию с доступным методическим и научным материалом.
Фундаментом статистики данных с интервальной неопределенностью является интервальный анализ. 
Основы интервального анализа представлены в  \cite{SPbSTU2020}. В  \cite{SPbSTU2021} дана картина применения интервальных данных и интервального анализа в более широком контексте. 
Наиболее полное изложение идей и методов интервального анализа дано в  \cite{SSharyBook}. Изучение материала  книги  \cite{SSharyBook} требует более основательной математической подготовки и рекомендуется для углубленного изучения вопроса.



%Учащиеся являются целевой аудиторией пособия. В целом оно адресовано всем, кто занимается обработкой и анализом данных: инженерам, технологам, исследователям.


%{\color{red} заимствовано}	

В самом общем виде задачи статистики данных с интервальной неопределенностью состоят в решении практических проблем в тех областях обработки данных, где недостаточны ранее развитые методы.
В практике обработки экспериментальных данных в настоящее время широко используются 
статистические методы, основанные на идеях и результатах теории вероятностей. Эти методы 
опираются на использование ряда допущений о вероятностных свойствах погрешностей 
измерений, а также на наличие выборок представительной длины (как минимум в несколько 
десятков измерений). 

Однако специалисты-практики часто сталкиваются с ситуациями, когда выборки 
измерений коротки, а погрешности 
измерений не могут быть адекватно описаны с помощью инструментов теории вероятностей, 
или же информация о вероятностных характеристиках погрешностей отсутствует. 

В этих ситуациях можно применить методы <<интервальной статистики>>, основанные на идеях и результатах интервального анализа, использующих его подходы, алгоритмы и соответствующее программное обеспечение. 
Интервальные методы широко представлены практически для всех популярных платформ программирования. В некоторых интегрированных средах, как, например, {\tt Mathematica}, {\tt Octave}, поддержка базовых интервальных конструкций встроенная. \index{Mathematica} \index{Octave} 
Для наиболее популярного в настоящее время языка программирования {\tt Python} также есть реализации основных конструкций и методов интервального анализа. \index{Python}

Терминология интервальной 
статистики наследует 
многое из традиционной статистики, в сложился развитый понятийный аппарат. 
Различным аспектам анализа интервальных данных посвящены, в частности,   \cite{SSharyJCT2017}-\cite{Kumkov2013}, \cite{NguyenKreinWuXiang}.

Следует отметить, что в статистике самые различные математические методы в XX в. продолжали развиваться и использоваться, но как будто не входя в математическую статистику. Дж.\,Тьюки в конце 50-х годов прошлого века предложил оформить 
новую научную дисциплину <<анализ данных>>,\index{анализ данных} в которой 
охватывались те математические методы обработки данных, которые не подпадали 
под математическую статистику в узком смысле этого слова \cite{Tukey1962}.


Материал учебного пособия апробирован в учебных курсах для студентов  
Высшей школы прикладной математики и вычислительной физики
 Физико-механического института Санкт-Петербургского политехнического университета Петра Великого
 и аспирантов Физико-технического интститута им. А.\,Ф.\,Иоффе Российской академии наук.
 

%{\color{red} заимствовано}	
\chapter[Краткие сведения о методах статистики и обработки данных]% 
{Краткие сведения о методах статистики и обработки данных}
\input{Chapter1.txt}

	\chapter[Базовые понятия и математический аппарат]% 
{Базовые понятия\\ и  математический аппарат} 
\label{PrimaryConceptChap} 

\input{Chapter2.txt}

	\chapter{Измерение постоянной величины} \label{MeasrConstChap}

%%%%%%%%%%%%%%%%%%%%%%%%%%%%%%%%%%%%%%%%%%%%%%%%%%%%%%%%%%%%%%%%%%%%%%%%%%%%%%%%%%%%%%%%  

\input{Chapter3.txt}

%	\chapter{Задача восстановления зависимостей} \label{FuncFitChap}

%\input{Chapter4.txt}

\input{Biblio.txt}
	
%	\addcontentsline{toc}{chapter}{Предметный указатель}
\raggedright\small\printindex   

% 1 стр пустая
%\blankpage
\thispagestyle{empty}

\begin{center}
	\hfill \break
	Министерство науки и высшего образования  Российской Федерации\\
	%	\hfill \break
	$\ov{~~~~~~~~~~~~~~~~~~~~~~~}$\\
	\normalsize{	САНКТ-ПЕТЕРБУРГСКИЙ \\
		ПОЛИТЕХНИЧЕСКИЙ УНИВЕРСИТЕТ ПЕТРА ВЕЛИКОГО}\\ 
	$\ov{~~~~~~~~~~~~~~~~~~~~~~~~~~~~~~~~~~~~~~~~~~~~~~~~~~~~~~~~~~~~~~~~~~~~~~~~~~~~~~~~~~~~~~~~~~~~~~}$\\	
	Физико-механический институт\\
	Высшая школа прикладной математики и вычислительной физики\\
	%\hfill \break
	%\hfill \break
	%	\large{Институт прикладной математики и механики}\\
	%\hfill \break		
	%\hfill \break
	%	\large{Кафедра «Прикладная математика»}\\
	%\hfill \break
	%\hfill \break
	\hfill \break
	
	
	
	
	
	\Large{\it А.\,Н.\,Баженов\\
		\hfill \break		\hfill \break		}
	{\Large	ВВЕДЕНИЕ В АНАЛИЗ ДАННЫХ\\
		С ИНТЕРВАЛЬНОЙ НЕОПРЕДЕЛЕННОСТЬЮ}\\
	\hfill \break 	\hfill \break	
	\Large{	Учебное пособие	
	}\\
\end{center}

\hfill \break		\hfill \break	
\begin{figure}[h]
	\centering
	\includegraphics[width=60mm]{PolytechPressRu.png}
	%	\label{f:cover}	
\end{figure}
%\hfill \break
%\hfill \break
\begin{center}\Large{Санкт-Петербург \\
		\hfill \break
		2022} \end{center}
\thispagestyle{empty} % выключаем отображение номера для этой страницы


\end{document}

	%%%%%%%%%%%%%%%%%%%%%%%%%%%%%%%%%%%%%%%%%%%%%%%%%%%%%%%%%%%%%%%%%%%  

	\begin{table}[h!]
	\centering
	\caption{Стабильные изотопы ртути.} 
	\medskip 
	\begin{tabular}{cc}
		Изотоп & Распространенность \\
		\hline
		$^{196}$Hg &  0,155 \% \\
		$^{198}$Hg & 10,04 \% \\
		$^{199}$Hg  & 16,94 \% \\
		$^{200}$Hg  & 23,14 \% \\
		$^{201}$Hg  &  13,17 \% \\
		$^{202}$Hg  &  29,74 \% \\
		$^{204}$Hg  &  6,82 \% \\
		\hline
	\end{tabular} 
	\label{Sulfur}
\end{table}		

%	Приведенные в таблице \ref{Sulfur} величины распространенности служат исходными данными для построения гистограммы частот. %, схематично представленной на Рис.~\ref{f:HistAtom}.
Конкретно для атомов ртути этот рисунок показан на Рис.~\ref{f:HistHg}.

	%%%%%%%%%%%%%%%%%%%%%%%%%%%%%%%%%%%%%%%%%%%%%%%%%%%%%%%%%%%%%%%%%%%  


\begin{example}{Полное сопротивление резонансного контура.}
	Полное сопротивление $Z$ цепи переменного тока складывается из активной и реактивной 
	составляющих, общая формула 
	\begin{equation}
	Z = \sqrt{ R^2 + (X_L - X_C)^2 }, \label{Zseq}
	\end{equation}
	где  $R$ --- активное сопротивление, $X_L = \omega\cdot L$ --- индуктивное 
	сопротивление, $X_C = 1/\omega C$ --- емкостное сопротивление (см., к примеру, 
	\cite{YavorskiDetlaf}). Как будет изменяться ток в цепи переменного тока с заданными 
	$R$, $L$, $C$, когда круговая частота тока меняется в интервале $[\,\omega_1, \omega_2 ]$? 
	
	Для ответа на вопрос можно воспользоваться законом Ома для цепи переменного тока
	\begin{equation*}
	I = U/Z, 
	\end{equation*} 
	и здесь нужно найти, прежде всего, интервал изменения $Z$. Выражение для полного 
	сопротивления $Z$ характерно тем, что индуктивное и емкостное сопротивления зависят 
	от частоты противоположным образом. 
	
	В качестве середины интервала частот возьмем $f= 13.56$ МГц. Диапазон вокруг этой частоты 
	может использоваться во всем мире без лицензий, он имеет специальное обозначение 
	--- ISM (Industrial, Scientific, Medical). 
	
	Рассмотрим практическое применение оборудования в диапазоне ISM: генератор 
	плазменного источника возбуждения-ионизации пробы для элементного анализа 
	с индуктивно-связанной плазмой (ИСП). 
	
	Индуктор плазмотрона ИСП (диаметр и длина 20 мм, три витка) имеет индуктивность 
	порядка $L=100$ нГ и реактивное сопротивление 8.52 Ом на частоте ISM. В таком случае 
	емкость резонансного контура имеет такое же реактивное сопротивление при $C=1.38$ нФ.  
	
	Пусть круговая частота $ \omega$ меняется в интервале $\ \pm 10 \%$ относительно 
	центра  диапазона ISM. 
	\begin{equation}
	\label{OmegaInt}
	\mbf{\omega} = [\,\omega_1, \omega_2 ] = [7.668, 9.372] \cdot 10^7 \ \text{Гц}.
	\end{equation}
	Напомним, что круговая частота связана с частотой колебаний $f$ соотношением 
	$\omega = 2 \pi f$. Активное сопротивление примем равным 1 Ом. Такой порядок имеет 
	сопротивление частично ионизированного аргона в плазмотроне. 
	
	Построим зависимости $X_L, X_C$ и $Z$ от частоты в интервале 
	$\mbf{\omega}$ \eqref{OmegaInt}. 
	
	%%%%%%%%%%%%%%%%%%%%%%%%%%%%%%%%%%%%%%%%%%%%%%%%%%%%%%%%%%%%%%%%%%%  
	
	\begin{figure}[ht]
		\centering\small  
		\setlength{\unitlength}{1mm} 
		\begin{picture}(80,55) 
		%\put(6,0){\includegraphics[width=75mm]{ZvalueNoLabel.png}} 
		\put(6,0){\includegraphics[width=75mm]{ZvalueNoLabelText.png}} 
		\put(40,-1){$\omega, 10^{7}$ Гц}
		\put(-10,25){$X_L, X_C, Z,$ Ом}
		\end{picture} 
		\caption{Зависимости $X_L, X_C$ и $Z$ от частоты.} 
		\label{Xresonance}  
	\end{figure} 
	
	
	%%%%%%%%%%%%%%%%%%%%%%%%%%%%%%%%%%%%%%%%%%%%%%%%%%%%%%%%%%%%%%%%%%%
	
	Определим границы изменения величины $Z$. В точке резонанса  $X_L = X_C$ и суммарное 
	сопротивление $LC$-части схемы равно нулю, а полное сопротивление равно активному, 
	т.\,е. $Z=R$. Таким образом, 
	\begin{equation*} 
	\min_{\omega} Z = R =1. 
	\end{equation*} 
	
	Величины индуктивного $X_L$ и емкостного $X_C$ сопротивлений зависят от частоты 
	противоположным образом и монотонны. Поэтому максимальные значения величины 
	$X_{LC} = | X_L - X_C | $ достигаются на одном из краев диапазона $\mbf{\omega}$: 
	\begin{equation*} 
	\max_{\omega} X_{LC} = \max \left\lbrace X_{LC} ( \un {\mbf{\omega}}), X_{LC} ( \ov {\mbf{\omega}}) 
	\right\rbrace =  \max \left\lbrace 1.6401, 1.7822 \right\rbrace = 1.7822. 
	\end{equation*} 
	Соответственно, получаем значение
	\begin{equation*} 
	\max Z = \sqrt{R^2 + \max_{\omega} X^2_{LC}} =2.0436.
	\end{equation*} 
	%		{\color{red}$\Delta$ - заменить} 
	
	Окончательно имеем точный интервал величины полного сопротивления в интервале 
	$\mbf{\omega}$ 
	\begin{equation*} 
	\mbf{Z} = [1, 2.0436].
	\end{equation*} 
	
	
	Вычислим также интервал изменения полного сопротивления $\mbf{Z}$ по формуле \eqref{Zseq}, 
	представленной двумя способами: 
	\begin{align}
	\mbf{Z}_1 &= \sqrt{R^2 + \left( \mbf{\omega} L - \frac{1}{\mbf{\omega} C} \right) ^2}, 
	\label{Z1} \\[3mm] 
	\mbf{Z}_2 &= \sqrt{R^2 + \mbf{\omega}^2\left( L - \frac{1}{\mbf{\omega}^2 C} \right)^2}. 
	\label{Z2}
	\end{align} 
	Вычисления дают величины сопротивлений соответственно
	\begin{align*}
	\mbf{Z}_1 &= [1, 2.0436] \ \text{Ом}, \\[1mm]
	\mbf{Z}_2 &= [1, 2.3968] \ \text{Ом}.
	\end{align*} 
	Нижние величины $Z_1, \ Z_2$ соответствуют частотному резонансу $X_L=X_C$ в середине 
	диапазона, верхние относятся к краям выбранного диапазона, иначе --- расстройке резонанса. 
	
	В выражении \eqref{Z2} радиус выражения больше, чем в \eqref{Z1} в связи с дополнительным умножением на величину $\mbf{\omega}^2$.
	
	%Для последовательного колебательного контура в $RLC$-цепях, в котором все три элемента включены последовательно добротность 
	%\begin{equation}
	%Q = \frac{1}{R} \sqrt{\frac{L}{C}}
	%\end{equation}
\end{example}


\begin{example}{Твины для описания составных ошибок измерений.}
	Рассмотрим измерение так называемых осцилляций нейтрино, результаты измерений которых 
	удобно представить в виде твинов. 
	
	При измерении осцилляций нейтрино в атмосфере Земли экспериментаторы традиционно 
	используют безразмерную величину $R$, характеризующую отношение числа разных сортов 
	нейтрино. Подборка значений $R$ из разных экспериментов приведена на стр.~872 
	в публикации \cite{UFN1997}. Приведем часть данных из этой публикации: 
	\begin{align}
	R_1 &=  0.60^{+0.07}_{-0.06} \pm 0.05, && \text{<<Kamiokande>>}, \label{Kamiokande}\\[4pt]  
	R_2 &=  0.54 \pm 0.05 \pm 0.12, &&   \text{<<IMB>>}. \label{IMB}
	\end{align} 
	Результаты измерений даны  в форме <<базовое значение, статистическая погрешность, 
	систематическая погрешность>>, а после числовых данных приводится название проекта, 
	в ходе которого они были получены.                                
	В первом примере (<<Kamiokande>>) статистическая погрешность дана в виде границ, 
	несимметричных относительно среднего значения. Такая ситуация возникает при оценке 
	значения величины, входящей в нелинейную функцию. Она соответствует 
	общему случаю модели погрешности \eqref{GeneralErrorModel}, рассмотренной выше, 
	в разделе \ref{MeasuResultSect}. 
	
	В виде твинов данные \eqref{Kamiokande} и \eqref{IMB} можно представить следующим 
	образом. В качеcтве первого шага выразим результаты в виде обычных интервалов $\mbf{r}_1$ 
	и $\mbf{r}_2$, с учетом только статистических погрешностей: 
	\begin{align}
	\mbf{r}_1  &= [\; 0.6-0.06, \  0.6+0.07 \;] \  = \  [\; 0.54, 0.67 \;], \label{r1} \\[3pt]  
	\mbf{r}_2  &= [\; 0.54-0.05,  0.54+0.05 \;] = [\; 0.49, 0.59 \;]. \label{r2} 
	\end{align}	
	При этом $\w{\mbf{r}_1} = 0.13 > \w{\mbf{r}_2} = 0.1$ ввиду более широких статистических 
	оценок для величины $R_1$.	
	
	Далее, произведем учет систематической погрешности, произведя <<интервализацию>> концов 
	интервалов $\mbf{r}_1$ и  $\mbf{r}_2,$ вычитая и добавляя величины систематических 
	погрешностей к величинам $\un{\mbf{r}}_1$, $\ov{\mbf{r}}_1$, $\un{\mbf{r}}_2$, 
	$\ov{\mbf{r}}_2$. 
	
	Обозначим получившиеся твины как $\mbf{R}_1$ и $\mbf{R}_2$:  
	\begin{align}
	\mbf{R}_1  
	&= \bigl[\;[ 0.54-0.05,  0.54+0.05], \  [0.67-0.05,  0.67+0.05]\;\bigr] \notag\\[3pt] 
	&= \bigl[\;[ 0.49,  0.59],  \ [0.62,  0.72 ]\;\bigr], \label{R1}     \\[3mm] 
	\mbf{R}_2  
	&= \bigl[\;[ 0.49-0.12,  0.49+0.12], \  [0.59-0.12,  0.59+0.12]\;\bigr] \notag\\[3pt] 
	&= \bigl[\;[ 0.37,  0.61], \  [0.47,  0.71 ]\;\bigr]. \label{R2} 
	\end{align}
	
	На рисунке \ref{TwinsRnu2} графически представлены твины $\mbf{R}_1$ и $\mbf{R}_2$. 
	Численные значения концов правого интервала смещены вверх. На сей раз твин $\mbf{R}_1$  
	<<\'{у}же>>, чем  твин $\mbf{R}_2$, ввиду более широких систематических погрешностей 
	для величины $R_2$. Следует заметить также, интервалы  $\un{\mbf{R}}_2$, $\ov{\mbf{R}}_2$  
	в форме \eqref{Twin} имеют ненулевое пересечение. Это пересечение дано более темной 
	заливкой, чем концы твина. 
	
	\begin{figure}[hbt]
		\centering\small 
		\setlength{\unitlength}{1mm}
		\begin{picture}(70,17)
		\put(0,0){\includegraphics[width=70mm]{TwinRnu}}
		\put(-5,6.6){\vector(1,0){80}} \put(71.5,7.6){$\mbb{R}$} 
		\put(18,10){{\footnotesize 0.49}} 
		\put(30,10){{\footnotesize 0.59}} 
		\put(37,12){{\footnotesize 0.62}} 
		\put(47,12){{\footnotesize 0.72}}  
		\put(35,1){$\mbf{R}_1$}  
		\end{picture}
		%\caption{Данные по массе нейтрино как твин.} 
		%	\label{TwinsRnu} 
	\end{figure}
	%	\vspace{-10mm}
	\begin{figure}[hbt]
		\centering\small 
		\setlength{\unitlength}{1mm}
		\begin{picture}(70,17)
		\put(0,0){\includegraphics[width=70mm]{TwinR2nu}}
		\put(-5,6.6){\vector(1,0){80}} \put(71.5,7.6){$\mbb{R}$} 
		\put(3,10){{\footnotesize 0.37}} 
		\put(16,12){{\footnotesize 0.47}} 
		\put(33,10){{\footnotesize 0.61}} 
		\put(47,12){{\footnotesize 0.71}} 
		\put(30,1){$\mbf{R}_2$}  
		\end{picture}
		\caption{Данные по физике нейтрино в форме твинов типа <<$\leq$>>.} 		
		\label{TwinsRnu2} 
	\end{figure} 
	
	%%%%%%%%%%%%%%%%%%%%%%%%%%%%%%%%%%%%%%%%%%%%%%%%%%%%%%%%%%%%%%%%%%%%%%%%%%%%%%%%%%%%
	
	Далее, можно проводить различный содержательный анализ величин $\mbf{R}_1$  и $\mbf{R}_2$. 
\end{example}  

	%%%%%%%%%%%%%%%%%%%%%%%%%%%%%%%%%%%%%%%%%%%%%%%%%%%%%%%%%%%%%%%%%%%%%%%%%%%%%%%%%%%%
Эти данные приводятся в Табл.~\ref{TableDataCover2}.

\begin{table}[h!]
	\begin{center}
		\begin{tabular}{| c | c c | }
			\hline
			Номер замера & Peak &   {\tt std}  Peak \\ % &  BG &   {\tt std} BG \\
			\hline
			1 &	-4.4 & 2.7 \\ % & 4.2 & 6.7 \\
			2 & -3.4 & 1.9 \\ % & -3.2 &	4.8 \\
			3 & -6.9 & 2.4 \\ % & 12.1 &	9 \\
			%				4 &	-1.2 & 2.4 \\ % & 12.4 &	7.2 \\
			%				5 &	-1.0 & 2.7 \\ % & 9.4 & 5.1 \\
			%				6 &	-10.8 &	3.5 \\ % &1	& 12.4 \\
			%				7 &	-10.2 &	2.8 \\ % &-0.6 &	6.1 \\
			8 &	-6.3 &	2 \\ % &	3.9 &	4.3\\
			%				9 &	-10.4 &	4.1 \\ % &	10.3 &	10\\
			%				10 & 0.6& 3.4 \\ % & -4.8 & 10.6\\
			%				11 &-1.8 &	2 \\ % &	4.6&	4.2\\
			12 &-6.6 & 2.1	\\ % &-5.7&4.6\\
			13 &-4.9 &2.1 \\ % &	13 &3 \\
			14 &-6.0 &	2.4 \\ % & 8.4	&4.6\\
			15 &-4.0 & 2.7 \\ %	& 10.6 &5.5\\
			\hline	
		\end{tabular}
	\end{center}
	\caption{Накрывающая подвыборка данных таблицы 1.}
	\label{TableDataCover2}
\end{table}
	%%%%%%%%%%%%%%%%%%%%%%%%%%%%%%%%%%%%%%%%%%%%%%%%%%%%%%%%%%%%%%%%%%%%%%%%%%%%%%%%%%%%
Рассмотрим теперь, что было бы при более низкой точности цифрового измерителя, чем в только что рассмотренном примере.  Пусть она равна 9-ти двоичным разрядам. 
На Рис.~\ref{DRS4ZeroLine100cell2} представлены 
интервальные результаты измерения нуля %цифрового измерителя напряжения 
%данные для 100 измерений базовой линии 
с теми же  данными $\left\lbrace \mathring{x}_k\right\rbrace _{k=1}^{100}$. 
\begin{figure}[htb]
	\centering\small 
	\unitlength=1mm
	\begin{picture}(100,58)
	\put(-10,53){\mbox{\small Данные}} 
	\put(-10,50){\mbox{\small измерений, В}}	
	\put(90,50){\mbox{\small $ \max_{1\leq k\leq n} \ov{\mbf{x}}_{k}$}} 
	\put(90,30){\mbox{\small $x_\text{c}  $}}	
	\put(10,0){\includegraphics[width=0.7\textwidth]{ZeroLineCh=1cell=1resolution=9.png}}
	\put(90,11){\mbox{\small $ \min_{1\leq k\leq n} \un{\mbf{x}}_{k}$}} 
	\put(87,5){\mbox{\small Номер}} 
	\put(87,2){\mbox{\small измерения}} 
	\end{picture}
	\caption{Диаграмма рассеяния интервальных   измерений}
	неопределенности нуля.
	Разрядность измерителя $\mathit{N \!O\!B} = 9$.
	\label{DRS4ZeroLine100cell2} 
\end{figure}  

В  случае более грубых измерений, характер совместности отдельных измерений существенно 
изменился. Для многих пар замеров $(i, j)$ имеет место непустота пересечения интервалов:  $ \mbf{x}_i \cap \mbf{x}_j \neq\varnothing$. 

%%%%%%%%%%%%%%%%%%%%%%%%%%%%%%%%%%%%%%%%%%%%%%%%%%%%%%%%%%%%%%%%%%%%%%%%%%%%%%%%%%%%%%%%  

\begin{figure}[htb]
	\centering\small 
	\unitlength=1mm
	\begin{picture}(100,58)
	\put(-10,53){\mbox{\small Число}} 
	\put(-10,50){\mbox{\small измерений}}
	\put(-10,47){\mbox{\small в столбце}}
	\put(-10,44){\mbox{\small гистограммы}}
	\put(10,0){\includegraphics[width=0.7\textwidth]{HISTZeroLineResolution=9.png}}
	\put(87,5){\mbox{\small Диапазон}} 
	\put(87,2){\mbox{\small измерений, В}} 
	\end{picture}
	\caption{Гистограмма данных $\left\lbrace \mathring{x}_k\right\rbrace _{k=1}^{100}$ 
		интервальных   измерений } неопределенности нуля.
	Разрядность измерителя $\mathit{N\!O\!B} = 9$.
	\label{HISTZeroLine2} 
\end{figure}  

%%%%%%%%%%%%%%%%%%%%%%%%%%%%%%%%%%%%%%%%%%%%%%%%%%%%%%%%%%%%%%%%%%%%%%%%%%%%%%%%%%%%%%%%%

Характер изменения статуса пересечения данных ярко демонстрируется гистограммой, 
представленной на Рис.~\ref{HISTZeroLine2}. Ширина столбца соответствует точности 
измерения. Распределение результатов измерения неопределенности нуля цифрового измерителя напряжения
%амплитуд смещений базовой линии 
весьма близко к равномерному. 
% неопределенность измерения нуля цифрового измерителя напряжения
Согласно выражению \eqref{UNInterval} имеем оценку информационного множества неопределенности нуля
\begin{equation*} 
%\label{UNInterval} 
\mbf{J}\; = \;\bigvee_{1\leq k\leq n} \mbf{x}_{k} \ 
= \  \Bigl[\,\min_{1\leq k\leq n} \un{\mbf{x}}_{k}, 
\max_{1\leq k\leq n} \ov{\mbf{x}}_{k}\,\Bigr] = \left[ -1.16 \cdot 10^{-3}, 8.79 \cdot 10^{-3} \right].
\end{equation*} 
Точечная оценка измеряемой величины \eqref{midUNInterval} не изменилась:
\begin{equation*}
x_\text{c} \  = \  \m\mbf{J} \   
= \  \tfrac{1}{2} \Bigl(\,\min_{1\leq k\leq n} \un{\mbf{x}}_{k} + 
\max_{1\leq k\leq n} \ov{\mbf{x}}_{k}\,\Bigr) = 3.82 \cdot 10^{-3}. 
\end{equation*} 

Результат более грубыx измерений выглядит, на первый взгляд более надежным, 
чем более точных. Гистограмма на Рис.~\ref{HISTZeroLine2} демонстрирует гораздо 
более высокую степень однородность представителей интервальной выборки Рис.~\ref{DRS4ZeroLine100cell2}. 

В случае решения задач восстановления зависимостей по интервальным  данным, см. \S\ref{FuncFitChap}, подобное  
явление носит название парадокса Е.З.\,Демиденко, см. \S\ref{DemidParadoxSect}, 
суть которого может быть кратко выражена фразами <<Чем грубее --- тем лучше>> или 
<<Чем точнее --- тем хуже>>.  \index{парадокс Демиденко} 

Это, тем не менее, неточно. Оценка информационного множества  $\mbf{J}$
заметно расширилась для более грубых измерений. В определенность смысле можно сказать, что огрубленная дискретность измерений поглотила в существенной степени  неучтенную систематическую 
погрешность, сделала ее менее заметной. При этом сами результаты не стали более качественными. 

Подводя итог, можно заметить, что более грубое измерение или, что эквивалентно, 
отбрасывание некоторого количества младших разрядов, не улучшает интерпретацию выборки 
интервальных данных с большим числом пустых множеств пересечения интервалов замеров. 
При этой огрублении утрачивается часть информации, объективно содержащейся в выборке $\left\lbrace \mathring{x}_k\right\rbrace _{k=1}^{100}$. 
Процедура варьирования неопределенности  по методике \S\ref{UncertAlterSect} для достижения совместности данных выглядит более содержательной, поскольку дает как достижение свойства накрытия модифицированной выборки, так и количественную оценку увеличения величины радиусов отдельных замеров.
	
	%%%%%%%%%%%%%%%%%%%%%%%%%%%%%%%%%%%%%%%%%%%%%%%%%%%%%%%%%%%%%%%%%%%%%%%%%%%%%%%%%%%%
	
		%%%%%%%%%%%%%%%%%%%%%%%%%%%%%%%%%%%%%%%%%%%%%%%%%%%%%%%%%%%%%%%%%%%%%%%%%%%%%%%%%%%%%%%
	\begin{figure}[htb]
		\centering\small 
		\unitlength=1mm
		%	\begin{picture}(90,67)
		%	\put(0,0){\includegraphics[width=90mm]{PgammaPhMean.png}}
		\includegraphics[width=0.7\textwidth]{PgammaPhMean.png}
		%	\put(85,2){\mbox{\small номер измерения}} 
		%	\end{picture}
		\caption{Диаграмма рассеяния интервальных измерений  и точечная оценка \eqref{xc}.} 
		\label{IMeanNonCover} 
	\end{figure} 
	%%%%%%%%%%%%%%%%%%%%%%%%%%%%%%%%%%%%%%%%%%%%%%%%%%%%%%%%%%%%%%%%%%%%%%%%%%%%%%%%%%%%%%%
	
	
	\paragraph{Относительная ширина интервала.} 
	Ширина интервала и его радиус являются мерами абсолютного рассеяния точек интервала. 
	Но иногда требуется охарактеризовать относительную меру этого рассеяния, т.\,е., 
	фактически, перейти к аналогу относительной погрешности. В интервальном анализе 
	не существует одной общеупотребительной конструкции для этой цели, так как  
	для различных типов интервалов приходится использовать разные меры. 
	%Основная  причина состоит в том, что интервалы, содержащие нуль, представляют относительную  неопределенность, которая превосходит 100\%, что очень много и совершено нетипично  для этой величины. Для адекватного оперирования с <<относительным рассеянием>> в самом общем случае требуются какие-то другие инструменты. 
	
	Полезной характеристикой интервала является так называемый \emph{функционал Рачека} $\chi$:
	\begin{equation*} 
	\chi(\mbf{a}) = 
	\left\{ \ 
	\begin{array}{ll}
	\un{\mbf{a}}/\ov{\mbf{a}}, & \text{ если } \;\un{\mbf{a}}\leq\ov{\mbf{a}},\\[1mm] 
	\ov{\mbf{a}}/\un{\mbf{a}}, & \text{ иначе. } 
	\end{array}
	\right. 
	\end{equation*} 
	Он характеризует <<относительную узость>> интервала, и обсуждение его свойств можно 
	найти в %работе \cite{IreneJCT1997} или в 
	книге \cite{SSharyBook}. 
	\index{функционал Рачека}
	
		
	\begin{table}[h!tb]
		\begin{center}
			\begin{tabular}{| c | c c | }
				\hline
				Номер замера & Peak &   {\tt std}  Peak \\ % &  BG &   {\tt std} BG \\
				\hline
				1 &	-4.4 & 2.7 \\ % & 4.2 & 6.7 \\
				2 & -3.4 & 1.9 \\ % & -3.2 &	4.8 \\
				3 & -6.9 & 2.4 \\ % & 12.1 &	9 \\
				4 &	-1.2 & 2.4 \\ % & 12.4 &	7.2 \\
				5 &	-1.0 & 2.7 \\ % & 9.4 & 5.1 \\
				6 &	-10.8 &	3.5 \\ % &1	& 12.4 \\
				7 &	-10.2 &	2.8 \\ % &-0.6 &	6.1 \\
				8 &	-6.3 &	2 \\ % &	3.9 &	4.3\\
				9 &	-10.4 &	4.1 \\ % &	10.3 &	10\\
				10 & 0.6& 3.4 \\ % & -4.8 & 10.6\\
				11 &-1.8 &	2 \\ % &	4.6&	4.2\\
				12 &-6.6 & 2.1	\\ % &-5.7&4.6\\
				13 &-4.9 &2.1 \\ % &	13 &3 \\
				14 &-6.0 &	2.4 \\ % & 8.4	&4.6\\
				15 &-4.0 & 2.7 \\ %	& 10.6 &5.5\\
				\hline	
			\end{tabular}
		\end{center}
		\caption{Данные таблицы 1  для величины $\delta \times 10^{5}$ \cite{Pgamma1992}.}
		\label{TableDataV}
	\end{table}

	\begin{figure}[h!] 
	\centering\small 
	\unitlength=1mm
	\begin{picture}(100,48)
	\put(5,39){\mbox{\small $w$}} 
	\put(-5,42){\mbox{\small Значения}}
	\put(10,0){\includegraphics[width=70mm]{weightL1.png}}
	\put(87,5){\mbox{\small номер}} 
	\put(87,2){\mbox{\small измерения}} 
	\end{picture}
	\caption{Значения весов в задаче оптимизации \eqref{L1opt}.}
	\label{weightL1fig}
\end{figure}

	\begin{table}[h!]
	\begin{center}
		\begin{tabular}{|c|c|}
			\hline
			Номер измерения & Данные энкодера\\	
			\hline
			1 & 388 \\
			2 & 737 \\
			3 & 951 \\
			4 & 1354 \\
			5 & 1756 \\
			6 & 1970 \\
			7 & 2399 \\
			8 & 2801 \\
			9&  3204 \\
			10& 3606 \\
			\hline
		\end{tabular}
		\caption{Подвыборка из данных на Рис.~\ref{EncoderStepData}.  }
		\label{TableDataEncoderpart2}
	\end{center}
\end{table}

\begin{equation} 
\label{InLinEqSys} 
\arraycolsep=2pt 
\left\{ \ 
\begin{array}{ccccccccccc}
\beta_0 &+& \mbf{x}_{11}\beta_1 &+& 
\mbf{x}_{12} \beta_2 &+& \ldots &+& \mbf{x}_{1m}\beta_m &=& \mbf{y}_{1}, \\[3pt] 
\beta_0 &+& \mbf{x}_{21}\beta_1 &+& 
\mbf{x}_{22} \beta_2 &+& \ldots &+& \mbf{x}_{2m}\beta_m &=& \mbf{y}_{2}, \\[3pt] 
\vdots &&  \vdots && \vdots && \ddots && \vdots && \vdots                  \\[3pt]  
\beta_0 &+& \mbf{x}_{n1}\beta_1 &+& 
\mbf{x}_{n2} \beta_2 &+& \ldots &+& \mbf{x}_{nm}\beta_m &=& \mbf{y}_{n}. 
\end{array} 
\right. 
\end{equation} 