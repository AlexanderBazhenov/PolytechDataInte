
%%%%%%%%%%%%%%%%%%%%% GREEK TEXT %%%%%%%%%%%%%%%%%%%%%%%%%%%%%%%%%%%%%%%%
\usepackage{textgreek}
\renewcommand{\mu}{\text{\textmu}} 
\renewcommand{\delta}{\text{\textdelta}} 
\renewcommand{\Delta}{\text{\textDelta}} 
\renewcommand{\alpha}{\text{\textalpha}} 
\renewcommand{\beta}{\text{\textbeta}} 
\renewcommand{\epsilon}{\text{\textepsilon}}
\renewcommand{\rho}{\text{\textrho}}
\renewcommand{\gamma}{\text{\textgamma}}
%%%%%%%%%%%%%%%%%%%%% GREEK TEXT %%%%%%%%%%%%%%%%%%%%%%%%%%%%%%%%%%%%%%%%



\usepackage{imakeidx}
\makeindex [title=ПРЕДМЕТНЫЙ УКАЗАТЕЛЬ]

%%%%%%%%%%%%%%%%%%%%%%%%%%%%% CHAPTER %%%%%%%%%%%%%%%%%%%%%%%%%%%%%
% https://latex.org/forum/viewtopic.php?f=5&t=2521#p9922
\usepackage[compact]{titlesec} 
\titleformat{\chapter}{%
	%	\centering\normalfont\Large\bfseries}{\thechapter}{1em}{}
\centering\normalfont\Large\bfseries}{ {\large  {\tt Г~л~а~в~а~ \thechapter}} }{1em}{}
\titlespacing*{\chapter} {0pt}{3.5ex plus 1ex minus .2ex}{2.3ex plus .2ex}
%  \titleformat{\section}{%
%	\centering\normalfont\Large\bfseries}{\thesection}{1em}{}
%%%%%%%%%%%%%%%%%%%%%%%%%%%%% CHAPTER %%%%%%%%%%%%%%%%%%%%%%%%%%%%%

%%%%%%%%%%%%%%%%%%%%%%%%%%%%% SECTION %%%%%%%%%%%%%%%%%%%%%%%%%%%%%
\titleformat{\section}
{\centering\Large\bfseries}{\thesection.}{1em}{}
\titleformat{\subsection}
{\centering\large\bfseries\itshape}{\thesubsection.}{1em}{}
%%%%%%%%%%%%%%%%%%%%%%%%%%%%% SECTION %%%%%%%%%%%%%%%%%%%%%%%%%%%%%	

%%%%%%%%%%%%%%%%%%%%%%%%%%%%% SECTION  Adds . in TOC %%%%%%%%%%%%%%
\usepackage{secdot}% Adds . after sectional unit numbers
\usepackage{etoolbox}
% \patchcmd{<cmd>}{<search>}{<replace>}{<success>}{<failure>}
\patchcmd{\numberline}{\hfil}{.\hfil}{}{}
%%%%%%%%%%%%%%%%%%%%%%%%%%%%% SECTION  Adds . in TOC %%%%%%%%%%%%%%	

%%%%%%%%%%%%%%%%%%%%%%%%%%%%% PAGE SIZE %%%%%%%%%%%%%%	
\textwidth=114truemm
\textheight=165truemm
\oddsidemargin=-1cm
\evensidemargin=\oddsidemargin
%\topmargin=-1cm
\topmargin=-2cm
\sloppy

\pagestyle{plain}
%\mathsurround=1pt
%\tolerance=400
%\hfuzz=2pt
%%%%%%%%%%%%%%%%%%%%%%%%%%%%% PAGE SIZE %%%%%%%%%%%%%%	

%%%%%%%%%%%%%%%%%%%%%%%%%%%%% PAGE NUMBERS %%%%%%%%%%%%%%%%%%%%%%%%%%%%%%
\usepackage{fancyhdr}
\fancyhf{} % clear all header and footers
\renewcommand{\headrulewidth}{0pt} % remove the header rule
\fancyfoot[LE,RO]{\thepage} % Left side on Even pages; Right side on Odd pages
\pagestyle{fancy}
\fancypagestyle{plain}{%
\fancyhf{}%
\renewcommand{\headrulewidth}{0pt}%
\fancyhf[lef,rof]{\thepage}%
}
%%%%%%%%%%%%%%%%%%%%%%%%%%%%% PAGE NUMBERS %%%%%%%%%%%%%%%%%%%%%%%%%%%%%%


%%%%%%%%%%%%%%%%%%%%%%%%%%%%%%%%%%%%%%%%%%%%%%%%%%%%%%%%%%%%%%%%%%%%%%%%%%%%%%%%%%%%%%%%%%%%%%%%%%%%%%%%%%%%
\usepackage{enumerate} % Тонкая настройка списков
\usepackage{indentfirst} % Красная строка после заголовка
\usepackage{float} % Расширенное управление плавающими объектами


%%%%%%%%%%%%%%%%%%%%%%% ОГЛАВЛЕНИЕ %%%%%%%%%%%%%%%%%%%%%%%
%\renewcommand{\contentsname}{ОГЛАВЛЕНИЕ}

\usepackage{tocloft} % Alter the style of the Table of Contents
% 2022-12-08
\renewcommand{\cftsecfont}{\rmfamily\mdseries\upshape}
%\renewcommand{\cftsecpagefont}{\rmfamily\mdseries\upshape} % No bold!

% Содержание
% ОГЛАВЛЕНИЕ
%\renewcommand{\cfttoctitlefont}{\normalfont\MakeUppercase}
%\renewcommand{\cfttoctitlefont}{\hspace*{\fill}\normalfont\MakeUppercase}
\renewcommand{\cfttoctitlefont}{\hspace{4cm}\normalfont\bfseries\MakeUppercase}
%\renewcommand{\cfttoctitlefont}{\centering\normalfont\bfseries\MakeUppercase}
%\renewcommand{\cftaftertoctitle}{\hfill}
% СОДЕРЖАНИЕ
%\renewcommand{\cftsecfont}{\hspace{0pt}}            % Имена секций в содержании не жирным шрифтом
\renewcommand\cftsecleader{\cftdotfill{\cftdotsep}} % Точки для секций в содержании
\renewcommand\cftsecpagefont{\mdseries}             % Номера страниц не жирные
\setcounter{tocdepth}{3}                            % Глубина оглавления, до subsubsection
\renewcommand{\cftsecaftersnum}{.}					% Точки после номера секций в содержании
\renewcommand{\cftsubsecaftersnum}{.}				% Точки после номера подсекций в содержании
%

%\addto\captionsenglish{\renewcommand{\chaptername}{Lecture}}
\addto\captionsrussian{\renewcommand{\chaptername}{Г~л~а~в~а}}
%\renewcommand{\chaptername}{Раздел}
\renewcommand{\cftchappagefont}{\normalfont}
%\renewcommand{\cftchapfont}{\normalfont\large\itshape}    % \chapter font in ToC
%\renewcommand{\cftchapfont}{\bfseries\large}    % \chapter font in ToC
\renewcommand{\cftchapfont}{\bfseries}    % \chapter font in ToC

\setlength{\cftchapnumwidth}{0pt}
%\renewcommand{\cftchappresnum}{\MakeUppercase\chaptername\ }
\renewcommand{\cftchappresnum}{\normalfont\chaptername\ }
\renewcommand{\cftchapaftersnum}{.}						% Точка после номера главы в содержании
%\renewcommand{\cftchapaftersnumb}{\newline}
\renewcommand{\cftchapaftersnumb}{~~~~~~~~~~~~~}
\renewcommand{\cftchapdotsep}{\cftdotsep}
\renewcommand\cftchapleader{\cftdotfill{\cftdotsep}}	% Точки после имени главы в содержании
%%%%%%%%%%%%%%%%%%%%%%%  PART %%%%%%%%%%%%%%%%%%%%%%%%%%%%%%%%%%%%%%%%%%%%%%%%%%%%%%%%%%
% для нестандартного оформления оглавления
\renewcommand{\cftpartpagefont}{\normalfont}   		 % номер страницы
\renewcommand{\cftpartfont}{\normalfont} 			 % шрифт раздела
\renewcommand\cftpartleader{\cftdotfill{\cftdotsep}} % Точки 
%%%%%%%%%%%%%%%%%%%%%%%  PART %%%%%%%%%%%%%%%%%%%%%%%%%%%%%%%%%%%%%%%%%%%%%%%%%%%%%%%%%%

%%%%%%%%%%%%%%%%%%%%%%% ОГЛАВЛЕНИЕ %%%%%%%%%%%%%%%%%%%%%%%

%%%%%%%%%%%%%%%%%%%%%%%%%%%%%%%%  Списки %%%%%%%%%%%%%%	
\usepackage{enumitem}
\setlist[enumerate,itemize]{leftmargin=0pt,itemindent=2.5em} % Отступы в списках
\makeatletter
\AddEnumerateCounter{\asbuk}{\@asbuk}{м)}
\makeatother
\setlist{nolistsep}
\renewcommand{\labelitemi}{--}
\renewcommand{\labelenumi}{\asbuk{enumi})}
\renewcommand{\labelenumii}{\arabic{enumii})}
%%%%%%%%%%%%%%%%%%%%%%%%%%%%%%%%  Списки %%%%%%%%%%%%%%	


%%%%%%%%%%%%%%%%%%%%%%%%%%%%%%%%%%%%%%%%%%%%%%%%%%%%%%%%%%%%%%%%%%%%%%%%%%%%%%%%%%%%%%%%%%%%%%%%%%%%%%%%%%%%


%%%%%%%%%%%%%%%%%%%%%%%%%%%%% CAPTIONS TABLES FIGURES %%%%%%%%%%%%%%%%%%%%%%%%%%%%%%
\captionsetup{font=small,labelsep=period,margin=7mm} 
%%%%%%%%%%%%%%%%%%%%%%%%%%%%% CAPTIONS %%%%%%%%%%%%%%%%%%%%%%%%%%%%%
\captionsetup[figure]{
	font = small,
	labelfont = it
}
\captionsetup[table]{
	font = small,
	labelfont = it
}	
%\usepackage{lipsum, mwe}
\setlength\abovecaptionskip{0pt plus 0pt minus 4pt}
\setlength\belowcaptionskip{0pt plus 0pt minus 4pt}	

\captionsetup[table]{labelformat=empty}
\newcommand{\TABLENAME}{\raggedleft\small\textit{\tablename}\hspace{1mm}}
%%%%%%%%%%%%%%%%%%%%%%%%%%%%% CAPTIONS TABLES FIGURES %%%%%%%%%%%%%%%%%%%%%%%%%%%%%%

%%%%%%%%%%%%%%%%%%%%%%%%%%%%%%% thebibliography %%%%%%%%%%%%%%%%%%%%%%%%%%%%%%% 
\makeatletter
\renewcommand\@biblabel[1]{#1.}
\makeatother

%%%%%%%%%%%%%%%% Библиография: отступы и межстрочный интервал %%%%%%%%%%%%%%%% 
\makeatletter
\renewenvironment{thebibliography}[1]
{\section*{\refname}
\list{\@biblabel{\@arabic\c@enumiv}}
{\settowidth\labelwidth{\@biblabel{#1}}
	\leftmargin\labelsep
	\itemindent 10mm %16.7mm
	\@openbib@code
	\usecounter{enumiv}
	\let\p@enumiv\@empty
	\renewcommand\theenumiv{\@arabic\c@enumiv}
}
\setlength{\itemsep}{0pt}
}
\makeatother
%%%%%%%%%%%%%%%% Библиография: отступы и межстрочный интервал %%%%%%%%%%%%%%%% 

%%%%%%%%%%%%%%%%%%%%%%%%%%%%%% СКОБКИ СВЕТЛЫЕ %%%%%%%%%%%%%%%%%%%%%%%%%%%%%%
%\usepackage{graphicx}
\usepackage{environ}
\NewEnviron{thincases}{\scalebox{0.5}[1]{$\displaystyle
	\left\{\scalebox{2}[1]{\setlength{\arraycolsep}{3pt}% <- I did not look up the "correct" value О	
		$\displaystyle\begin{array}{ll}
			\BODY
		\end{array}$}\right.$}}%}
%%%%%%%%%%%%%%%%%%%%%%%%%%%%%% СКОБКИ  СВЕТЛЫЕ %%%%%%%%%%%%%%%%%%%%%%%%%%%%%%
